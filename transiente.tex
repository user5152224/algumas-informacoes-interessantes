%!tex

%\hsize        = 15cm
%\vsize        = 15cm
\baselineskip = 15pt
\parskip      = 20pt

% Roman
\font\xiirm  = cmr10  scaled 1200
\font\xviirm = cmr10  scaled 700
\font\xvrm   = cmr10  scaled 500

% Itálico
\font\xiimmi  = cmmi10 scaled 1200
\font\xviimmi = cmmi10 scaled 700
\font\xvmmi   = cmmi10 scaled 500

% Caligráfico
\font\xiisy  = cmsy10 scaled 1200
\font\xviisy = cmsy10 scaled 700
\font\xvsy   = cmsy10 scaled 500

\def\xiiptrm{
  \textfont0=\xiirm
  \scriptfont0=\xviirm
  \scriptscriptfont0=\xvrm
  \def\rm{\fam0\xiirm}

  \textfont1=\xiimmi
  \scriptfont1=\xviimmi
  \scriptscriptfont1=\xvmmi
  \def\it{\fam1\xiimmi}

  \textfont2=\xiisy
  \scriptfont2=\xviisy
  \scriptscriptfont2=\xvsy
  \def\cal{\fam2\xiisy}

  %TODO: Itálico para modo matemático
\rm}

\def\ra{\rightarrow}
\def\d{\wedge}

\xiiptrm

Meu prop\'osito inicial era contruir uma imers\~ao isom\'etrica de $X={\cal L}((R^m)^{p+1}, R^n)$ em $Y={\cal L}(\ell^1(R^m),R^n)$, para $p\in\omega$ fixo.

Para tal definamos a seguinte aplica\c c\~ao ${\Lambda}:X\ra Y$ por
$$
  \Lambda(T)(v)=T(v|_{p+1})\leqno(1)
$$
Sem muita dificuldade, prova-se que $\Lambda$ \'e linear e $\ker\Lambda=\{0\}$, i.e., $\Lambda$ \'e um isomorfismo linear. Ademais,
$$
  \|T\|=\sup\{\|T(u)\|:u\in(R^m)^{p+1}\,\d\,\|u\|\leq 1\}
$$
\'e igual a
$$
\|\Lambda(T)\|_1=\sup\{\|\Lambda(T)(v):v\in\ell^1(R^m)\,\d\,\|v\|_1\leq 1\},
$$
i.e., $\Lambda$ \'e uma isometria linear.



\bye

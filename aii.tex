%!latex
\documentclass[12pt, a4paper, leqno]{book}

\usepackage{%
  amsmath,
  amssymb, 
  amsthm, 
  dsfont,
  enumitem, 
  fancyhdr,
  geometry,
  hyperref,
  mathrsfs,
  newtxtext,
  newtxmath,
  setspace,
  titlesec,
  xcolor,
}

%\usepackage[natbib=true,style=authoryear,backend=bibtex,useprefix=true]{biblatex}
\usepackage{biblatex}
\addbibresource{bibliografia.bib}

\usepackage[brazil]{babel}

% Espaçamento de letra para o cabeçalho dos capítulos
\usepackage[letterspace=203]{microtype}

% Macro para criação de partes (adaptado de https://tex.stackexchange.com/questions/364330/how-to-divide-the-toc-in-two-parts)
\newcount\partnum
\partnum 1    

\def\parte#1{
  \clearpage
  \thispagestyle{empty}
  \null\vfill%
  {%
    \centering
    {\fontsize{80}{80}\selectfont \expandafter\MakeUppercase{\romannumeral\the\partnum}}% 
            
    \vspace{1cm}\noindent
    {%            
      \label{partnum:\the\partnum}%
      \hypertarget{partnum:\the\partnum}{\lsstyle\scshape\MakeUppercase{#1}}%
    }
    \par
  }
 % Need \protect to prevent breaking of commands during write process to the .aux file.
   \addtocontents{toc}{%
     \vskip 1cm
     \protect\centering\protect
     \hyperlink{partnum:\the\partnum}{\scshape Parte \expandafter\MakeUppercase{\romannumeral\the\partnum}\hskip .28cm \textbullet\hskip .28cm #1}
     \protect\par
   }
  \global\advance\partnum by 1%    
  \vfill\null%
  \newpage\thispagestyle{empty}
}

% Formatação do título dos capítulos
\titleformat{\chapter}[display]
{}%
{
   \centering\fontsize{80pt}{80pt}%
   {\selectfont\thechapter}
}%
{1em}%
{\centering\scshape\lsstyle\MakeUppercase}%
[]

% Formatação do título das se\c c\~oes e subseções
\titleformat{\section}{\centering\large\scshape}{\thesection}{1em}{}[]
\titleformat{\subsection}{\centering\scshape}{\thesubsection}{1em}{}[]

% Estilo dos cabeçalhos
\def\sc{\MakeUppercase}
\renewcommand{\chaptermark}[1]{\markboth{#1}{}}
\renewcommand{\sectionmark}[1]{\markright{\thesection\: #1}}
\renewcommand{\headrulewidth}{.1ex}
\fancypagestyle{est}{%
  \def\fontface{\bf}
  \fancyhf{}
  \fancyhead[CE]{\fontface\sc\leftmark}
  \fancyhead[CO]{\fontface\sc\rightmark}
  \fancyhead[LE,RO]{\fontface\thepage}
  \setlength{\headheight}{16pt}
  \addtolength{\topmargin}{-0.1pt}
}

% Numeração dos capítulos em algarismos romanos
\def\theHchapter{\arabic{chapter}\thechapter}
%\def\thechapter{\Roman{chapter}}

% Remover páginas em branco
\let\cleardoublepage=\clearpage

% Geometria da página
\geometry{top=2cm, bottom=2cm, left=2.5cm, right=2.5cm}

% Modificar o estilo da numeração das notas de rodapé
\renewcommand{\thefootnote}{\bfseries\arabic{footnote}}

\newcount\axiomcounter
\newcount\corollarycounter
\newcount\theoremcounter
\newcount\definitioncounter
\newcount\lemmacounter
\newcount\examplecounter

\def\prop#1#2{%
  \noindent{\bf #1.}\ {\sl#2}\medskip
}

\def\proof{%
  \bigskip
  \noindent{\it Prova.\/}
}

\def\rn#1{
  \uppercase\expandafter{\romannumeral#1}
}

\def\corollary#1{
  \medskip
  \advance\corollarycounter by 1
  \noindent{\bf Corol\'ario \expandafter{\the\corollarycounter}#1.\ }
}

\def\definition#1{
  \bigskip
  \advance\definitioncounter by 1
  \noindent{\bf Defini\c c\~ao \expandafter{\the\definitioncounter}#1.\ }
}

\def\lemma#1{
  \bigskip
  \advance\lemmacounter by 1
  \noindent{\bf Lema \expandafter{\the\lemmacounter}#1.\ }
}

\def\theorem#1{
  \bigskip
  \advance\theoremcounter by 1
  \noindent{\bf Teorema \expandafter{\the\theoremcounter}#1.\ }
}

\def\axiom#1{
  \bigskip
  \advance\axiomcounter by 1
  \noindent{\bf Axioma \expandafter{\the\axiomcounter}#1.\ }
}

\def\example#1{
  \bigskip
  \advance\examplecounter by 1
  \noindent{\bf Exemplo \uppercase\expandafter{\romannumeral\the\examplecounter}#1.\ }
}


\def\Qed{
  \hfill$\heartsuit$\bigskip
}





\def\AEx{{\bf AEx}\ }
\def\AS{{\bf AS}\ }
\def\scr#1{\mathscr{#1}}
\def\C{\mathds{C}}
\def\K{\mathds{K}}
\def\L{\mathds{L}}
\def\Q{\mathds{Q}}
\def\R{\mathds{R}}
\def\Z{\mathds{Z}}
\def\af{{\it a fortiori}}
\def\a{\wedge}
\def\[{\begin{equation}}
\def\cp{\mathbin{\neg}}
\def\]{\end{equation}}
\def\fr#1#2{\frac{#1}{#2}}
\def\frak{\mathfrak}
\def\la{\leftarrow}
\def\lge{\langle}
\def\lla{\longleftarrow}
\def\llra{\longleftrightarrow}
\def\lra{\longrightarrow}
\def\o{\;\vee\,}
\def\parallel{\mathbin{/\kern-3pt/}}
\def\pma{{\bfseries PMA}}
\def\prova{\noindent{\bfseries\scshape Prova}\quad}
\def\qed{\nopagebreak\hbox{ }\hfill\rule{.42em}{1em}\bigskip}
\def\ra{\rightarrow}
\def\res{\noindent{\itshape\bfseries Resolu\c c\~ao.\/}\ }
\def\rge{\rangle}
\def\sen{\,{\rm sen}\,}
\def\tvm{{\bf TVM}}
\def\U{{\cal U}}
\def\rf#1#2{{\rm\scshape\bfseries\hyperref[#2]{#1 \ref{#2}}}}
\def\Cap{{\textstyle\bigcap}\,}
\def\Cup{{\textstyle\bigcup}\,}
%\def\ct#1{[\hyperref[cite.#1]{#1}]}


\def\begdoc{\begin{document}}
\def\enddoc{\end{document}}

% Define um estilo de teorema
\newtheoremstyle{slanted}
{10pt}
{10pt}
{\slshape}
{}
{\bfseries}
{}
{0.5em}
{\scshape\thmname{\scshape #1} \thmnumber{\scshape #2}\bf\thmnote{ \scshape(#3)}}

\theoremstyle{slanted}

\newtheorem{axi}{Axioma}
\newtheorem{cor}{Corol\'ario}
\newtheorem{Def}{Defini\c c\~ao}
\newtheorem{exe}{Exemplo}
\newtheorem{lem}{Lema}
\newtheorem{obs}{Observa\c c\~ao}
\newtheorem{pro}{Proposi\c c\~ao}
\newtheorem{teo}{Teorema}


\begdoc
  %\tableofcontents

  % Remover paginação da tabela de conteúdos
  %\addtocontents{toc}{\protect\thispagestyle{empty}}

  \pagestyle{est}
  
  \chapter*{Prel\'udio}
    O texto que segue trata-se de um comp\^endio de assuntos que acho relevantes; portanto, decerto, o texto reflete minha subjetividade. Desta forma, este texto estar\'a em constru\c c\~ao cont\'inua, de maneira que n\~ao haver\'a vers\~ao final enquanto as circunst\^ancias da vida sobrepujarem o meu \'impeto de colecionar tais informa\c c\~oes. Alguns resultados s\~ao de minha autoria, mas certamente est\~ao enviesados por alguma obra, de maneira que n\~ao descobri a roda, mas me sustentei sobre o trabalho de v\'arios indiv\'iduos perspicazes que pela primeira vez se depararam com tais problemas. Al\'em disso, \'e imposs\'ivel colecionar tudo o que nos interessa; portanto, me ative aos que, ao menos, tive alguma ideia de demonstra\c c\~ao. 
  
  \parte{Ci\^encias formais}
  
  \iffalse % IGNORAR?

  \chapter{L\'ogica}
  Para fins de nota parafraseio sem prova e duma maneira n\~ao formal o seguinte
  \begin{teo}[\scshape Teorema da dedu\c c\~ao ou lei da dedu\c c\~ao de Tarski]
    Todo teorema de uma teoria dedutiva \'e satisfeito por qualquer modelo do sistema de axiomas desta teoria; al\'em disso, a qualquer teorema corresponde uma senten\c ca geral a qual pode ser formulada e provada dentro da estrutura da l\'ogica e que estabelece o fato que o teorema em quest\~ao \'e satisfeito para um modelo qualquer do sistema de axiomas.
  \end{teo}
  \qed
  
  \begin{Def}[Sistema de axiomas mutualmente independente]
    Um sistema de axiomas \'e dito {\rm\scshape mutualmente independente} se nenhum axioma do sistema pode ser derivado dos outros por m\'etodos de infer\^encia l\'ogica, i.e., da l\'ogica proposicional e de disciplinas precedentes\footnote{Isto est\'a em conformidade segundo Tarski. Quais seriam estas disciplinas?}.
  \end{Def}
  \chapter{Teoria dos conjuntos}
  
  Decidi abordar a constru\c c\~ao dada no ap\^endice do livro {\it General Topology} de John L. Kelley ({\it vide} pp. 250--281). Conforme atestei, o sistema empregado por Kelley, \'e uma adapta\c c\~ao do sistema de A. P. Morse, como ele bem afirma em sua obra.
  
  Nosso sistema axiom\'atico constituir\'a al\'em da l\'ogica proposicional, de objetos indefinidos chamados {\bf classes}, denotadas doravante por letras do alfabeto latino. Adicionalmente, com uma rela\c c\~ao $\in$, chamada pertin\^encia.
  
  \begin{Def}[Conjuntos]
    $$
      \forall x\bigr(\varsigma(x)\longleftrightarrow\exists y(x\in y)\bigr).
    $$
    \`A cada classe $x$ tal que $\varsigma(x)$, daremos o nome de conjunto. Desta maneira $\varsigma(x)$ se, e somente se, $x$ \'e um conjunto.
  \end{Def}
  
  \begin{axi}[II Classification axiom-scheme\;\textbullet\;Axioma-esquema da classifica\c c\~ao]
    Seja $\phi$ uma fun\c c\~ao proposicional tendo como par\^ametros as classes. Ent\~ao
    $$
      \forall y\bigl(y\in\{x:\phi(x)\}\longleftrightarrow\varsigma(y)\wedge\phi(y)\bigr).
    $$
  \end{axi}
  
  Enfatizo que os objetos da teoria de classes, resumem-se, como \'e esperado, \`a classes. Portanto, `$\{x:\phi(x)\}$' denota uma classe, da\'i o nome de {\it classifier} ou classificador.
  
  \begin{Def}
    $\lge\phi\rge=\{x:\phi(x)\}.$
  \end{Def}
  
  
  \begin{axi}[I Axiom of extent\;\textbullet\;\AEx: Axioma da extensionalidade]
    $$
      \forall x,y\bigl(x=y\longleftrightarrow\forall z(z\in x\longleftrightarrow z\in y)\bigr).
    $$
  \end{axi}
  
  Friso categoricamente que uma prova rigorosa da igualdade de classes, requer o uso expl\'icito do axioma da extensionalidade \AEx. Portanto, a rigor, uma cadeia de igualdades de classes dadas por classificadores, mesmo que \'obvia para leitores maduros, n\~ao caracteriza, segundo minha visão, uma prova. Enfatizo este ponto, pois em est\'agios anteriores na confec\c c\~ao deste comp\^endio eu usei tais m\'etodos.
  
  \begin{teo}
    $$
      \forall x\bigl(x=\{y:y\in x\}\bigr).
    $$
  \end{teo}
  
  \prova 
  $$
    \forall z\bigl(z\in\{y:y\in x\}\longleftrightarrow z\in x\bigr).
  $$
  \qed
  
  \begin{Def}
    $x\cup y=\{z:z\in x\vee z\in y\}$.
  \end{Def}
  
  \begin{Def}
    $x\cap y=\{z:z\in x\wedge z\in y\}$.
  \end{Def}
  
  Aos s\'imbolos $x\cup y$ e $x\cap y$, d\'a-se o nome de uni\~ao e intersec\c c\~ao de $x$ com $y$, respectivamente. 
  
  \begin{teo}
    $$
      \forall x,y,z\Bigl(x\cap(y\cup z)=(x\cap y)\cup(x\cap z)\;\a\; x\cup(y\cap z)=(x\cup y)\cap(x\cup z)\Bigr)
    $$
  \end{teo}
  
  \prova Vamos provar a segunda proposi\c c\~ao. Sejam $x,y$ e $z$ classes, notemos que
  \[
    \begin{split}
      \forall w\bigl(w\in x\cup(y\cap z) &\llra w\in x\vee w\in y\cap z\cr
                                         &\llra w\in x\vee(w\in y\wedge w\in z)\cr
                                         &\llra (w\in x\vee w\in y)\wedge(w\in x\vee w\in z)\cr
                                         &\llra w\in(x\cup y)\wedge w\in(x\cup z)\cr
                                         &\llra w\in(x\cup y)\cap(x\cup z)\bigr).\cr
    \end{split}
  \]
  A outra prova \'e inteiramente an\'aloga. De fato, decorre das propriedades dos conectivos `$\wedge$' e `$\vee$', a conjun\c c\~ao e disjun\c c\~ao l\'ogica, respectivamente.
  \qed
  
  \begin{Def}
    $$
      \forall x,y(x\notin y\longleftrightarrow \neg(x\in y)).
    $$
  \end{Def}
  
  \begin{Def}
    $\cp x=\{y:y\notin x\}$.
  \end{Def}
  
  A classe $\cp x$ chama-se complemento absoluto de $x$.
  
  \begin{teo}
    Seja $\phi$ uma fun\c c\~ao proposicional cujos par\^ametros sejam classes. Ent\~ao $\cp\lge\phi\rge=\lge\neg\phi\rge$.
  \end{teo}
  
  \prova
  $$
    \forall x\bigl(x\in\cp\lge\phi\rge\llra\varsigma(x)\wedge\neg\phi(x)\llra x\in\lge\neg\phi\rge\bigr).
  $$
  \qed
  
  \begin{teo}
    $$
      \forall x\bigl(\cp(\cp x)=x\bigr).
    $$
  \end{teo}
  
  \prova
  \[
    \forall y\bigl(y\in\cp(\cp x) \llra\neg(y\in\cp x)\llra \neg\bigr(\neg(y\in x)\bigr)\llra y\in x\bigl)
  \]
  \qed
  
  \begin{teo}[Leis de De Morgan]
    $$
      \forall x,y\bigl(\cp(x\cup y)=\cp x\cap\cp y\;\wedge\;\cp(x\cap y)=\cp x\cup\cp y\bigr)
    $$
  \end{teo}
  
  \prova A prova segue diretamente da defini\c c\~ao de $\cup$ e $\cap$, e das leis de De Morgan para os conectivos $\vee$ e $\wedge$.\qed
  
  \begin{Def}
    $x\cp y=x\cap\cp y$.
  \end{Def}
  
  Ao s\'imbolo `$x\cp y$' d\'a-se o nome de diferen\c ca de $x$ e $y$ ou complemento de $y$ relativo a $x$.
  
  \begin{teo}
    $$
      \forall x,y,z\bigl(x\cap(y\cp z)=(x\cap y)\cp z\bigr)
    $$
  \end{teo}
  \prova Provar.\qed
  
  \begin{Def}
    Seja $\phi$ uma contradi\c c\~ao qualquer. Definimos
    $\emptyset=\lge\phi\rge.$
  \end{Def}
  
  Observe que esta defini\c c\~ao independe da contradi\c c\~ao. Com efeito, temos o
  
  \begin{teo}
    Se $\phi$ e $\psi$ s\~ao duas contradi\c c\~oes, ent\~ao $\lge\phi\rge=\lge\psi\rge$.
  \end{teo}
  \prova Com efeito,
  \[
    \forall x\bigl(x\in\lge\phi\rge\llra\varsigma(x)\a\phi(x)\llra\varsigma(x)\a\psi(x)\llra x\in\lge\psi\rge\bigr),
  \]
  pois ambos os membros da bicondicional s\~ao falsos, logo a equival\^encia \'e v\'alida e, consequentemente o quantifica\c c\~ao \'e verdadeira.\qed
  \bigskip
  
  A classe $\emptyset$ \'e chamada de classe nula ou vazia.
  
  \begin{teo}
    \[
      \forall x(x\notin\emptyset).
    \]
  \end{teo}
  \prova 
  \[
    \forall x(x\in\emptyset\longleftrightarrow\varsigma(x)\wedge x\neq x).
  \]
  Como o lado direito \'e trivialmente falso pela defini\c c\~ao de igualdade, decorre que $\neg(x\in\emptyset)$, ou equivalentemente $x\notin\emptyset$ \'e verdadeira para todo $x$.\qed
  
  \begin{teo}
    \[
      \forall x(x\cup\emptyset=x\;\wedge\; x\cap\emptyset=\emptyset).
    \]
  \end{teo}
  
  \prova Seja $x$ uma classe, temos em conformidade
  \[
    \forall y\bigl(y\in x\cup\emptyset\longleftrightarrow \varsigma(y)\wedge(y\in x\vee y\in\emptyset)\longleftrightarrow\varsigma(y)\wedge y\in x\longleftrightarrow y\in x\bigr)
  \]
  o que segundo o \AEx segue-se a identidade. 
  
  Analogamente,
  \[
    \forall y\bigl(y\in x\cap\emptyset\longleftrightarrow \varsigma(y)\wedge(y\in x\a y\in\emptyset)\longleftrightarrow\varsigma(y)\a y\in\emptyset\longleftrightarrow y\in\emptyset\bigr)
  \]
  novamente pelo \AEx infere-se a igualdade.\qed
  
  \begin{Def}
    \[
      {\cal U}=\cp\emptyset
    \]
  \end{Def}
  
  \begin{teo}
    \[
      \forall x\bigl(x\in{\cal U}\llra\varsigma(x)\bigr).
    \]
  \end{teo}
  
  \prova
  \[
    \forall x\bigr(x\in{\cal U}\llra \varsigma(x)\a x\not\in\emptyset\llra \varsigma(x)\bigr).
  \]
  \qed
  
  \begin{teo}
    Seja $\phi$ uma tautologia. Ent\~ao $\lge\phi\rge={\cal U}$.
  \end{teo}
  
  \prova Notemos que 
  \[
    \forall x\bigl(x\in\lge\phi\rge\llra\varsigma(x)\wedge\phi(x)\llra\varsigma(x)\llra x\in{\cal U}\bigr),
  \]
  de \AEx decorre o teorema. Outra maneira seria usar a hip\'otese que $\neg\phi$ \'e uma contradi\c c\~ao, e concluir de
  \[
    \lge\phi\rge=\cp\lge\neg\phi\rge=\cp\emptyset={\cal U}.
  \]
  \qed
  
  \begin{teo}
    \[
      \forall x\bigl(x\cup\U=\U\a x\cap\U=x\bigr)
    \]
  \end{teo}
  \prova 
  \[
    \forall y\bigl(y\in x\cup\U\llra y\in x\o y\in \U\llra y\in\U\bigr)
  \]
  segue pelo \AEx.
  
  Seguidamente,
  \[
    \forall y\bigl(y\in x\cap\U\llra y\in x\a y\in\U \llra y\in x\bigr)
  \]
  novamente decorre por \AEx.\qed
  
  \begin{Def}
    \[
      \Cap x=\bigl\{z:\forall y(y\in x\lra z\in y)\bigr\}.
    \]
  \end{Def}
  \begin{Def}
    \[
      \Cup x=\bigl\{z:\exists y(y\in x\a z\in y)\bigr\}.
    \]
  \end{Def}
  
  A classe $\Cap x$ \'e a interse\c c\~ao dos membros de $x$ e, a classe $\Cup x$ \'e a uni\~ao dos membros de $x$.
  
  \begin{teo}
    \[
      \Cap\emptyset=\U\a\Cup\emptyset=\emptyset.
    \]
  \end{teo}
  \prova
  \[
    \forall x\Bigl(x\in\Cap\emptyset\llra\varsigma(x)\a\forall y(y\in\emptyset\llra x\in y)\llra x\in\U\Bigr),
  \]
  pois $\phi$ definida por
  \[
    \phi(x)\llra\forall y(y\in\emptyset\llra x\in y)
  \]
  \'e uma tautologia.
  
  Em seguida,
  \[
    \forall x\bigl(x\in\Cup\emptyset\llra\varsigma(x)\a\exists(y\in\emptyset\a x\in y)\llra x\in\emptyset\bigr),
  \]
  porquanto, $\psi$ definida por
  \[
    \psi(x)\llra\exists y(y\in\emptyset\a x\in y)
  \]
  \'e uma contradi\c c\~ao.\qed
  \begin{Def}
    \[
      \forall x\bigl(x\subset y\llra\forall z(z\in x\lra z\in y)\bigr).
    \]
  \end{Def}
  
  Uma classe $x$ \'e uma subclasse de $y$, ou est\'a contida em $y$, ou $y$ contem $x$, se, e somente se, $x\subset y$.
  
  \begin{teo}[26 Theorem]
    \[
      \forall x(\emptyset\subset x\a x\subset\U)
    \]
    \label{T26}
  \end{teo}
  \prova Seja $x$ uma classe. Temos primeiramente
  \[
    \forall y(y\in\emptyset\lra y\in x),
  \]
  pois o antecedente da condicional \'e sempre falso. O que prova a primeira inclus\~ao.
  
  Por outro, lado temos
  \[
    \forall y(y\in x\lra\varsigma(y)\lra y\in\U).
  \]
  o que conclui a prova.\qed
  
  \begin{teo}[27 Theorem]
    \[
      \forall x,y(x=y\llra x\subset y\a y\subset x).
    \]
    \label{T27}
  \end{teo}
  \prova Sejam $x$ e $y$ classes quaisquer 
  \[
    \begin{split}
      x\subset y\a y\subset x &\llra\forall z\bigl((z\in x\lra z\in y)\a(z\in y\lra z\in x)\bigr)\cr
                              &\llra\forall z(z\in x\llra z\in y)\cr
                              &\llra x=y.
    \end{split}
  \]
  \qed
  
  \begin{teo}[28 Theorem]
    \[
      \forall x,y,z(x\subset y\a y\subset z\lra x\subset z)
    \]
    \label{T28}
  \end{teo}
  \prova Sejam $x,y,z$ classes quaisquer. Temos
  
  \[
    \begin{split}
      \bigl( x\subset y\a y\subset z &\lra \forall w(w\in x\lra w\in y\a w\in y\lra w\in z)\cr
                                     &\lra \forall w(w\in x\lra w\in z)\cr
                                     &\lra x\subset z\bigr)\cr
    \end{split}
  \]
  \qed
  
  \begin{teo}
    \[
      \forall x,y,z(x\subset y\lra x\cap z\subset y\cap z\a x\cup z\subset y\cup z)
    \]
    \label{monoto}
  \end{teo}
  \prova Sejam $x,y,z$ classes quaisquer. Primeiramente,
  \[
    \forall w(w\in x\cap z\lra w\in x\a w\in z\lra w\in y\a w\in z\lra w\in y\cap z).
  \]
  Segundo e, por fim,
  \[
    \forall w(w\in x\cup z\lra w\in x\o w\in z\lra w\in y\o w\in z\lra w\in y\cup z).
  \]
  \qed
  
  %\begin{teo}
  %  \[
  %    \forall w,x,y,z(w\subset y\a x\subset z\lra w\cup x\subset y\cup z\a w\cap x\subset y\cap z)
  %  \]
  %\end{teo}
  %\prova Sejam $w,x,y,z$, classes arbitr\'arias, segue-se do {\bf Teorema \ref{monoto}} que
  %\[
  %  w\subset y\a x\subset z\lra w\cap x\subset w\cap y\a x\cap y\subset y\cap z\lra w\cap x\subset y\cap z
  %\]
  %e
  %\[ 
  %  w\subset y\a x\subset z\lra w\cup x\subset w\cup y\a x\cup y\subset y\cup z\lra w\cup x\subset y\cup z.
  %\]
  %\qed
  
  \begin{teo}
    \[
      \forall x,y(x\subset y\llra x\cup y=y)
    \]
  \end{teo}
  \prova Sejam $x,y$ classes arbitr\'arias, segue-se
  \[
   x\subset y\a y\subset x\cup y\llra x\cup y\subset y\a y\subset x\cup y\llra x\cup y=y.
  \]
  \qed
  
  \begin{teo}[\scshape 30 Theorem]
    \[
      \forall x,y(x\subset y\llra x\cap y=x)
    \]
  \end{teo}
  \prova Dadas as classes $x$ e $y$, temos
  \[
    x\subset y\a x\cap y\subset x\llra x\subset x\cap y\a x\cap y\subset x\llra x\cap y=x.
  \]
  \qed
  
  \begin{teo}[\scshape 31 Theorem]
    \[
      \forall x,y\Bigl(x\subset y\lra \Cup x\subset \Cup y\a\Cap y\subset\Cap x\Bigr).
    \]
  \end{teo}
  \prova Sejam $x,y,z$ classes arbitr\'arias, tais que $x\subset y$. Primeiro temos
  \[
    z\in\Cup x \lra\exists w(w\in x\a z\in w)\lra\exists w(w\in y\a z\in w)\lra z\in\Cup y.
  \]
  Segundo,
  \[
    z\in\Cap y \lra\forall w(w\in y\lra z\in w)\lra\exists w(w\in x\lra z\in w)\lra z\in\Cap x.
  \]
  \qed
  \begin{teo}[\scshape 32 Theorem]
    \[
      \forall x,y\Bigl(x\in y\lra x\subset\Cup y\a \Cap y\subset x\Bigr)
    \]
    \label{T32}
  \end{teo}
  \prova Sejam $x,y,z$, classes aleat\'orias tais que $x\in y$. Inicialmente, temos
  \[
    z\in x\lra \exists w(w\in y\a z\in w)\lra z\in\Cup y.
  \]
  Finalmente,
  \[
    z\in\Cap y\lra\forall w(w\in y\lra z\in w)\lra z\in x.
  \]
  \qed
  
  \section{Exist\^encia de conjuntos}
  
  \begin{axi}[III Axiom\;\textbullet\;AS: Axioma de subconjuntos]
    \[
      \forall x\Bigl(\varsigma(x)\lra\exists y\bigl(\varsigma(y)\a\forall z(z\subset x\lra z\in y)\bigr)\Bigr).
    \]
    \label{AxSub}
  \end{axi}
  
  \begin{teo}[33 Theorem]
    \[
      \forall x,z\bigl(\varsigma(x)\a z\subset x\lra\varsigma(z)\bigr).
    \]
    \label{T33}
  \end{teo}
  \prova Sejam as classes $x$ e $z$, tais $\varsigma(x)$ e $z\subset x$, tem se
    \[
     \varsigma(x)\mathop{\lra}^{\AS}\exists y(w\subset x\lra w\in y)\lra z\in w\lra\varsigma(z).
    \]
  \qed
  
  \begin{teo}[34 Theorem]
    \[
      \emptyset = \Cap\U\a\U=\Cup\U.
    \]
  \end{teo}
  \prova Primeiramente suponhamos por {\it reductio ad absurdum} que exista $x\in\Cap\U$, ent\~ao $\varsigma(x)$, pelo \rf{Teorema}{T26} segue-se $\emptyset\subset x$; assim necessariamente $\varsigma(\emptyset)$, ou equivalentemente $\emptyset\in\U$. Pelo \rf{Teorema}{T32} tem-se que $\Cup\U\subset\emptyset$, novamente pelo \rf{Teorema}{T26} temos $\emptyset\subset\Cap\U$, logo, pelo \rf{Teorema}{T27} e da suposi\c c\~ao inicial decorre que $x\in\Cap\U=\emptyset$, o que \'e uma contradi\c c\~ao. Consequentemente, n\~ao \'e o caso que exista $x\in\Cap\U$, i.e., $\Cap\U=\emptyset$[\footnote{Segui a mesma linha de racioc\'inio de Kelley, mas com a minhas adapta\c c\~oes.}.
  
  Agora, seja $x\in\U$, pelo \AS existe $y\in\U$, tal que $x\in y$, pois $x\subset x$, pela defini\c c\~ao de $\Cup\U$, segues-se que $x\in \Cup\U$, consequentemente $\U\subset\Cup\U$. Por outro lado, pelo \rf{Teorema}{T26}, temos $\Cup\U\subset\U$, consequentemente pelo \rf{Teorema}{T27}, $\Cup\U=\U$.\qed
  
  \begin{teo}[35 Theorem]
    Para toda classe $x$, se $x\neq\emptyset$, ent\~ao $\varsigma(\Cap x)$.
    \label{T35}
  \end{teo}
  \prova Se $x\neq\emptyset$, ent\~ao existe $y\in x$, logo $\varsigma(y)$ e pelo \rf{Teorema}{T32} tem-se $\Cap x\subset y$, da\'i e do \rf{Teorema}{T32}, decorre que $\varsigma(\Cap x)$.\qed 
  
  \begin{Def}
    $2^x=\{y:y\subset x\}$.
  \end{Def}
  
  \begin{teo}[37 Theorem]
    $2^\U=\U.$
    \label{T37}
  \end{teo}
  \prova Certamente pelo \rf{Teorema}{T26} segue que $2^\U\subset\U$. Agora, dado $x\in\U$, ent\~ao evidentemente $x\subset\U$, consquentemente $x\in 2^\U$. Pelo \rf{Teorema}{T27}, conclu\'imos que $2^\U=\U$.\qed
  
  \begin{teo}[38 Theorem]
    Se $\varsigma(x)$, ent\~ao $\varsigma(2^x)$. 
  \end{teo}
  \prova Se $\varsigma(x)$, pelo \AS existe $y\in\U$, tal que para todo $z$, se $z\subset y$, ent\~ao $z\in y$. Como consequ\^ecnia, $2^x\subset y$, pelo \rf{Teorema}{T33} segue-se que $\varsigma(2^x)$.\qed
  
  \begin{teo}[39 Theorem]
    $\neg\varsigma(\U)$.
  \end{teo}
  \prova Consideremos a classe $\mathcal{R}=\{x:x\notin x\}$[\footnote{O ${\cal R}$ \'e em homenagem a Bertrand Russel, um dos primeiros a descobrir paradoxos na teoria ing\^enia dos conjuntos, criada pelo matem\'atico eminente Georg Cantor.}. Provemos que $\neg\varsigma(\mathcal{R})$, i.e., $\mathcal{R}$. Para tanto, suponhamos por {\it reductio ad absurdum} que $\varsigma(\mathcal{R})$, temos ent\~ao a equival\^encia
  \[
    \varsigma(\mathcal{R})\a \mathcal{R}\in\mathcal{R}\;\;\hbox{se, e somente se,}\;\;\varsigma(\mathcal{R})\a \mathcal{R}\notin\mathcal{R},
  \]
  o que \'e uma inconsist\^encia, i.e., uma contradi\c c\~ao. Logo, $\neg\varsigma(\mathcal{R})$. Ademais, da pr\'opria defini\c c\~ao de classificadores $\mathcal{R}\subset\U$, como $\mathcal{R}$ \'e uma classe pr\'opria, necessariamente $\U$ tamb\'em o ser\'a. Com efeito, suponha por {\it reductio ad absurdum} que $\varsigma(\U)$, pelo \rf{Teorema}{T33}, inferir\'iamos que $\varsigma(\mathcal{R})$, o que \'e uma contradi\c c\~ao com o argumento anterior.\qed

  \begin{teo}[Defini\c c\~ao por indu\c c\~ao]
    Sejam $X$ um conjunto e $a\in X$. Suponha que exista uma fun\c c\~ao $F:\bigcup_{p\in\omega}X^{p+1}\lra X$. Ent\~ao existe uma \'unica fun\c c\~ao $f\in X^\omega$, tal que $f(0)=a$ e $f(p+1)=F(f|_{p+1})$, para todo $p\in\omega$.
  \end{teo}
  \prova
  Primeiramente, para cada $p\in\omega$, definamos
  \[
    \mathfrak{F}_p=\bigl\{g:g\in X^{p+2}\,\a\,g(0)=a\,\a\,\forall q\bigl(q\in p+1\lra g(q+1)=F(g|_{q+1})\bigr\}.
  \]
  Seguidamente definamos,
  \[
    C=\bigl\{p:p\in\omega\,\a\,\exists! g(g\in\mathfrak{F}_p)\bigr\}.
  \]
  Primeiro, considere $g\in X^{2}$, tal que $g(0)=a$ e $g(1)=F(g|_1)$. Naturalmente se $\gamma\in{\mathfrak F}_0$, ent\~ao $\gamma(0)=a=g(0)$, logo, $\gamma|_1=g|_1$ e consequentemente $\gamma(1)=F(\gamma|_1)=F(g|_1)=g(1)$, o que prova que $0\in C$. Em seguida, suponhamos que $p\in\omega$, segue que existe uma \'unica $h\in{\mathfrak F}_p$, defina $g\in X^{p+3}$, tal que $g|_{p+2}=h$ e $g(p+2)=F(g|_{p+2})$. Seja agora $\gamma\in{\frak F}_{p+1}$, pela unicidade de $h$ segue-se que $\gamma|_{p+2}=h=g|_{p+2}$, por conseguinte $\gamma(p+2)=F(h)=g(p+2)$, que por sua vez, acarreta $\gamma=g$, o que prova a unicidade de $g$, pois $p\in\omega$. Pelo princ\'ipio de indu\c c\~ao matem\'atica $C=\omega$. 

  Em verdade provamos que ${\frak F}_p$ \'e unit\'ario para todo $p\in\omega$. Digamos que $f_p\in{\frak F}_p$ e definamos $f=\bigcup_{p\in\omega} f_p$. Suponhamos que $(a,b),(a,c)\in f$. Naturalmente, existem $p,q\in\omega$ tais que $(a,b)\in f_p$ e $(a,c)\in f_q$, os quais sem perda de generalidade podemos supor $p\leq q$. Como $f_q|_{p+2}\in{\frak F}_p$, podemos inferir que $f_q|_{p+2}=f_p$; assim $b=f_p(a)=f_q(a)=c$, o que prova que $f\in X^\omega$. Agora seja $p\in\omega$, notemos que $f|_{p+2}=f_p$, consequentemente $f(0)=a$ e $f(p+1)=F(f|_{p+1})$, portanto $f$ satisfaz as hip\'oteses do teorema, i.e., provamos a exist\^encia. Ademais, seja $\phi\in X^\omega$ uma outra fun\c c\~ao que satisfaz as estipula\c c\~oes do teorema, temos que para todo $p\in\omega$, vale $f|_{p+2},\phi|_{p+2}\in{\frak F}_p$, por conseguinte $f|_{p+2}=\phi|_{p+2}$, que por sua vez implica que $f=\phi$, pois $p\in\omega$ \'e arbitr\'ario, provando portanto a unicidade de $f$.\qed
  
  \vfill\break


  \theorem{\ (Bernstein-Cantor-Schr\"oder)}{Sejam $X$ e $Y$ conjuntos tais que existe uma inje\c c\~ao pr\'opria de $X$ em $Y$ e vice-versa. Ent\~ao $X$ \'e equivalente a $Y$.}
  
  \proof Supomos que $f(X)\subset Y$ e $g(Y)\subset X$ (aqui $\subset$ \'e a inclus\~ao pr\'opria). Destas inclus\~oes deriva-se facilmente a proposi\c c\~ao
  $$
    (\Delta)\qquad\forall i\Bigl(i\in N\longrightarrow(g\!f)^i(X)\supset(g\!f)^i\bigl(g(Y)\bigr)\supset(g\!f)^{i+1}(X)\Bigr).
  $$
  Para cada $i\in N$, fa\c camos:
  $$
    A_{2i}=(g\!f)^i(X)\quad\hbox{e}\quad A_{2i+1}=(g\!f)^i\bigl(g(Y)\bigr)
  $$
  em vista de $(\Delta)$ decorre que $\{A_i:i\in N\}$ \'e uma fam\'\i lia de conjuntos estritamente decrescente.
  
  Sejam agora
  $$
    C_i=(g\!f)^i\bigr(X\setminus g(Y)\bigr)\quad\hbox{e}\quad D_i=(g\!f)^ig\bigl(Y\setminus f(X)\bigr)
  $$
  com $i\in N$. 
  
  Definamos
  $$
    A=\bigcap_{i\in N}A_i,\quad C=\bigcup_{i\in N}C_i\quad e\quad D=\bigcup_{i\in N}D_i.
  $$
  Estes s\~ao disjuntos aos pares. De fato, inicialmente provemos que $C\cap D=\emptyset$. Note que
  $$
    \forall i,j\bigl(i,j\in N\longrightarrow C_i\cap D_j=\emptyset\bigr),
  $$
  pois dados $i,j\in N$, temos
  $$
    C_i\cap D_j=A_{2i}\cap A_{2j+1}\setminus\bigl(A_{2i+1}\cup A_{2(j+1)}\bigr)\subset A_a\setminus A_b=\emptyset,
  $$
  sendo
  $$
    a=\max\{2i, 2j\}\quad\hbox{e}\quad b=\min\{2(i+1), 2(j+1)\}.
  $$
  uma vez que $A_b\subset A_a$, como consequ\^encia $C\cap D=\emptyset$.
  
  Seguidamente, dados $i,j\in N$, obtem-se
  $$
    A\cap (C_i\cup D_j)=A\setminus\bigl(A_{2i+1}\cap A_{2(j+1)}\bigr)=\emptyset,
  $$
  consequentemente $A\cap(C\cup D)=\emptyset$.
  
  Afirmo que 
  $$
    X=A\cup C\cup D
  $$
  Decerto, observe que se $x\in X\setminus(C\cup D)$, ent\~ao necessariamente $x\in A_i$ para todo $i\in N$, pois, se existir $i\in N$ tal que $x\notin A_i$, ent\~ao consideremos $m=\min\{i\in N:x\notin A_i\}$, \'e certo que $m>0$. Assim, existe $l\in N$, tal que $l+1=m$, e consequentemente $x\in A_l\setminus A_{l+1}\subset C\cup D$, uma contradi\c c\~ao.
  
  De maneira inteiramente an\'aloga para cada $i\in N$, fa\c camos:
  $$
    B_{2i}=(f\!g)^i(Y)\quad\hbox{e}\quad B_{2i+1}=(f\!g)^i\bigl(f(X)\bigr);
  $$
  $$
    E_i=(f\!g)^i\bigr(Y\setminus f(X)\bigr)\quad\hbox{e}\quad F_i=(f\!g)^if\bigl(X\setminus g(Y)\bigr).
  $$
  Prova-se sem dificuldades que
  $$
    Y=B\cup E\cup F,
  $$
  em que
  $$
    B=\bigcap_{i\in N}B_i,\quad E=\bigcup_{i\in N}E_i\quad e\quad F=\bigcup_{i\in N}F_i.
  $$
  Em verdade, basta trocar os pap\'eis de $f$ com $g$ e $X$ com $Y$, concomitamente.
  
  Agora observemos que por indu\c c\~ao matem\'atica decorre que
  $$
    \forall i\bigl(i\in N\longrightarrow \phi(\gamma\phi)^i=(\phi\gamma)^i\phi\bigr).
  $$
  para quaisquer que sejam as fun\c c\~oes $\phi$ e $\gamma$, em que a composi\c c\~ao fa\c ca sentido. Em verdade, para $i=0$ temos necessariamente
  $$
    \phi(\gamma\phi)^0=\phi=(\phi\gamma)^i\phi.
  $$
  Suponhamos por hip\'otese de indu\c c\~ao que a propriedade seja v\'alida para $i\in N$, notemos que
  $$
    \phi(\gamma\phi)^{i+1}=\phi(\gamma\phi)^i(\gamma\phi)=(\phi\gamma)^i(\phi\gamma)\phi=(\phi\gamma)^{i+1}\phi
  $$
  o que prova que \'e v\'alida para $i+1$, conclu\'\i mos por indu\c c\~ao matem\'atica que a propriedade \'e v\'alida para todo $i\in N$.
  
  Posteriormente, em vista do resultado anterior observemos que para cada $i\in N$
  $$
    A_{2i+1}=(g\!f)^ig(Y)=g(f\!g)^i(Y)=g(B_{2i})
  $$
  e
  $$
    B_{2i+1}=(f\!g)^if(X)=f(g\!f)^i(X)=f(A_{2i}).
  $$
  Em consequ\^encia para todo $i\in N$
  $$
    B_{2i}=g^{-1}(A_{2i+1})\quad\hbox{e}\quad B_{2i+1}=f(A_{2i}).
  $$
  Em seguida, definamos $\varphi:X\rightarrow Y$ por
  
  $$
    \varphi(x)=
    \left\{
    \begin{matrix}
      \hfill f(x),      & x\in A\cup C;\hfill   \cr
      \hfill g^{-1}(x), & x\in D.\hfill \cr
    \end{matrix}
    \right.
  $$
  Pela pr\'opria defini\c c\~ao $\varphi$ \'e injetiva.
  
  Imediatamente observe que para todo $i\in N$
  $$
    \varphi(C_i)=f\Bigl((g\!f)^i\bigl(X\setminus g(Y)\bigr)\Bigr)=(f\!g)^if\bigl(X\setminus g(Y)\bigr)=F_i
  $$
  e
  $$
    \varphi(D_i)=g^{-1}\Bigl((g\!f)^ig\bigl(Y\setminus f(X)\bigr)\Bigr)=(f\!g)^i\bigl(Y\setminus f(X)\bigr)=E_i.
  $$
  Al\'em do mais, para todo $i\in N$, temos
  $$
    \varphi(A_{2i})=f(A_{2i})=f(g\!f)^i(X)=(f\!g)^if(X)=B_{2i+1}
  $$
  e
  $$
    \varphi(A_{2i+1})=f(A_{2i+1})=f(g\!f)^ig(Y)=(f\!g)^if\!g(Y)=B_{2(i+1)}.
  $$
  Em suma,
  $$
    \forall i \bigl(i\in N\longrightarrow\varphi(A_i)=B_{i+1}\wedge\varphi(C_i)=F_i\wedge\varphi(D_i)=E_i\bigr).
  $$
  Conformemente
  $$
    \varphi(A)=B,\quad\varphi(C)=F\quad\hbox{e}\quad\varphi(D)=E.
  $$
  {\it a fortiori}
  $$
    \varphi(X)=\varphi(A\cup C\cup D)=\varphi(A)\cup\varphi(C)\cup\varphi(D)=B\cup E\cup F=Y,
  $$
  i.e., $X\sim Y$.\qed
  \bigskip
  
  \medskip
  \lemma{}{\sl%
    Sejam $f:X\rightarrow Y$ e $g:Y\rightarrow X$. Ent\~ao existe um conjunto $A\subseteq X$, tal que
    $$
      X\setminus A=g\bigl(Y\setminus f(A)\bigr).
    $$
  }
  
  
  \noindent{\it Prova alternativa do teorema.\ } Existe uma sugest\~ao dada pelo Halmos que embora esteja errada, ela me sujeriu um processo simples de demonstra\c c\~ao. Apesar de tudo, n\~ao consegui determinar quais propriedades m\'\i nimas $f$ e $g$ devem possuir para garantir o resultado auxiliar. Em vista desse impasse, nos ateremos ao caso em que ambas $f$ e $g$ sejam injetivas.
  
  \bigskip
  
  \noindent{\bf Digress\~ao a respeito de \'algebras de Boole.} Segundo o livro {\sl Introduction to Boolean Algebras} de Halmos e Givant, uma \'algebra booleana \'e um conjunto $B$ munido de duas opera\c c\~oes bin\'arias $\cap$ e $\cup$, uma opera\c c\~ao un\'aria $'$ e dois elementos distintos $0$ e $1$, satisfazendo os seguintes axiomas:
  
  \bigskip
  \rn{1.}\hfill $0'=1\quad \wedge\quad 1'=0$;\hfill\hbox{}
  
  \rn{2.}\hfill $x\cap 0\quad \wedge\quad x\cup1=1$;\hfill\hbox{}
  
  \rn{3.}\hfill $x\cap1=x\cup0=x$;\hfill\hbox{}
  
  \rn{4.}\hfill $x\cap x'=0\quad\wedge\quad x\cup x'=1$;\hfill\hbox{}
  
  \rn{5.}\hfill $x''=x$;\hfill\hbox{}
  
  \rn{6.}\hfill $x\cap x=x\cup x=x$;\hfill\hbox{}
  
  \rn{7.}\hfill $(x\cap y)'=x'\cup y'$;\hfill\hbox{}
  
  \rn{8.}\hfill $x\cap y=y\cap x\quad\wedge\quad x\cup y=y\cup x$;\hfill\hbox{}
  
  \rn{9.}\hfill $x\cap(y\cap z)=(x\cap y)\cap z\quad\wedge\quad x\cup(y\cup z)=(x\cup y)\cup z$;\hfill\hbox{}
  
  \rn{10.}\hfill $x\cap(y\cup z)=(x\cap y)\cup(x\cap z)\;\wedge\;x\cup(y\cap z)=(x\cup y)\cap(x\cup z)$;\hfill\hbox{}
  
  
  \bigskip
  \theorem{\ (\'Algebra de Boole dada na introdu\c c\~ao do livro {\it Introduction to Mathematical Logic} de Elliot Mendelson, p\'agina xxiii)}{\sl Sejam $B$ um conjunto com pelo menos dois elementos, $\cap\in B^{B^2}$ uma opera\c c\~ao bin\'aria e $'\in B^B$ uma opera\c c\~ao un\'aria.
  
  Admitiremos que $\cap$ e $'$ satisfazem os seguintes axiomas para quaisquer $x,y,z\in B${\rm :}
  
  1. $x\cap y=y\cap x$;
  
  2. $(x\cap y)\cap z\longleftrightarrow x\cap(y\cap z)$;
  
  3. $(x\cap y'=z\cap z')\longleftrightarrow x\cap y=x$.
  \medskip
  
  Ademais defina 
  $$
    \forall x,y\,\bigl(x\cup y=(x'\cap y')'\bigr)\quad\hbox{Def.}
  $$
  
  \noindent Ent\~ao  $\langle B,\cap, \cup, '\rangle$ \'e uma \'algebra booleana se, e somente se, os axiomas 1--3 forem satisfeitos.
  }
  
  \proof Primeiramente, provemos que $x\cap x=x$, para todo $x\in B$. Suponhamos que exista $x\in B$ tal que $x\cap x\neq x$, da equival\^encia no item 3, fixado $y=x$ temos
  $$
    \forall z\,\bigl(z\in B\longrightarrow (x\cap x\neq x\longleftrightarrow x\cap x'\neq z\cap z')\bigr)
  $$
  o que \'e falso para $z=x$, consequentemente a identidade $x=x\cap x$ \'e v\'alida para todo $x\in B$. Novamente, da equival\^encia no item 3 temos que fixados $x\in B$ e $y=x$, decorre que
  $$
    \forall z(x\cap x=x\longleftrightarrow x\cap x'=z\cap z')
  $$
  Definamos $0=x\cap x'$.% e $1=0'$.
  
   Em seguida, observemos
  $$
    \forall z\,\bigl(z\cap0=z\cap(z\cap z')=(z\cap z)\cap z'=z\cap z'=0\bigr)
  $$
  % provemos em seguida que $0''=0$. Para tanto observemos as equival\^encias, ambas dadas pelo item 3:
  %$$
  %  0\cap0''=0\longleftrightarrow 0'\cap0=0
  %$$
  %e
  %$$
  %  0\cap0''=0''\longleftrightarrow0'\cap0''=0.
  %$$
  %Como consequ\^encia 
  %$$
  %  0=0\cap0''=0''
  %$$
  %{\it a fortiori} $1'=0$. Portanto,
  %$$
  %  \forall z\bigl(z\in B\longleftrightarrow (z\cap 1'=z\cap0=0\longleftrightarrow z\cap1=z)\bigr)
  %$$
  
  Seguidamente observemos, tamb\'em em decorr\^encia do 3 item do rol de axiomas, que dado $z$ arbitrariamente temos
  $$
    z'''=z'\cap z'''\longleftrightarrow z'''\cap z''=0,
  $$
  $$
    (\Gamma)\qquad z\cap z'''=z\cap(z'\cap z''')=(z\cap z')\cap z'''=0\longleftrightarrow z\cap z''=z
  $$
  e
  $$
    (\Delta)\qquad z'\cap z''=0\longleftrightarrow z\cap z''=z''
  $$
  de $(\Gamma)$ e $(\Delta)$ decorre que $z=z''$. Adicionalmente,
  $$
    \forall z\,\bigl(z\cup z=(z'\cap z')'=(z')'=z''=z\bigr)
  $$
  
  Seguidamente provemos que para quaisquer $x,y,z\in B$ vale
  $$
    x\cap(y\cup z)=(x\cap y)\cup(x \cap z)
  $$
  em outros termos, queremos provar que
  $$
    x\cap(y'\cap z')'=\bigl((x\cap y)'\cap(x\cap z)'\bigr)'
  $$
  Antes provemos uma identidade auxiliar
  $$
    (I)\quad\forall u,v\bigl(u,v\in B\Longrightarrow u=u\cap(u'\cap v)'\bigr)
  $$
  mas do item 3 temos que 
  $$
    u\cap(u'\cap v)'=u\Longleftrightarrow u\cap(u'\cap v)=0.
  $$
  Mas o segundo membro da bicondicional \'e trivialmente verdadeiro, em consequ\^encia a proposi\c c\~ao $(I)$ \'e verdadeira. 
  
  Proseguindo primeiro provemos que
  $$
    x\cap(y'\cap z')'\cap\bigl((x\cap y)'\cap(x\cap z)'\bigr)'=x\cap(y'\cap z')'
  $$
  do item 3 do rol de axiomas temos que a \'ultima identidade \'e equivalente a
  $$
    x\cap(y'\cap z')'\cap(x\cap y)'\cap(x\cap z)'=0
  $$
  que por sua vez \'e equivalente a
  $$
    x\cap y'\cap z'\cap(x\cap y)'\cap(x\cap z)'=x\cap(x\cap y)'\cap(x\cap z)'
  $$
  
  \bigskip
  
  \theorem{\ (\'Algebra de Boole dada no livro de Hewitt~e Stromberg {\it Real and Abstract Analysis}) [Huntington]}{\sl Seja $B$ um conjunto munido com uma opera\c c\~ao bin\'aria $\cup$ e uma opera\c c\~ao un\'aria $'$ satisfazendo os seguintes axiomas:
    \medskip
    \halign{
      \qquad\rm# & \rm#\hfill & \rm\quad#\hfill \cr
      a) & $a',a\cup b\in B$                & ({\it Estabilidade de $'$ e $\cup$}); \cr
      b) & $a\cup b=b\cup a$                & ({\it Comutatividade\/});             \cr
      c) & $a\cup a=a$                      & ({\it Idempot\^encia\/});             \cr
      d) & $a\cup(b\cup c)=(a\cup b)\cup c$ & ({\it Associatividade\/});            \cr
      e) & $a'=(a'\cup b')'\cup(a'\cup b)'$ & ({\it Caracteriza\c c\~ao de $'$}).   \cr
    }
    \medskip
    Ent\~ao $\langle B,\cup,'\rangle$ \'e uma \'algebra de Boole. Tais axiomas $($ou postulados\/$)$ s\~ao conhecidos como postulados de Huntington.
  }
  
  Resumo das propriedades encontradas at\'e ent\~ao:
  
  Sejam $a,b\in B$, temos
  $$
    \vbox{
      \halign{
        # & #\cr
        $a\cup a'$ & $=\bigl((a'\cup b'')'\cup(a'\cup b')'\bigr)\cup\bigl((a''\cup b'')'\cup(a''\cup b')'\bigr)$\hfill\cr 
        \omit      & $=\bigl((b'\cup a'')'\cup(b'\cup a')'\bigr)\cup\bigl((b''\cup a'')'\cup(b''\cup a')'\bigr)$\hfill\cr 
        \omit      & $=b\cup b'$\hfill                                                                           \cr
      }
    }
  $$
  
  Do resultado anterior trocando $a$ e $b$, por $a'$ e $a''$ respectivamente, obtemos $a'\cup a''=a''\cup a'''$, para todo $a\in B$, {\it a fortiori} para todo $a\in B$, vale
  $$
    a''=(a'''\cup a'')'\cup(a'''\cup a')'=(a'\cup a'')'\cup(a'\cup a''')'=a.
  $$
  
  \noindent{\bf Defini\c c\~ao.}\ Um conjunto ordenado $S$ \'e dito ter a propriedade da menor cota superior se, para todo conjunto n\~ao vazio $E$ limitado superiormente admite a menor das cotas superiores de $E$, denotada por $\sup E$.
  
  \theorem{} Todo conjunto ordenado que tem a propriedade da menor cota superior tem a propriedade da maior cota superior.
  
  \proof Sejam $E\subseteq S$, n\~ao vazio limitado inferiormente e $L$ o cojunto das cotas inferiores de $E$ que, como sabemos \'e n\~ao vazio, pois $E$ \'e limitado inferiormente. Em vista da hip\'otese de $S$ ter a propriedade da menor cota superior existe $\sup L$. Seja agora $\gamma>\sup L$. Ent\~ao $\gamma$ n\~ao \'e cota inferior de $E$, pois do contr\'ario $\gamma\leq\sup L$, desta forma $\sup L$ \'e a maior das cotas inferiores de $E$, como consequencia $\inf E=\sup L$. Fica, portanto, completa a prova do teorema.\qed 
  
  \theorem{}{\sl\ Sejam $x,y\in \R$.
  
    a) Se $x>0$, ent\~ao existe $m\in Z$, tal que $mx>y$;
    
    b) Existe $q\in Q$ tal que $x<q<y$.
  }
  
  \proof 
  a) Suponhamos que a implica\c c\~ao \'e falsa, ent\~ao existe $x>0$, tal que para todo $m\in Z$ vale $mx\leq y$.
  
  Particularmente, $x/y>0$, pois $x,y>0$. Da\'\i\ vem que o conjunto $M=\{m(x/y)\in \R:m\in Z\}$ \'e limitado superiormente por $1$, como aquele \'e n\~ao vazio, decorre da propriedade da menor cota superior de $\R$ que existe $\varsigma=\sup M$. Temos, portanto, que para todo $m\in Z$ vale 
  $$
    (m+1)\bigl({x\over y}\bigr)\leq\varsigma
  $$
  como consequ\^encia para todo $m\in Z$ vale
  $$
    m({x\over y})\leq\varsigma-{x\over y}<\varsigma,
  $$
  i.e., $\varsigma-x/y$ \'e uma cota superior de $M$ o que contradiz a minimalidade de $\varsigma$.
  
  b) Primeiro, provaremos a seguinte
  
  \noindent{\bf Afirma\c c\~ao.}\ {\sl
    Para todo $\varrho\in \R$, existe um \'unico inteiro $m$, tal que $m\leq\varrho<m+1$.
  } 
  
  \proof Suponha o contr\'ario que para todo $m\in Z$, se $m\leq\varrho$, ent\~ao $m+1\leq\varrho$, note que esta \'ultima condicional \'e equivalente a $\neg(m\leq\varrho<m+1)$. Mas, isto implica por indu\c c\~ao matem\'atica que para todo $m\in N$ vale $m\leq\varrho$, o que contradiz o item a). Conformemente esta contradi\c c\~ao nos leva a concluir de que nossa suposi\c c\~ao inicial est\'a incorreta, provando portanto a exist\^encia. 
  
  Sejam $m_i$ $(i=1,2)$, tais que $m_i\leq\varrho<m_i+1$, da totalidade da ordem $<$ em $\R$ e supondo por redu\c c\~ao ao absurdo que $m_1\neq m_2$, podemos supor sem perda de generalidade que $m_1<m_2$. Da\'\i, decorre imediatamente que $m_1+1\leq m_2$, pois suponha por redu\c c\~ao ao absurdo que $m_1+1>m_2$, ter\'\i amos
  $$
    0<m_2-m_1<1
  $$
  o que \'e um absurdo, pois $m_2-m_1\in Z$ e n\~ao existe inteiro entre $0$ e $1$. Por maior raz\~ao $m_1+1\leq m_2$, mas isto por sua vez acarreta que
  $$
    \varrho<m_1+1\leq m_2\leq \varrho
  $$
  um outro absurdo. Estas contradi\c c\~oes nos levam a concluir que a nossa suposi\c c\~ao $m_1\neq m_2$, \'e uma suposi\c c\~ao inver\'\i dica. Destarte fica provada a unicidade.\Qed
  
  Da\'\i\ podemos inferir que dado $\varrho\in \R$, tomando $m$ satisfazendo a afirma\c c\~ao anterior, decorre necessariamente que $m+1\leq\varrho+1$, pois suponha o contr\'ario, i.e., que $m+1>\varrho+1$, disto decorre que
  $$
    1=(\varrho+1)-\varrho<(m+1)-m=1
  $$
  um absurdo. Concluimos, portanto, que $m+1\in(\varrho,\varrho+1]$, provamos portanto a
  \medskip
  \noindent{\bf Afirma\c c\~ao.}\ {\sl
    Para todo $\varrho\in \R$, existe um inteiro $m$, tal que $m\in(\varrho,\varrho+1]$.
  }\Qed 
  
  Consequentemente, usando o item a) existe $n\in N$, tal que $(y-x)n>1$ e da afirma\c c\~ao imediatamente anterior existe $m\in(nx,nx+1]$, assim vale
  $$
    nx<m\leq nx+1<ny,
  $$
  logo, fazendo $q=m/n$, temos da cadeia anterior que $x<q<y$, concluindo portanto a prova do item b).\qed
  
  Vale tamb\'em a
  \medskip
  \noindent{\bf Afirma\c c\~ao.\ }{\sl Sejam $\varrho\in \R$, $n\leq\varrho$ e $m\in Z$, tal que $m\leq\varrho<m+1$. Ent\~ao $n\leq m$.
  }
  \medskip
  
  \proof Ora, se $n>m$, ent\~ao $n\geq m+1$, da\'\i\ vem
  $$
    \varrho<m+1\leq n\leq\varrho
  $$
  uma contradi\c c\~ao.\qed
  
  Podemos inferir da\'\i\ que $m$ com aquela propriedade \'e o maior inteiro tal que $m\leq\varrho$.
  
  \theorem{\ (Digress\~ao relativa \`a representa\c c\~ao decimal de um n\'umero real)}{\sl Seja $\varrho\in \R_+^*$, escolhamos $n_0\in N$, tal que
    $$
      n_0\leq\varrho<n_0+1.
    $$
    Suponhamos escolhidos $n_0,\ldots,n_k$, escolhamos $n_{k+1}\in N$, tal que
    $$
      (\Delta)\qquad n_{k+1}\leq10^{k+1}(\varrho-\alpha_k)<n_{k+1}+1,
    $$
    em que
    $$
      \alpha_k=n_0+\ldots+{n_k\over10^k},
    $$
    podemos, portanto, por indu\c c\~ao considerar o conjunto
    $$
      A=\{\alpha_k:k\in N\}.
    $$
    Ent\~ao nestas condi\c c\~oes $\varrho=\sup A$.
  }
  
  \proof De $(\Delta)$ segue necessariamente que 
  $$
    0\leq\varrho-\alpha_{k+1}=\varrho-\alpha_k-{n_{k+1}\over10^{k+1}}<{1\over10^{k+1}},
  $$
  para qualquer $k\in N$.
  Observemos que dado $\gamma<\varrho$, temos peremptoriamente que $\varrho-\gamma>0$, dos teoremas anteriores existe $k\in N$ tal que
  $$
    10^{k+1}>2^{k+1}>k>{1\over\varrho-\gamma}>0
  $$
  donde
  $$
    \varrho-\alpha_{k+1}<{1\over10^{k+1}}<\varrho-\gamma,
  $$
  i.e., existe $\alpha\in A$, tal que $\alpha=\alpha_{k+1}>\gamma$, i.e., todo elemento $\gamma\in \R$, tal que $\gamma<\varrho$, n\~ao \'e cota superior de $A$. Al\'em disso $\varrho\geq\alpha$, para todo $\alpha\in A$, i.e., $\varrho$ \'e uma cota superior de $A$, que, conforme conclu\'\i mos, \'e tal que $\varrho=\sup A$.\qed
  
  Quando $\varrho=\sup A$, representa-se $\varrho$ por
  $$
    n_0.n_1n_2n_3\ldots
  $$
  por exemplo 
  $$
    \pi=3.141592\ldots
  $$
  A representa\c c\~ao do n\'umero corresponde \`a uma sequ\^encia finita de seus primeiros d\'\i gitos seguido de retic\^encias. Subentende-se que os outros d\'\i gitos sejam determinados indutivamente. 
  
  \chapter{Constru\c c\~ao dos n\'umeros reais}
  Doravante admitamos que letras gregas representem cortes e letras romanas representem n\'umeros racionais, salvo quando for explicitamente dito o contr\'ario. Dado um corte $\alpha$, por conveni\^encia definiremos que
  
  
  % TODO: Simplificar.
  \begin{Def}
  Seja $\alpha$ um corte tal que $\alpha>0^*$, i.e., um corte estritamente positivo\footnote{veja o [Rudin] ou [Spivak]).}. Definamos
  \[
    \alpha^{-1}=\{r\in \Q: \exists s,t(s,t\in \Q\a t>1 \a1/st\notin\alpha\a r\leq s)\}.
  \]
  \end{Def}
  
  Antes de prosseguirmos, precisaremos do
  
  \lemma{\ (Adaptado de [Spivak])}Sejam $\alpha>0^*$ e $p\in Q$, tal que $p>1$. Ent\~ao existem $q,r\in Q$, tais que $q\in\alpha$, $1/r\in\alpha^{-1}$ e $p=r/q$.
  
  \proof Seja $p$ como nas premissas, afirmo que existe $s\in Q$, tal que $0<s<1$, tal que $sp\in\alpha$, de fato tome $t\in\alpha$, tal que $t>0$, o que \'e garantido pela nossa suposi\c c\~ao de que $\alpha>0^*$. Da propriedade arquimediana de $Q$, existe $n\in N$, tal que $n>p/t$, da\'\i\ vem que $p/n<t$, fazendo $s=1/n$, temos consequentemente $sp<t$, {\it a fortiori} $sp\in\alpha$, pois $\alpha$ \'e um corte e $t\in\alpha$. Observando que
  $$
    sp^n=s\bigl(1+(p-1)\bigr)^n\ge sn(p-1)
  $$
  existe $n\in N$, tal que
  $$
    sp^n\in\alpha\a sp^{n+1}\notin\alpha
  $$
  consequentemente fazendo $q=sp^n$ e $r=sp^{n+1}$, temos $p=r/q$. Ademais, existe $t>1$, tal que $rt\notin\alpha$ e $qt\in\alpha$, isto decorre do fato de $\alpha$ n\~ao possuir um maior elemento. De fato, tome $q'>q$ tal que $q'\in\alpha$ e fa\c ca $t=q'/q$, observe que $q'=qt$ e $r'=rt$ s\~ao tais que $r'/q'=p$. Da\'\i\ existe $t>1$ tal que $1/(t/r')=r'/t=r\notin\alpha$, portanto, podemos sem perdas admitir que $q\in\alpha$ e $1/r\in\alpha^{-1}$.\qed
  
  \theorem{}{\sl Se $\alpha>0^*$, ent\~ao $\alpha\alpha^{-1}=1^*$.}
  
  \proof Primeiro provaremos que $\alpha^{-1}$ \'e um corte.
  
  (I) seja $p\notin\alpha$ e $t>1$, temos que existe $n\in N$, tal que $n>pt$, fazendo $s=1/n$, obtemos que $pst<1$, consequentemente $p<1/st$, como $p\notin\alpha$ e este \'ultimo \'e um corte $1/st\notin\alpha$, por maior raz\~ao infere-se que $s\in\alpha^{-1}$, provando que $\alpha\neq\emptyset$. Agora seja $q\in\alpha$, tomemos $r=1/q$, notemos que para todo $t>1$, tem-se $1/rt\in\alpha$, pois $1/t<1$ e $q\in\alpha$, o que prova que $q\notin\alpha^{-1}$, provando que $\alpha^{-1}\neq Q$;
  
  (II) Consideremos agora $p\in Q$ e $q\in\alpha^{-1}$, tais que $p<q$, segue evidentemente que existem $s\geq q>p$ e $t>1$, tais que $1/st\notin\alpha$, consequentemente $p\in\alpha^{-1}$;
  
  (III) Seja agora $p\in\alpha^{-1}$, ent\~ao existem $s,t\in Q$, tais que $t>1$ e $1/st\notin\alpha$, tome $t'\in Q$, tal que $1<t'<t$, em seguida observemos que
  $$
    {1\over (st/t')t'}={1\over st}\notin\alpha
  $$
  fazendo $q=st/t'$ temos evidentemente que $q>s\geq p$, o que prova que existe $q>p$, tal que $q\in\alpha^{-1}$.
  
  Seguidamente provemos que $\alpha\alpha^{-1}=1^*$, para tanto, seja $p\in\alpha$ e $q\in\alpha^{-1}$, temos que existe $s,t\in Q$, tais que $t>1$ e  $1/st\notin\alpha$, da\'\i\ vem que $p<1/st$, como consequ\^encia 
  $$
    pq\leq ps<pst<1,
  $$ 
  que por sua vez acarreta $\alpha\alpha^{-1}\subset 1^*$. Por outro lado seja $p\in 1^*$, i.e., $p<1$, que podemos supor sem perda de generalidade $p>0$, pois $1^*=(1^*)^2>0$. Segue do lema que existem $r\notin\alpha$ e $q\in\alpha$, tais que $1/p=r/q$ e $1/r\in\alpha^{-1}$, da\'\i\ vem $p=q/r\in\alpha\alpha^{-1}$, o que prova a inclus\~ao $1^*\subset\alpha\alpha^{-1}$. Decorre do axioma da extensionalidade $\alpha\alpha^{-1}=1^*$.\qed
  
  Vale observar que em virtude da defini\c c\~ao provis\'oria do produto de cortes ser limitado apenas a $\R_+^*$, n\~ao faz sentido $\alpha0^*$, pois isto n\~ao est\'a definido, uma vez que n\~ao existem elementos estritamente positivos em $0^*$, {\it a fortiori}, muito menos estar\'a definido $(0^*)^{-1}$.
  
  \theorem{\ (Lei distributiva)}{\sl Se $\alpha,\beta,\gamma>0^*$}, ent\~ao
  $$
    \alpha(\beta+\gamma)=\alpha\beta+\alpha\gamma.
  $$
  
  \proof Suponha que $p\in\delta$ com $\delta=\alpha(\beta+\gamma)$, da defini\c c\~ao existe $(q,r)\in\alpha^\bullet\times(\beta+\gamma)^\bullet$ tal que $p\leq qr$, como $r\in\alpha+\beta$ existe $(s,t)\in\alpha\times\beta$, tal que $r\leq s+t\in\alpha+\beta$, portanto,
  $$
    p\leq qr\leq q(s+t)=qs+qt\in\alpha\beta+\alpha\gamma.
  $$
  como $\epsilon=\alpha\beta+\alpha\gamma$ \'e um corte decorre que $p\in\epsilon$, o que prova a inclus\~ao $\delta\subset\epsilon$.
  
  Seja agora $p\in\epsilon$ existem portanto $r,t\in\alpha^\bullet$ e $(s,u)\in\beta^\bullet\times\gamma^\bullet$, tais que
  $p\leq rs+tu$, tomando $v=\max\{r,t\}$, temos que $p\leq v(s+t)\in\delta$, o que prova a outra inclus\~ao $\epsilon\subset\delta$. Do axioma da extensionalidade $\delta=\epsilon$, o que conclui a prova.\qed
  
  Conforme observado em [Rudin] se $\alpha<\beta$, ent\~ao $\alpha+\gamma<\beta+\gamma$, para todo corte $\gamma$, como consequ\^encia se $\alpha>0^*$, ent\~ao $-\alpha<0^*$. Podemos, portanto, definir a multiplica\c c\~ao em $\R^*$ por 
  $$
    (\Pi)\qquad\alpha\beta=
    \left\{
    \begin{matrix}
      -\bigl((-\alpha)\beta\bigr), & \alpha<0^*\a\beta>0^*; \cr
      -\bigl(\alpha(-\beta)\bigr), & \alpha>0^*\a\beta<0^*; \cr
      (-\alpha)(-\beta),           & \alpha<0^*\a\beta<0^*. \cr
    \end{matrix}
    \right.
  $$
  
  Sabemos $0^*$ \'e o elemento neutro da adi\c c\~ao em $\R$, espera-se naturalmente pelas propriedades de corpos que este seja tal que $\alpha0^*=0^*\alpha=0^*$, para todo $\alpha\in \R$ e, por este motivo define-se o produto com $0^*$ desta forma. Feito isto podemos neste est\'agio (seguido o roteiro dado em [Rudin]) afirmar categoricamente que $\R$ tem todas as propriedades da adi\c c\~ao e multiplica\c c\~ao satisfeitas, com exce\c c\~ao da propriedade distributiva.
  
  Consideremos $\alpha,\beta,\gamma\in \R$, observe que se alguns destes n\'umeros reais for $0^*$, ent\~ao o produto \'e trivial. Desta forma, admitiremos que $\alpha,\beta,\gamma\in \R^*$. Consideremos, a t\'\i tulo de exemplo o caso em que $\alpha<0$, $\beta+\gamma>0$ e $\gamma<0$, neste caso temos que $\beta>0$ e
  $$
    (-\alpha)(\beta+\gamma)+\alpha\gamma=(-\alpha)(\beta+\gamma)+(-\alpha)(-\gamma)=(-\alpha)\beta
  $$
  donde vem
  $$
    \alpha(\beta+\gamma)=-(-\alpha)(\beta+\gamma)=\alpha\gamma-(-\alpha)\beta=\alpha\beta+\alpha\gamma.
  $$
  Os outros casos s\~ao tratados de maneira an\'aloga.
  
  Daqui podemos dizer que conclu\'\i mos a constru\c c\~ao de um corpo ordenado  com a propriedade da menor cota superior, a saber, o conjunto $\R$ dos cortes de Dedekind.
  
  Refa\c co as suas demonstra\c c\~oes [Rudin] com adapta\c c\~ao e ponho as minhas no que segue para fins de documenta\c c\~ao, que, existe um corpo ordenado $Q^*$ em $\R$ isomorfo ao conjunto dos n\'umeros racionais.
  
  \medskip
  \noindent{\bf Defini\c c\~ao.}{ \sl Seja $q\in Q$ definamos
    $$
      q^*=\{p\in Q:p<q\}\qquad\hbox{Def.}
    $$
  }
  
  \theorem{\ (Indentifica\c c\~ao de $Q$ em $\R$)}{%
    \sl Sejam $r,s\in Q$.
    \smallskip
  
    \hskip .1ex
    \vbox{
      \halign{
        \hfill# & #\hfill \cr
         {\rm a)} & $r^*\in \R${\rm;}                  \cr
         {\rm b)} & $r^*+s^*=(r+s)^*${\rm;}           \cr
         {\rm c)} & $r^*s^*=(rs)^*${\rm;}             \cr
         {\rm d)} & $r^*<s^*$, se e somente se $r<s$. \cr
      }
    }
  }
  
  \proof a) Provar.
  
  b) Suponha que $(t,u)\in r^*\times s^*$, certamente que $t+u<r+s$, consequentemente $t+u\in(r+s)^*$, o que prova a inclus\~ao $r^*+s^*\subset(r+s)^*$.
  
  Por outro lado suponhamos que $p\in(r+s)^*$, i.e., $p<r+s$, tomando $t\in Q$, tal que $p-s<t<r$ temos certamente que $u=p-t<s$, consequentemente 
  $$
    p=t+(p-t)=t+u\in r^*+s^*,
  $$
   culminando a inclus\~ao $(r+s)^*\subset r^*+s^*$, que, em conjunto com a outra pode-se inferir pelo axioma da extensionalidade a iqualdade requerida, {\it viz\/} $r^*+s^*=(r+s)^*$.
  
  c) Observe tamb\'em que se $r=0$ ou $s=0$, a igualdade \'e, em verdade, trivial. Provado o item a) podemos inferir facilmente que $-(r)^*=(-r)^*$, portanto se provarmos que c) \'e v\'alida para quaisquer $r,s>0$, tanto ser\'a verdadeiro para $r,s$ n\~ao nulos quaisquer. Destarte, nos ateremos ao caso em que $r,s>0$. Seja $p\leq uv$, com $u,v>0$ e $(u,v)\in r^*\times s^*$, \'e consp\'\i cuo que $p<rs$, consequentemente $r^*s^*\subset(rs)^*$. 
  
  Seja agora, $p<rs$, podemos supor sem perda de generalidade que $p>0$, pois do contr\'ario tome $p'\in Q$, tal que $\max\{0,p\}<p'<rs$. Escolhendo $u\in Q$, tal que $p/r<u<s$, decorre que $t=p/u<r$. Ademais
  $$
    p=\bigl({p\over u}\bigr)u=tu\in r^*s^*,
  $$
  consequentemente $(rs)^*\subset r^*s^*$. O que novamente deduz-se pelo axioma da extensionalidade a igualdade requerida. 
  
  d) Por fim, se $r^*<s^*$, ent\~ao existe $p\in s^*\setminus r^*$, segue que $r\leq p<s$. 
  
  Por outro lado, se $r<s$ ent\~ao $r^*\subset s^*$ e $r\in s^*\setminus r^*$.\qed 
  
  O teorema anterior nos diz que o conjunto $Q^*=\{q^*\in \R:q\in Q\}$ \'e isomorfo a $Q$, desta forma podemos dizer que precipuamente $Q\subset \R$.
  
  \chapter{Unicidade dos reais}
  Os cortes de Dedekind nos fornece um caminho para a constru\c c\~ao de um corpo ordenado $\R$ com a propriedade da menor cota superior. Existem certamente outros caminhos como usando as sequ\^encias de Cauchy, resultado devido a Cantor. Surgem as perguntas: O que individualiza o conjunto $\R$? Ser\'a que este conjunto assim constru\'\i do \'e de fato o conjunto n\'umerico conhecido desde a escola prim\'aria? As constru\c c\~oes mencionadas n\~ao resultam em dois objetos matem\'aticos aparentemente distintos? Em verdade, o conjunto dos n\'umeros reais conhecidos na escola s\~ao na realidade somas parciais de s\'eries num\'ericas, ao passo que aquele que constru\'\i mos \'e uma cole\c c\~ao de subconjuntos de n\'umeros racionais. Como bem observado em [Elon], intrinsecamente estes objetos s\~ao distintos, mas do ponto de vista de isomorfismos s\~ao id\^enticos. Para tratar bem do caso das s\'eries deve-se concluir o exerc\'\i cio 2 do cap\'\i tulo 29 de [Spivak].  A seguir empenhar-nos-emos a provar que $\R$ \'e \'unico a menos de isomorfismos, para tanto precisaremos de algumas constru\c c\~oes auxiliares. 
  
  \begin{Def}
    Seja $\K$ um corpo com zero $0_\K$, unidade $1_\K$ e $x\in \K$ definamos
    \[
      0*x=0_\K
    \]
    e indutivamente
    \[
      (n+1)*x=n*x+x 
    \]
    para todo $n\in\omega$[\footnote{Esta defini\c c\~ao requer uma prova da defini\c c\~ao por recurs\~ao, um resultado muito famoso na teoria dos conjuntos.}.
  \end{Def}
  
  \begin{teo}
    Seja $\K$ um corpo, com zero $0_\K$. Ent\~ao para quaisquer $m,n\in\omega$ e $x,y\in \K$ valem
    \begin{enumerate}[label = (\alph*)]
      \item{%
        $n*0_K=0_\K$;
      }
      \item{%
        $m*x+n*x=(m+n)*x$;
      }
      \item{%
        $n*x+n*y=n*(x+y)$;
      }
      \item{%
        $m*(n*x)=mn*x$;
      }
      \item{%
        $(n*x)y=n*xy$;
      }
      \item{%
        $(m*x)(n*y)=mn*xy$.
      }
    \end{enumerate}
  \end{teo}
  \prova Sejam $m\in\omega$ e $x,y\in \K$.
  \begin{enumerate}[label = (\alph*)]
    \item{%
      Definamos o conjunto
      \[
        S=\{n:n\in\omega\a n*0_\K=0_\K\}.
      \]
      Certamente que $0\in S$, porquanto pela defini\c c\~ao vale
      \[
        0*0_k=0_\K.
      \]
      Se $n\in S$, temos que
      \[
        (n+1)*0_\K=n*0_\K+0_\K=0_\K+0_\K=0_\K.
      \]
      Consequentemente $n+1\in S$, donde pela minimalidade de $\omega$, inferimos que $S=\omega$[\footnote{Esta \'e simplesmente a consequ\^encia da defini\c c\~ao de $\omega$ como o conjunto sucessor minimal, seguindo a terminologia de Halmos.}. 
    }
    \item{%
      Consideremos o conjunto
      \[
        S=\{n:n\in\omega\a m*x+n*x=(m+n)*x\}.
      \]
      Primeiramente, observemos que $0\in S$, pois
      \[
        m*x+0*x=m*x+0_\K=m*x=(m+0)*x.
      \]
      Em seguida, suponhamos que $n\in S$. Temos conformemente
      \[ 
        \begin{split}
          m*x+(n+1)*x &=m*x+n*x+x\cr
                      &=(m+n)*x+x\cr
                      &=\bigr(m+(n+1)\bigr)*x.
        \end{split}
      \]
      Portanto, $n+1\in S$. Pela minimalidade de $\omega$ decorre que $S=\omega$.
    }
    \item{%
      Da mesma maneira seja
      \[
        S=\{n:n\in\omega\a n*x+n*y=n*(x+y)\}.
      \]
      Temos que $0\in S$, pois
      \[
        0*x+0*y=0_\K=0*(x+y).
      \]
      Se $n\in S$, ent\~ao
      \[
        \begin{split}
        (n+1)*x+(n+1)*y&=(n*x+x)+(n*y+y)\cr
                       &=(n*x+n*y)+(x+y)\cr
                       &=n*(x+y)+(x+y)\cr
                       &=(n+1)*(x+y),
        \end{split}
      \]
      i.e., $n+1\in S$. Consequentemente, $S=\omega$.
    }
    \item{%
      Consideremos novamente
      \[
        S=\{n:n\in\omega\a m*(n*x)=mn*x\}.
      \]
      Notemos que $0\in S$, uma vez que
      \[
        m*(0*x)=m*0_\K=0_\K=0*x=m0*x.
      \]
      Se $n\in S$, ent\~ao
      \[ 
        \begin{split}
          m*\bigl((n+1)*x\bigr) &=m*\bigl(n*x+x\bigr)\cr
                                &=mn*x+m*x\cr
                                &=(mn+m)*x\cr
                                &=\bigl(m(n+1)\bigl)*x.
        \end{split}
      \]
      Portanto, $n+1\in S$, logo $S=\omega$.
    }
    \item{%
      Seja
      \[
        S=\{n:n\in\omega\a (n*x)*y=n*xy\}.
      \]
      Primeiro, $0\in S$, pois
      \[
        (0*x)y=0_\K y=0_\K;
      \]
      Segundo, se $n\in S$, ent\~ao
      \[
        \begin{split}
           \bigl((n+1)*x\bigr)y&=(n*x+x)y\cr
                               &=(n*x)y+xy\cr
                               &=n*xy+xy\cr
                               &=(n+1)*xy.
        \end{split}
      \]
      Portanto, $n+1\in S$ e consequentemente $S=\omega$.
    }
    \item{%
      Defina
      \[
        S=\{n:n\in\omega\a(m*x)(n*y)=mn*xy\}.
      \]
      Note primeiramente que $0\in S$, porque
      \[
        (m*x)(0*y)=(m*x)0_\K=0_\K.
      \]
      Se $n\in S$, ent\~ao
      \[
        \begin{split}
          (m*x)\bigl((n+1)*y\bigr)&=(m*x)(n*y)+(m*x)y\cr
                                  &=mn*xy+m*xy\cr
                                  &=(mn+m)*xy\cr
                                  &=\bigl(m(n+1)\bigr)*xy.
        \end{split}
      \]
      Logo, $n+1\in S$ e, portanto, $S=\omega$.
  
    }
  \end{enumerate}
  \qed
  
  
  \begin{Def}
    Definamos $\sigma:\Z\rightarrow \K$, por
    \[
      \sigma(n)=
      \begin{cases}
        \hfill  1_\K, & n>0\cr
        \hfill  0_\K, & n=0\cr
        \hfill -1_\K, & n<0.\cr
      \end{cases}
    \]
    Al\'em disso, para $n\in \Z$ definamos
    \[
      n^*=|n|*\sigma(n)
    \]
    e
    \[
      n*x=n^*\cdot x.
    \]
    \label{def_int_K}
  \end{Def}
  
  \begin{lem}
    Sejam $\K$ um corpo com unidade $1_\K$. Ent\~ao valem
    \begin{enumerate}[label = (\alph*)]
      \item{%
        \[
          \forall n\bigl(n\in \Z\lra\sigma(-n)=-\sigma(n)\bigr);
        \]
      }
      \item{%
        \[
          \forall m,m\bigl(n,m\in \Z\lra\sigma(mn)=\sigma(m)\sigma(n)\bigr).
        \]
      }
    \end{enumerate}
  \end{lem}
  \prova Sejam $m,n\in\omega$.
  \begin{enumerate}[label = (\alph*)]
    \item{%
      Se $n\in\omega$, ent\~ao
      \[
        \sigma(-n)=-1_\K=-\sigma(n).
      \]
      Caso contr\'ario, $-n\in\omega$, da\'i
      \[
        \sigma\bigl(-(-n)\bigr)=-\sigma(-n)\lra\sigma(-n)=-\sigma(n).
      \]
    }
    \item{%
      Sejam $m,n\in\omega$. \'E suficiente notar que
      \[
        \begin{split}
          \sigma(m)\sigma(n) &=\sigma(mn)\cr
                             &=\sigma\bigl((-m)(-n)\bigr)\cr
                             &=-\sigma(-mn)\cr
                             &=-\sigma\bigl((-m)n\bigr)\cr
        \end{split}
      \]
    }
  \end{enumerate}
  \qed
  
  \section{Caracter\'istica de corpos}
  
  \begin{Def}
    Seja $\K$ um corpo com unidade $1_\K$ e zero $0_\K$. Definimos a caracter\'istica de $\K$ por 
    \[
      X(\K)=\bigcap\{n:n\in\omega\cp\{0\}\a n^*=0_\K\}.
      \label{corpo_carac}
    \]
    Observe que $X(\K)$ \'e sempre um n\'umero natural, podendo inclusive ser $0=\emptyset$.
    \label{def_X}
  \end{Def}
  
  \begin{teo}
    Seja $\K$ um corpo tal que $X(\K)=0$. Ent\~ao existe um $\Z^*\subset \K$ isomorfo a $\Z$.
  \end{teo}
  
  \prova De posse da defini\c c\~ao {\bf \ref{def_int_K}}, podemos definir uma fun\c c\~ao $\zeta:\Z\lra \K$ dada por
  \[
    \zeta(m)=m^*.
  \]
  Doravante, nosso objetivo limitar-se-a \`a prova de que $\zeta$ \'e um isomorfismo de an\'eis.
  
  Para todo $n\in \Z$, temos
  \[
    \begin{split}
      \zeta(n)+\zeta(-n)&=|n|*\sigma(n)+|n|*\sigma(-n)\cr
      &=|n|*(\sigma(n)+\sigma(-n))\cr
      &=|n|*(\sigma(n)-\sigma(n))\cr
      &=0_\K,
    \end{split}
  \]
  i.e., 
  \[
    \forall n\bigr(n\in \Z\lra\zeta(-n)=-\zeta(n)\bigr)
  \]
  
  Sejam $m,n\in \Z$. Se $m+n=0$, ent\~ao \'e imediato que
  \[
    \zeta(m+n)=\zeta(m)+\zeta(n).
  \]
  Com efeito, se $m+n=0$, ent\~ao vimos que $-\zeta(m)=\zeta(-m)=\zeta(n)$, da\'i vem
  \[
    \zeta(m+n)=0_\K=\zeta(m)-\zeta(m)=\zeta(m)+\zeta(-m)=\zeta(m)+\zeta(n)
  \]
  
  Agora suponhamos que $m+n\neq0$. Se $m$ e $n$ t\^em o mesmo sinal, ent\~ao
  \[
    \zeta(m+n)=|m+n|*\sigma(m+n)=|m|*\sigma(m)+|n|*\sigma(n)=\zeta(m)+\zeta(n),
  \]
  porquanto, $\sigma(m)=\sigma(n)=\sigma(m+n)$.
  
  Suponhamos que $m$ e $n$, n\~ao tenham o mesmo sinal. Ent\~ao $m+n$ tem o mesmo sinal de $-m$ ou $-n$. Pois se, $m+n$ tem um sinal e este difere dos de $-m$ e $-n$, ent\~ao estes, por sua vez, ter\~ao o mesmo sinal e consequentemente $m$ e $n$ tamb\'em. Em conformidade, e sem perda de generalidade suponhamos que $m+n$ e $-m$ tenham o mesmo sinal, conforme vimos
  
  \[
    \zeta(m+n)+\zeta(-m)=\zeta\bigl((m+n)-m\bigr)=\zeta(n)
  \]
  o que por sua vez acarreta
  \[
    \zeta(m+n)=\zeta(m)+\zeta(n).
  \]
  
  Em seguida, observemos que para quaisquer $m,n\in \Z$, vale
  \[
    \begin{split}
      \zeta(mn)&=(mn)^*\cr
      &=|mn|*\sigma(mn)\cr
      &=(|m||n|)*\bigl(\sigma(m)\sigma(n)\bigr)\cr
      &=\bigl(|m|*\sigma(m)\bigr)\bigl(|n|*\sigma(n)\bigr)\cr
      &=m^*n^*\cr
      &=\zeta(m)\zeta(n).
    \end{split}
  \]
  
  Por fim, seja $n\in\ker\zeta$, temos que $\zeta(n)=n^*=0_\K$. Se $n\neq 0$, podemos sempre supor que $n>0$, bastando para isso considerar $-n$. Conformemente incorrer\'iamos na exist\^encia de um $n\in\omega\cp\{0\}$ tal que $n^*=0_\K$, a fortiori $X(\K)\neq0$, o que \'e uma contradi\c c\~ao. Isto implica por sua vez que $\ker\zeta=\{0\}$, confirmando que $\zeta$ \'e um isomorfismo. Assim, para concluirmos basta tomar $\Z^*=\zeta(\Z)$.
  \qed
  
  Sob as condi\c c\~oes do teorema anterior temos a
  \begin{Def}
    Seja $\K$ um corpo e $X(\K)=0$. Ent\~ao definimos $\Z^*=\zeta(\Z)$.
  \end{Def}
  
  \begin{teo}
    Se $\K$ \'e um corpo tal que $X(\K)=0$, ent\~ao existe $\Q^*\subset \K$ isomorfo ao conjunto dos n\'umeros racionais $\Q$.
    \label{copia_de_Q_em_K}
  \end{teo}
  \prova Em $\Z^*\times(\Z^*\cp\{0_\K\})$ definimos a seguinte rela\c c\~ao
  \[
    (m^*,n^*)\sim^* (p^*,q^*)\llra m^*q^*=n^*p^*.
  \]
  Doravante, admitamos que $m,n,p,q,r,s\in\Z$. 
  
  Primeiramente, observemos
  \[
    (m^*,n^*)\sim^*(m^*,n^*)\llra m^*n^*=n^*m^*.
  \]
  Portanto, $\sim^*$ \'e reflexiva.
  
  Em seguida, 
  \[
    (m^*,n^*)\sim^*(p^*,q^*)\llra m^*q^*=n^*p^*\llra p^*n^*=q^*m^*\llra(p^*,q^*)\sim^*(m^*,n^*).
  \]
  Logo, $\sim^*$ \'e sim\'etrica.
  
  E por fim, suponhamos que 
  \[
    (m^*,n^*)\sim^*(p^*,q^*)\a (p^*,q^*)\sim^*(r^*,s^*).
  \]
  Segue portanto que
  \[
    m^*q^*s^*=n^*p^*s^*=n^*q^*r^* \lra m^*s^*=n^*r^*\llra (m^*,n^*)\sim^*(r^*,s^*).
  \]
  Consequentemente, $\sim^*$ \'e transitiva, {\it a fortiori}, $\sim^*$ \'e uma rela\c c\~ao de equival\^encia. 
  
  Podemos construir uma bije\c c\~ao natural $\rho:\Z\times(\Z\cp \{0\})\ra \Z^*\times(\Z^*\cp\{0_\K\})$, dada por
  \[
    \rho\bigl((m,n)\bigr)=(m^*,n^*).
  \]
  Esta bije\c c\~ao por sua vez, nos fornece um isomorfismo natural definido em 
  \[
    \Q={\Z\times(\Z\cp\{0\})\over\sim}
  \]
  com imagens em
  \[
    \Q^*={\Z^*\times(\Z^*\cp\{0_\K\})\over\sim^*}
  \]
  
  Com efeito, sejam $m,n\in \Z$, denotaremos a classe cujo representante \'e $(m,n)$ por $m/n$, e a classe de representante $(m^*,n^*)$ por $m^*/n^*$. Agora, basta definir $\varrho:\Q\ra\Q^*$ por
  \[
    \varrho\bigl({m\over n}\bigr)={m^*\over n^*}.
  \]
  A prova de que $\varrho$ preserva a adi\c c\~ao e a multiplica\c c\~ao \'e imediata ao isomorfismo $\zeta:\Z\ra\Z^*$; al\'em disso, \'e trivialmente sobrejetiva. Suponha que
  \[
    {m^*\over n^*}=\varrho\bigl({m\over n}\bigr)=\varrho\bigl({p\over q}\bigr)={p^*\over q^*}
  \]
  Da\'i vem 
  \[
    m^*q^*=n^*p^*\llra mq=\zeta^{-1}(m^*q^*)=\zeta^{-1}(n^*p^*)=np,
  \]
  por conseguinte $m/n=p/q$ e consequentemente $\varrho$ \'e injetiva, portanto, bijetiva, {\it a fortiori}, um isomorfismo de $\Q$ em $\Q^*$.
  
  Finalmente, observemos que $\Q^*$ pode ser observado como subconjunto de $\K^*$, bastanto para isso considerar a identifica\c c\~ao $\iota:\Q^*\ra\K$ dada por $\iota(m/n)=m^*(n^*)^{-1}$. Em verdade, sejam $(m,n)$ e $(p,q)$ dois representantes de $m/n$, temos em conformidade 
  \[
    m^*q^*=n^*p^*\llra m^*(n^*)^{-1}=p^*(q^*)^{-1},
  \]
  i.e., a defini\c c\~ao de $\iota$ n\~ao \'e amb\'igua. Com um racioc\'inio an\'alogo, v\^e-se que $\iota$ \'e injetiva. Destarte, sem perdas podemos identificar $\Q^*$ com $\iota(\Q^*)\subset K$, e abusar da nota\c c\~ao escrevendo $\Q^*\subset \K$.\qed
  
  \begin{Def}
    Seja $\K$ um corpo e $X(\K)=0$. Ent\~ao definimos $\Q^*$ como no teorema {\bf\ref{copia_de_Q_em_K}}. Ademais, da identifica\c c\~ao podemos sem ambigudade supor
    \[
      {m^*\over n^*}=m^*(n^*)^{-1}.
    \]
  \end{Def}
  
  \begin{teo}
    Seja $\K$ um corpo ordenado com a propriedade da menor cota superior tal que $X(\K)=0$. Ent\~ao $\K$ \'e arquimediano e o conjunto $\Q^*$ \'e denso em $\K$.
    \label{K_arqui}
  \end{teo}
  
  \prova Sejam $x>0_\K$ e $y\in \K$, consideremos o conjunto
  \[
    M=\{n*x:n\in\omega\}
  \]
  Afirmo que existe $n\in\omega$ tal que $n*x>y$. De fato, suponha que $y$ seja um majorante do conjunto $M$, decorre de $\K$ ter a propriedade da menor cota superior que existe $\sigma=\sup M$, da\'\i\ vem que para todo $n\in\omega$
  \[
    n*x\leq\sigma
  \]
  por maior raz\~ao, para todo $n\in\omega$, segue-se
  \[
    (n+1)*x\leq\sigma,
  \]
  donde para todo $n\in\omega$ vale
  \[
    n*x=(n+1)*x-x\leq\sigma-x<\sigma,
  \]
  contradizendo o fato de $\sigma$ ser a menor cota superior do conjunto $M$. 
  
  Imediatamente, suponhamos que $x,y\in \R$ sejam tais que $x<y$, da propriedade arquimediana de $\K$ existe $n\in\omega$, tal que $n*(y-x)>1^*$. Seja $m$ o maior inteiro tal que $m^*\leq n*x$. Destarte, de 
  
  \[
    m^*\leq n*x<(m+1)^*=m^*+1^*,
  \] 
  conclu\'\i mos que 
  \[
    (m+1)^*\leq n*x+1^*
  \]
  pois do contr\'ario ter\'\i amos
  \[
    1^*=m^*+1^*-m^*>n*x+1^*-n*x=1^*,
  \]
  um absurdo. Logo,
  \[
    n^*\cdot x=n*x<(m+1)^*\leq n*x+1^*<n*y=n^*\cdot y
  \]
  donde
  \[
    x<{(m+1)^*\over n^*}<y.
  \]
  Em outros termos $\Q^*$ \'e denso em $\K$.\qed
  
  \begin{teo}
    Sejam $\K$ e $\Q^*$ como no teorema {\bf \ref{K_arqui}}. Ent\~ao
    \[
      \Q^*=\{q^*:q^*=m^*/n^*\in \Q^*\a m^*\geq0^*\},
    \]
    e
    \[
      {m\over n}<{p\over q}\llra{m^*\over n^*}<{p^*\over q^*}.
    \]
    \label{iso_ord_Q}
  \end{teo}
  \prova \'E suficiente observar que vale
  \[
    {m^*\over n^*}={-m^*\over -n^*}
  \]
  e
  \[
    {m\over n}<{p\over q}\llra mq<np\llra m^*q^*<n^*p^*\llra{m^*\over n^*}<{p^*\over q^*}.
  \]
  \qed
  
  A constru\c c\~oes a seguir foram motivadas por alguns exerc\'\i cios do livro do [Elon] {\it Curso de An\'alise} Volume 1.
  
  \begin{teo}[Unicidade de $\R$ a menos de isomorfismo]
    Seja $\K$ um corpo e $\Q^*$ como teorema {\bf \ref{K_arqui}}.  Ent\~ao $\R$ \'e ordenadamente isomorfo a $\K$.
  \end{teo}
  
  \prova Seja $\rho:\Q\ra\Q^*$ um isomorfismo, doravante quando for coveniente denotaremos $\rho(q)$ por $q^*$. Defina a aplica\c c\~ao $\varphi:\R\rightarrow\K$, da seguinte maneira
  \[
    \varphi(x)=\sup\rho(\Q\,\cap\,]-\infty,x[).
  \]
  Afirmo que $\varphi$ \'e um isomorfismo. \'E o que nos empenharemos a demonstrar doravante.
  
  
  Inicialmente, observemos que $\varphi$ est\'a bem definida pois o supremo \'e \'unico e o conjunto sobre o qual se toma o supremo \'e n\~ao vazio e limitado superiormente em $\Q^*$. Efetivamente, basta tomar $q\in \Q\cap[x,\infty[$, da\'i e do teorema {\bf\ref{iso_ord_Q}} segue que $q^*$ \'e um majorante de $\rho(\Q\,\cap\,]-\infty,x[)$.
  
  Em seguida, provemos que $\varphi$ \'e injetiva. Ora, se $x<y$ ent\~ao existem $p,q\in \Q$ tais que $x<p<q<y$, consequentemente
  $$
    \varphi(x)\leq p^*<q^*\leq\varphi(y),
  $$
  observe que isto tamb\'em prova que $\varphi$ \'e estritamente crescente.
  
  Imediatamente, Seja $\beta\in \K$, consideremos 
  \[
    x=\sup\rho^{-1}(\Q^*\,\cap\,]\infty,\beta[).
  \] 
  Observe que faz sentido considerarmos o supremo uma vez que existe $q^*\in\K$, tal que $q^*>\beta$, logo o conjunto sobre o qual toma-se o supremo \'e limitado superiormente, pois $q$ \'e um majorante de tal conjunto; al\'em disso, tal conjunto \'e trivialmente n\~ao vazio. Queremos provar que $\beta=\varphi(x)$. Para tanto, observemos que se $q<x$, necessariamente $q^*<\beta$, pois se $q^*\geq\beta$, ent\~ao $q$ seria uma cota superior do conjunto $\rho^{-1}(\Q^*\,\cap\,]\infty,\beta[)$, logo $q\geq x$, uma contradi\c c\~ao. Desta forma, $\beta\geq\varphi(x)$.
  
  Seja agora $\alpha<\beta$, do teorema {\bf\ref{iso_ord_Q}} podemos considerar $p^*,q^*\in \K$, tais que
  \[
    \alpha<p^*<q^*<\beta
  \]
  segue-se que $p<q\leq x$, i.e., existe $p<x$, tal que $p^*>\alpha$, portanto $\alpha$ n\~ao \'e cota superior do conjunto $\varphi(x)$, em consequ\^encia $\varphi(x)=\beta$, ou seja, provamos que dado $\beta\in K$ existe $x\in \R$, tal que $\varphi(x)=\beta$, i.e., $\varphi$ \'e sobrejetiva.
  
  Consideremos $p<x+y$, da\'\i\ vem que $p-x<y$, da densidade de $\Q$ em $\R$ existe $q\in \Q$, tal que 
  \[
    p-x<q<y,
  \] 
  como consequ\^encia 
  \[
    p-q<x\a q^*\leq\varphi(y),
  \]
  da\'i\ decorre que 
  \[
    p^*-q^*\leq\varphi(x),
  \]
  portanto,
  \[
    p^*=(p^*-q^*)+q^*\leq\varphi(x)+\varphi(y),
  \]
  como $p<x+y$ \'e arbitr\'ario temos
  \[
    \varphi(x+y)\leq\varphi(x)+\varphi(y).
  \]
  Seja agora $\alpha<\varphi(x)+\varphi(y)$, segue-se que $\alpha-\varphi(y)<\varphi(x)$, como $\varphi(x)$ \'e a menor cota superior existe $q<x$  tal que 
  \[
    \alpha-\varphi(y)<q^*\leq\varphi(x).
  \]
  Por outro lado, como $\alpha-q^*<\varphi(y)$, pelo mesmo motivo existe $p<y$, tal que 
  \[
    \alpha-q^*<p^*\leq\varphi(y),
  \]
  donde podemos inferir que existe 
  \[
    r=p+q<x+y,
  \]
  tal que $\alpha<r^*$, i.e., $\alpha$ n\~ao \'e cota superior do conjunto $\varphi(x+y)$, conformemente $\varphi(x+y)=\varphi(x)+\varphi(y)$. Podemos, portanto, concluir que para todo $x\in \R$ vale
  \[
    \varphi(-x)=\varphi(-x)+\varphi(x)-\varphi(x)=\varphi(x-x)-\varphi(x)=-\varphi(x).
  \]
  Desta forma, \'e suficiente provarmos que 
  \[
    \varphi(xy)=\varphi(x)\varphi(x)
  \]
  para $x,y>0$, pois se ao menos um deles for nulo a identidade \'e imediata. Caso, digamos $x<0<y$, basta considerar $-x$, pois 
  \[
    \varphi(-xy)=\varphi(-x)\varphi(y)\longrightarrow -\varphi(xy)=-\varphi(x)\varphi(y)
  \]
  como consequ\^encia
  \[
    \varphi(xy)=\varphi(x)\varphi(y).
  \]
  Dessarte, consideremos $x,y>0$. Seja $p<xy$, podemos sem perda de generalidade supor que $p>0$.  Tome $q\in \Q$, tal que 
  \[
    {p\over y}<q<x
  \]
  como $p/q<y$, temos necessariamente que
  \[
    p^*=q^*{p^*\over q^*}\leq\varphi(x)\varphi(y)
  \]
  como consequ\^encia
  \[
    \varphi(xy)\leq\varphi(x)\varphi(y).
  \]
  Seja $\alpha<\varphi(x)\varphi(y)$, podemos supor sem perda de generalidade que $\alpha>0^*$, pois $\varphi(x),\varphi(y)>0^*$. Existe, portanto, $p<x$, tal que
  $$
    {\alpha\over\varphi(x)}<p^*\leq\varphi(y)
  $$
  da\'\i\ tamb\'em encontramos $q<x$, tal que
  $$
    {\alpha\over p^*}<q^*\leq\varphi(x)
  $$
  donde
  $$
    r^*=p^*q^*
  $$
  \'e tal que $r<xy$ e $\alpha<r^*$, provando que $\alpha$ n\~ao \'e cota superior do conjunto $\varpi(xy)$, concluindo portanto que $\varphi(xy)=\varphi(x)\varphi(y)$. O que conclui a prova do teorema.
  
  Ademais, vale observar que em particular
  \[
    \varphi(q)=q^*.
  \]
  Com efeito, seja $\sigma=\varphi(q)$ evidentemente $\sigma\leq q^*$. Se $\sigma<q^*$, ent\~ao existe $p^*\in Q^*$, tal que 
  \[
    \sigma<p^*<q^*
  \]
  como $\Q$ \'e ordenadamente isomorfo a $\Q^*$, ent\~ao $p<q$, mas isto contradiria o fato de $\sigma$ ser uma cota superior de $\varphi(q)$, segue, portanto, a igualdade mencionada.
  \qed
  
  \begin{teo}
    Se $\psi:\R\rightarrow\R$ \'e um isomorfismo, ent\~ao $\psi=\iota$, em que $\iota$ \'e a aplica\c c\~ao identidade. Em particular, se $\K$ e $\L$ s\~ao dois corpos ordenados com a propriedade da menor cota superior, ent\~ao existe um \'unico isomorfismo de $\K$ sobre $\L$.
  \end{teo}
  
  \proof Se $\psi$ \'e um isomorfismo, ent\~ao $\psi(1)=1$, donde conclui-se $\psi(m)=m$, para todo $m\in \Z$, que por sua vez implica $\psi\bigl(q\bigr)=q$, para todo $q\in\Q$. 
  
  Se $x<\varphi(x)$, ent\~ao existe $p\in \Q$, tal que 
  $$
    x<p=\varphi(p)<\varphi(x)
  $$
  donde se conclui 
  $$
    x<p<x
  $$
  uma contradi\c c\~ao. Uma prova an\'aloga pode ser dada para o caso em que $\varphi(x)<x$. Podemos portanto concluir que $\varphi=\iota$.
  
  Em seguida observemos que se $\varrho,\varsigma:\R\rightarrow\K$, s\~ao dois isomorfismos de $\R$ sobre $\K$, ent\~ao $\varsigma^{-1}\varrho=\iota$, consequentemente $\varrho=\varsigma$, i.e., existe um \'unico isomorfismo de $\R$ sobre $\K$.
  
  Seja agora um isomorfismo arbitr\'ario $\tau:\K\rightarrow\L$ e,  $\varphi:\R\rightarrow\K$ e $\psi:\R\rightarrow\L$ os \'unicos isomorfismos ent\~ao $\psi^{-1}\tau\varphi=\iota$, logo $\tau=\psi\varphi^{-1}$, desta forma $\tau$ \'e \'unico.
  \qed

  
  \vfill\break
  
  \begin{teo}[Exerc\'icio 58 {[Elon I]}]
    Seja $G$ um subgrupo aditivo de $\R$. Defina
    \[
      G^+=\{x:x\in\;]0,\infty[\;\a x\in G\}.
    \]
    Se $G\neq\{0\}$, ent\~ao
    \begin{enumerate}[label = {\rm\Roman*.}]
      \item{%
        Se $\inf G^+=0$, ent\~ao $\overline{G}=\R$, i.e., $G$ \'e denso em $\R$;
      }
      \item{%
        Se $\iota=\inf G^+>0$, ent\~ao $\iota\in G^+$ e
        \[
          G=\{x:\exists n(n\in\Z\a x=\iota n)\}=\iota\Z;
        \]
      }
      \item{%
        Se $\alpha\in\R\cp\Q$, ent\~ao $\Z+\alpha\Z$ \'e denso em $\R$.
      }
    \end{enumerate}
  \end{teo}
  \prova Seja $x\in\R$, suponhamos sem perdas que $x\geq 0$, pois caso contr\'ario basta considerar $-x$. Dado $\varepsilon$, existe $x_\varepsilon\in G^+$, tal que $0<x_\varepsilon\leq\varepsilon$, seja 
  \[
    m=\min\{n:n\in\omega\a nx_\varepsilon>x\},
  \]
  certamente $(n-1)x_\varepsilon\leq x<nx_\varepsilon$. Ademais, $|x-x_\varepsilon|=|x_\varepsilon-x|<\varepsilon$. Da arbitrariedade de $x$ incorremos que $x\in\overline{G}$, provando, portanto o item I.  

  Provar.
  \qed
  
  A seguir relembremos de alguns exerc\'\i cios interessantes de [Rudin], quanto a constru\c c\~ao de pot\^encias qualquer e da constru\c c\~ao do logaritmo, ambos de base $b>0$ qualquer.
  
  \theorem{}{\sl%
    Sejam $m/n, p/q\in Q$, tais que $m/n=p/q$ $($podemos supor que $n,q>0$ sem perda de generalidade\/$)$ e $b>1$. Ent\~ao
    $$
      (b^m)^{1/n}=(b^p)^{1/q}.
    $$
    Desta forma podemos definir
    $$
      b^{m/n}=(b^m)^{1/n}
    $$
    sem nos preocupar com a fra\c c\~ao representante do n\'umero racional. 
  
    Notado isto, vale
    $$
      \forall r,s(r,s\in Q\longrightarrow b^{r+s}=b^rb^s).
    $$
    Para $x\in \R$, definindo 
    $$
      B(x)=\{b^r\in \R:r\leq x\a r\in Q\},
    $$
    vale,
    $$
      \forall r\bigl(r\in Q\longrightarrow b^r=\sup B(r)\bigr).
    $$
    Assim, faz sentido definir
    $$
      b^x=\sup B(x)
    $$
    para $x\in \R$. Al\'em disso, vale tamb\'em
    $$
      b^{x+y}=b^xb^y.
    $$
  }
  
  \proof Primeiro, se $m/n=p/q$, ent\~ao $mq=np$. Desta maneira,
  $$
    \bigr((b^m)^{1/n}\bigl)^q=(b^{mq})^{1/n}=(b^{np})^{1/n}=b^p,
  $$
  da unicidade da raiz temos necessariamente que
  $$
    (b^m)^{1/n}=(b^p)^{1/q}.
  $$
  Segundo, 
  $$
    \vbox{
      \halign{
        \hfill#&#\hfill\cr
        $b^{m/n+p/q}$ & $=b^{(mq+np)/nq}$ \cr
        \omit         & $=(b^{mq+np})^{1/nq}$ \cr
        \omit         & $=(b^{mq})^{1/nq}\,(b^{np})^{1/nq}$ \cr
        \omit         & $=\bigl((b^{1/n})^{mnq}\bigr)^{1/nq}\bigl((b^{1/q})^{npq}\bigr)^{1/nq}$ \cr
        \omit         & $=(b^{1/n})^{m}(b^{1/q})^{p}$ \cr
        \omit         & $=(b^{m})^{1/n}(b^{p})^{1/q}$ \cr
        \omit         & $=b^{m/n}b^{p/q}$. \cr
      }
    } 
  $$
  
  Terceiro, as pot\^encias de expoente racional e base fixa $b>1$ preservam a ordem. Assim, se $r<s$, ent\~ao $b^r<b^s$. De fato, suponha  que $m/n>p/q$, como $n,q>0$, necessariamente $mq>np$, donde $b^{mq}>b^{np}$. Observando que
  $$
    (b^{1/n})^{1/q}=b^{1/nq},
  $$
  pois
  $$
    \bigl((b^{1/n})^{1/q}\bigr)^{nq}=b,
  $$
  {\it a fortiori}
  $$
    b^{m/n}=(b^{mq})^{1/nq}>(b^{np})^{1/nq}=b^{p/q},
  $$
  como afirmamos. Destarte, $b^r\geq\sup B(r)$, mas como $b^r\in B(r)$, obrigatoriamente $b^r=\sup B(r)$, conforme dissemos.
  
  Quarto, devemos considerar dois casos, a saber, $x+y\in Q$ e $x+y\notin Q$. Antes de prosseguirmos para os casos, provemos inicialmente que nenhum $y<b^xb^y$ \'e cota superior do conjunto $B(x+y)$. De fato, suponha que $y<b^xb^y$, da\'\i\ vem que $y/b^y<b^x$, existe, portanto, $r\leq x$, tal que $y/b^y<b^r\leq b^x$. Analogamente existe $s\leq y$, tal que $y/b^r<b^s\leq b^y$, em consqu\^encia existe $t=r+s\leq x+y$, tal que $y<b^t$, provando que $y$ n\~ao pode ser cota superior de $B(x+y)$, particularmente, temos $b^xb^y\leq b^{x+y}$. Suponhamos primeiramente que $x+y\notin Q$, dado $s\in Q$, tal que $s\leq x+y$, necessariamente $s<x+y$, podemos portanto tomar $t\in Q$, tal que $s-x<t<y$, donde vem que 
  $$
    b^s=b^{s-t}b^t\leq b^xb^y
  $$
  como $s\leq x+y$ \'e arbitr\'ario necessariamente $b^{x+y}\leq b^{x}b^y$, como conclu\'\i mos que $b^xb^y\leq b^{x+y}$, conclusivamente $b^{x+y}=b^xb^y$. Por outro lado, suponhamos que $x+y=r\in Q$. Antes, sabe-se que para $u,v\in \R$ \'e bem conhecida a identidade
  $$
    v^n-u^n=(v-u)\sum_{i=0}^{n-1}u^iv^{n-1-i}
  $$
  donde fazendo $v=1$ e $u=(1/b)^{1/n}$ obtemos
  $$
    1-{1\over b}=\Bigl(1-{1\over b^{1/n}}\Bigr)\sum_{i=0}^{n-1}b^{i+1}>n\Bigl(1-{1\over b^{1/n}}\Bigr)
  $$
  donde deriva-se
  $$
    1-{1\over b^{1/n}}<{b-1\over bn}.
  $$
  Portanto,
  $$
    b^r-b^{r-1/n}=b^r(1-b^{-1/n})<{b^{r-1}(b-1)\over n}.
  $$
  Com isto em m\~aos, dado $\varepsilon>0$ existe $s<r$, tal que $b^r-b^s<\varepsilon$. De fato, tome $n>b^{r-1}(b-1)/\varepsilon$, temos necessariamente que existe $s=r-1/n<r$, tal que 
  $$
    b^r-b^s\leq{b^{r-1}(b-1)\over n}<\varepsilon
  $$
  Desta forma, dado $\varepsilon>0$ existe $s<r=x+y$, tal que
  $$
    b^r-\varepsilon<b^s<b^r
  $$
  
  Consideremos $t\in Q$ tal que $s-x<t<y$, assim $s-t<x$ e
  $$
    b^s=b^{s-t}b^t\leq b^xb^y,
  $$
  logo,
  $$
    b^r-\varepsilon<b^s\leq b^xb^y\leq b^{x+y}=b^r
  $$
  como $\varepsilon$ \'e arbitr\'ario, decorre que $b^{x+y}=b^r=b^xb^y$.
  \qed
  
  \'E interessante indagar se uma defini\c c\~ao an\'aloga funcionaria para $0<b<1$. Primeiro devemos observar as pot\^encias de expoente racional $r$. Como \'e f\'acil de se observar, para $0<b<1$, quanto maior for $s\leq r$ menor ser\'a a pot\^encia $b^s$, isto decorre naturalmente observando as pot\^encias do rec\'\i proco de $b$, j\'a tratadas. De fato, suponha que $s<t$, logo $-t<-s$, que por sua vez acarreta
  $$
    b^t=\Bigl({1\over b}\Bigr)^{-t}<\Bigl({1\over b}\Bigr)^{-s}=b^s.
  $$
  A seguir provaremos que ao inv\'es de tormarmos o supremo para a defini\c c\~ao de $b^x$, poder\'\i amos sem problemas considerar o \'\i nfimo.
  
  Definamos para $b\in \R_+$ e $x\in \R$, os conjuntos
  $$
    L(b,x)=\{b^t\in \R:t\leq x\a t\in Q\}\qquad\hbox{Def.}
  $$
  e
  $$
    U(b,x)=\{b^t\in \R:t\geq x\a t\in Q\}\qquad\hbox{Def.}
  $$
  Conformemente temos o 
  
  \theorem{}{\sl%
    Para $0<b<1$, vale 
    $$
      \inf L(b,x)=\sup U(b,x),
    $$
    e para $b>1$, vale
    $$
      \sup L(b,x)=\inf U(b,x).
    $$
  }
  
  \proof Com efeito, seja $0<b<1$, dado $\varepsilon>0$, tomemos $n>(1-b)/b\varepsilon$. Fazendo $v=1/b^{1/n}$ e $u=1$ na identidade 
  $$
    v^n-u^n=(v-u)\sum_{i=0}^{n-1}v^iu^{n-1-i}
  $$
  obtemos
  $$
    {1\over b}-1=\Bigr[\Bigl({1\over b}\Bigr)^{1/n}-1\Bigr]\sum_{i=0}^{n-1-i}\Bigl({1\over b}\Bigr)^{i/n}
  $$
  sabendo que $1/b>1$ temos em consequ\^encia
  $$
    {1-b\over b}>\Bigr[\Bigl({1\over b}\Bigr)^{1/n}-1\Bigr]n
  $$
  donde
  $$
    {1\over b^{1/n}}-1<{1-b\over bn}.
  $$
  Desta forma, tomando $s,r\in Q$, tais que $r\leq x\leq s$ e $s-r<1/n$ temos
  $$
    b^{r-s}-1<b^{-1/n}-1<{1-b\over bn}<\varepsilon
  $$
  donde
  $$
    0\leq\inf L(b,x)-\sup U(b,x)\leq b^r-b^s<b^s\varepsilon<\varepsilon.
  $$
  Como $\varepsilon>0$ \'e arbitr\'ario, necessariamente
  $$
    \inf L(b,x)=\sup U(b,x).
  $$
  Seguidamente, seja $b>1$, dado $\varepsilon>0$, tomemos 
  $$
    n>{(b-1)\sup L(b,x)\over\varepsilon}.
  $$
  Tomando $r,s\in Q$, tais que $r\leq x\leq s$ e $s-r<1/n$ obtemos
  $$
    0<b^{s-r}-1<b^{1/n}-1<{b-1\over n}<{\varepsilon\over\sup L(b,x)}
  $$
  donde segue que
  $$
    0\leq\inf U(b,x)-\sup L(b,x)\leq b^s-b^r<{\varepsilon b^r\over \sup L(b,x)}\leq\varepsilon.
  $$
  Visto que $\varepsilon>0$, \'e arbitr\'ario decorrer\'a que
  $$
    \inf U(b,x)=\sup L(b,x).
  $$
  \qed
  
  Agora relebremos que no teorema fizemos
  $$
    b^x=\sup B(x)=\sup L(b,x).
  $$
  Al\'em disso, vimos que se $x\in Q$, ent\~ao 
  $$
    b^x=\sup L(b,x).
  $$
  Nesta altura, definimos o s\'\i mbolo $b^x$ para $x\in \R$ e $0<b<1$ ou $1<b$, n\~ao \'e dif\'\i cil ver que para $b\in \{0,1\}$ temos $L(b,x)=U(b,x)=\{b\}$. Assim, \'e natural definir $b^x=b$, nestes casos. Desta forma, ficam definidas as pot\^encias de base $b\in \R_+$ qualquer.
  
  Em seguida, provemos o
  
  \theorem{}{\sl%
    Se $b>0$, ent\~ao
    $$
      b^x\Bigl({1\over b}\Bigr)^x=1.
    $$
  }
  
  \proof Para tanto, provemos que as desigualdades $b^x(1/b)^x<1$ e $b^x(1/b)^x>1$ geram contradi\c c\~oes. Provaremos apenas para o caso $0<b<1$, pois o caso $b>1$ pode ser provado analogamente considerando-se $a=1/b$ e conforme mencionado $1^x=1$. 
  
  Se $b^x(1/b)^x<1$, ent\~ao $(1/b)^x<1/b^x$, existe portanto $r\in Q$, tal que $r\geq x$ e 
  $$
    \Bigl({1\over b}\Bigr)^x\leq\Bigl({1\over b}\Bigr)^r<{1\over b^x},
  $$
  donde $b^x<b^r$. Segue igualmente que existe $s\in Q$ tal que $s\leq x$ e $b^x\leq b^s<b^r$, da\'\i\ vem $b^{r-s}>1$, contradizendo a hip\'otese $b<1$. Analogamente, se $b^x(1/b)^x>1$, ent\~ao existe $r\in Q$, tal que $r\leq x$ e 
  $$
   {1\over b^x}<\Bigl({1\over b}\Bigr)^r\leq\Bigl({1\over b}\Bigr)^x.
  $$
  Destarte, existe $s\in Q$ tal que $s\geq x$ e $b^r<b^s\leq b^x$, donde inferimos que $b^{s-r}>1$, outra contradi\c c\~ao com o fato de $b<1$.
  \qed
  
  \theorem{\ (Uma defini\c c\~ao de logaritmo)}{\sl%
    Se $b>1$ e 
    $$
      x=\sup\{w\in \R: b^w<y\},
    $$
    ent\~ao $b^x=y$. Alem disso, $x$ \'e o \'unico numero real tal que $b^x=y$.
  }
  
  \proof
  \qed
  
  \noindent{\bf Exerc\'\i cio 20 do cap\'\i tulo 1 de [Rudin].\ }{\sl%
    Provisoriamente diremos que $\alpha\subset Q$ \'e um corte se os seguintes axiomas s\~ao satisfeitos
    
    \bigskip
    I. $\emptyset\neq\alpha\neq Q$;
    
    II. $\forall p,q(p\in\alpha\a q\in Q\a q<p\longrightarrow q\in\alpha).$
    \bigskip
    
    Seja $R$ a cole\c c\~ao de todos os cortes, provemos que $R$ munido com a ordem
    $$
      \alpha<\beta\longleftrightarrow\alpha\subset\beta\quad\hbox{Def.}
    $$
    (em que $\subset$ \'e a inclus\~ao pr\'opria) tem a propriedade da menor cota superior. Ademais, se 
    $$
      \alpha+\beta=\{p+q\in Q:(p,q)\in\alpha\times\beta\}\quad\hbox{Def.}
    $$
     ent\~ao A1--A4 s\~ao satisfeitas, mas A5 falha.
  }
  
  \proof Seja $S\subseteq R$ um conjunto n\~ao vazio limitado superiormente, fa\c ca\-mos $\varsigma=\bigcup S$, provemos que $\varsigma$ \'e a menor das cotas superiores de $S$. Com efeito, primeiro provemos que $\varsigma$ satisfaz I e II. De fato, sabendo que $S\neq\emptyset$, existe $\alpha\subset\varsigma$ tal que $\alpha\neq\emptyset$, em consequ\^encia $\varsigma\neq\emptyset$. Seja $\beta$ uma cota superior de $S$ que \'e garantida das hip\'oteses. Dado $\alpha\in S$, inferimos que $\alpha\subseteq\beta$, desta maneira $\varsigma\subseteq\beta$. Assim, basta notar $Q\setminus\varsigma\supseteq Q\setminus\beta\neq\emptyset$, em outros termos $\varsigma\neq Q$, logo vale I. Seja agora, $p\in\varsigma$ e $q\in Q$ com $q<p$, segue que existe $\alpha\in S$, tal que $p\in\alpha$, como este \'ultimo \'e um corte $q\in\alpha$, como $\alpha\subseteq\varsigma$, segue que $q\in\varsigma$, fica portanto provada II. Como $\varsigma$ esta contida em toda cota superior de $S$, decorre que $\sup S=\varsigma$. Consequentemente $R$ tem a propriedade da menor cota superior.
  
  Dados $\alpha,\beta\neq\emptyset$, necessariamente $\alpha+\beta\neq\emptyset$. Agora se $r\notin\alpha$ e $s\notin\beta$, ent\~ao $r+s$ majora estritamente todo elemento de $\alpha+\beta$, consequentemente $r+s\notin\alpha+\beta$, e portanto $\alpha+\beta\neq Q$, fica provadado portanto que a soma satisfaz I. Sejam agora $r\in\alpha+\beta$ e $s\in Q$ com $s<r$, segue que existe $(p,q)\in\alpha\times\beta$, tal que $r=p+q$. Assim, $s<p+q$, logo $s-p<q$, isto \'e, $s-p\in\beta$, da\'\i\ vem que 
  $$ 
    s=p+(s-p)\in\alpha+\beta.
  $$
  o que prova II. Isto prova que adi\c c\~ao de cortes \'e est\'avel, em outros termos A1.
  
  Trivialmente valem A2 e A3. 
  
  Cosideremos agora 
  
  $$
    0^*=\{p\in Q:p\leq 0\}\quad\hbox{Def.}
  $$
  Provemos que $0^*$ \'e um corte. De fato, trivialmente $\emptyset\neq0^*\neq Q$. E, se $p\in0^*$ e $q\in Q$ com $q<p$, ent\~ao $q\leq0$, i.e., $q\in0^*$, portanto, $0^*$ \'e um corte. 
  
  Seja $\alpha\in R$ e $(p,q)\in0^*\times\alpha$ temos que $p+q\leq q\in\alpha$, consequentemente $p+q\in\alpha$, i.e., $0^*+\alpha\subseteq\alpha$. Por outro lado, seja $p\in\alpha$, trivialmente $p=0+p\in0^*+\alpha$, i.e., $\alpha\subseteq0^*+\alpha$. Destarte $\alpha=0^*+\alpha$, como $\alpha$ \'e arbitr\'ario segue que $0^*$ \'e o elemento neutro para adi\c c\~ao.
  
  Dado $p$, n\~ao \'e dif\'\i cil ver que 
  $$
    \alpha=\{p\in Q:p<q\}
  $$
  \'e um corte. Suponhamos que existisse $\beta\in R$, tal que $\alpha+\beta=0^*$, ent\~ao necessariamente existiria $(r,s)\in\alpha\times\beta$, tal que $r+s=0$, da\'\i\ $s=-r\in\beta$. Consideremos $t\in Q$ tal que $r<t<q$, temos que $t\in\alpha$, mas $s+t>0$, i.e., existe $(t,s)\in\alpha\times\beta$, tal que $t+s\notin 0^*$, contradizendo a igualdade $\alpha+\beta=0^*$.
  \qed
  
  \lemma{}{%
    \sl seja $\pi$ uma fun\c c\~ao sentencial cujo universo de discurso \'e o subconjunto dos n\'umeros naturais $i$ tais que $i\leq n+1$. Ademais, suponha que 
    \medskip
  
    I. $\pi(0)$ \'e verdadeira;
  
    II. Se $i\leq n$ e $\pi(i)$ \'e verdadeira, ent\~ao $\pi(i+1)$ \'e verdadeira.
    \medskip
  
    Ent\~ao $\pi$ \'e verdadeira para todo natural $i\leq n+1$.
  }
  
  \proof Use o fato de $N$ ser bem ordenado.\qed
  
  \noindent{\bf Exerc\'\i cio 24 do livro {\it Calculus} de M. Spivak.} Consideremos dada uma sequ\^encia de n\'umeros $\{a_n\}$, denotaremos uma soma qualquer dos $n\geq 1$ primeiros termos, aparacendo na ordem natural, por $s(a_1,\ldots,a_n)$, em outras palavras, $a_i$ vem antes de $a_{i+1}$ na express\~ao da soma. 
  
  Definiremos indutivamente,
  $$
    S_0=0\quad\hbox{e}\quad S_{k+1}=a_{n-k}+S_k.
  $$
  Desta forma, intuitivamente podemos considerar definida a soma
  $$
    a_1+(a_2+(a_3+(\ldots+(a_{n-1}+a_n)))),
  $$
  tal soma \'e na realidade $S_n$. Denotaremos tal soma por $S(a_1,\ldots,a_n)$, para indicar quais termos a soma depende. No seu livro, {\it Calculus}, Spivak define 
  $$
    a_1+\ldots+a_n=S(a_1,\ldots,a_n)\qquad\hbox{Def.}
  $$
  e prossegue o exerc\'icio usando esta simbologia. No entanto, nos ateremos \`a nota\c c\~ao $S(a_1,\ldots,a_n)$ para se referir aquela soma.
  
  Afirmo que para todo $n\in N$
  $$
    S(a_1,\ldots,a_{n+2}+a_{n+3})=S(a_1,\ldots,a_{n+3}).
  $$
  A soma no membro esquerdo da \'ultima igualdade pode ser reescrita como
  $$
    S(b_1,\ldots,b_{n+2})
  $$
  em que $b_i = a_i$, para todo $i\leq n+1$ e $b_{n+2}=a_{n+2}+a_{n+3}$.
  Esta soma por sua vez \'e dada indutivamente por
  $$
    S_0=0\quad\hbox{e}\quad S_{j+1}=b_{n+2-j}+S_{j}.
  $$
  E a soma $S(a_1,\ldots,a_{n+2})$ \'e dada indutivamente por
  $$
    S'_0=0\quad\hbox{e}\quad S'_{j+1}=a_{n+3-j}+S'_{j}.
  $$
  Seguidamente, provemos indutivamente que
  $$
    S'_{j+2}=S_{j+1},
  $$
  para todo $j\leq n+1$.
  
  Primeiro, para $j=0$ temos
  $$
    S'_2=a_{n+2}+S'_1=a_{n+2}+a_{n+3}=b_{n+2}=S_1.
  $$
  Portanto, a propriedade \'e v\'alida para $j=0$. Suponhamos que ela seja v\'alida para $j\leq n$. Temos em seguida que
  $$
    S'_{j+3}=a_{n+1-j}+S'_{j+2}
  $$
  e
  $$
    S_{j+2}=b_{n+1-j}+S_{j+1}.
  $$
  Como sabemos por hip\'otese de indu\c c\~ao que $S'_{j+2}=S_{j+1}$ e tamb\'em sabemos que $a_{n+1-j}=b_{n+1-j}$, para $0\leq j\leq n$, temos em conformidade
  $$
    S'_{j+3}=a_{n+1-j}+S'_{j+2}=b_{n+1-j}+S_{j+1}=S_{j+2}
  $$
  portanto a propriedade \'e v\'alida para $j+1$. Pelo princ\'ipio de indu\c c\~ao matem\'atica (sob a forma do lema anterior) a propriedade \'e v\'alida para todo $j\leq n+1$. Em particular,
  $$
    S(a_1,\ldots,a_{n+3})=S'_{n+3}=S_{n+2}=S(a_1,\ldots,a_{n+2}+a_{n+3}).
  $$
  
  Em seguida, provemos indutivamente que para todo $n\in N$
  $$
    S(a_1,\ldots,a_{n+1})+a_{n+2}=S(a_1,\ldots,a_{n+2}).
  $$
  
  Inicialmente, verifiquemos que
  $$
    S(a_1)+a_2=a_1+a_2=S(a_1,a_2),
  $$
  i.e., a propriedade \'e v\'alida para $n=0$. Suponhamos por hip\'otese de indu\c c\~ao que ela seja v\'alida para $n$, temos em conformidade
  $$
    \vbox{
      \halign{
        \hfill#&#\hfill\cr
        $S(a_1,\ldots,a_{n+2})+a_{n+3}$ & $=\hphantom{^\dagger} \bigl(S(a_1,\ldots,a_{n+1})+a_{n+2}\bigr)+a_{n+3}$ \cr
                                        & $=\hphantom{^\dagger} S(a_1,\ldots,a_{n+1})+(a_{n+2}+a_{n+3})$ \cr
                                        & $=^{\dagger} S(b_1,\ldots,b_{n+2})$ \cr
                                        & $=^{\ddagger} S(a_1,\ldots,a_{n+3})$ \cr
      }
    }
  $$
  
  \noindent($\dagger$) Fazendo $b_k=a_k$, para $k\leq n+1$ e $b_{n+2}=a_{n+2}+a_{n+3}$;
  
  \noindent($\ddagger$) Pelo que j\'a foi provado.
  
  Fica portanto provado que a propriedade \'e v\'alida para $n+1$, novamente pelo princ\'ipio de indu\c c\~ao matem\'atica segue que a propriedade \'e v\'alida para todo $n\in N$.
  
  Imediatamente provemos que para todo $l\in N$ tal que $l\geq1$ e para todo $n\geq l+1$
  $$
    S(a_1,\ldots,a_l)+S(a_{l+1},\ldots,a_n)=S(a_1,\ldots,a_n).
  $$
  Para $l=1$ e $n\geq 2$ certamente
  $$
    S(a_1)+S(a_2,\ldots,a_n)=a_1+S(a_2,\ldots,a_n)=S(a_1,\ldots,a_n)
  $$
  Suponhamos por hip\'otese de indu\c c\~ao que a propriedade \'e v\'alida para $l$, temos para $n\geq l+2$
  $$
    \vbox{
      \halign{
        \hfill#&#\hfill\cr
        $S(a_1,\ldots,a_{l+1})+S(a_{l+2},\ldots,a_{n})$ & $=\bigl(S(a_1,\ldots,a_l)+a_{l+1}\bigr)+S(a_{l+2},\ldots,a_{n})$ \cr
                                                        & $=S(a_1,\ldots,a_l)+\bigl(a_{l+1}+S(a_{l+2},\ldots,a_{n})\bigr)$ \cr
                                                        & $=S(a_1,\ldots,a_l)+S(a_{l+1},\ldots,a_{n})$ \cr
                                                        & $=S(a_1,\ldots,a_n),$ \cr
      }
    }
  $$
   i.e., a propriedade \'e v\'alida para $l+1$, em consequ\^encia pelo princ\'ipio de indu\c c\~ao matem\'atica ela ser\'a v\'alida para todo $l\in N$.
  
  Consideremos uma soma arbitr\'aria com pelo menos dois elementos $s(a_1,\ldots,a_n)$, conforme estipulado no in\'icio da resolu\c c\~ao do problema. Para esta soma arbitr\'aria temos duas possibilidades
  
  I. A soma come\c ca ou termina com um termo, neste caso temos
  $$
    s(a_1,\ldots,a_n)=a_1+s(a_2,\ldots,a_n)
  $$
  ou
  $$
    s(a_1,\ldots,a_n)=s(a_1,\ldots,a_{n-1})+a_n.
  $$
  
  Antes de exibirmos o outro caso, estipularemos que se um par\^entese ocorre entre par\^enteses correspondentes, ent\~ao o seu correspondente ocorrer\'a entre os par\^enteses correspondentes.
  
  II. A soma come\c ca e termina com par\^enteses digamos `$(_1$' e `$)_2$', respectivamente. Seja `$)_1$' e `$(_2$' os correspondentes de `$(_1$' e `$)_2$', respectivamente. Ap\'os $)_1$ deve vir necessariamente um s\'imbolo `+', pois a outra possibilidade seria um par\^entese `$)_3$', em conformidade com nossa estipula\c c\~ao `$(_1$' deve vir entre `$(_3$' e `$)_3$', mas isto contradiria o fato de que `$(_1$' \'e o primeiro par\^entese. Ap\'os `+' deve vir um par\^entese `$(_3$', este par\^entese deve coincidir com `$(_2$', pois do contr\'ario `$(_2$' e `$)_2$' viriam entre `$(_3$' e `$)_3$' contradizendo o fato de `$)_2$' ser o \'ultimo par\^entese.
  
  Em qualquer caso antes existe um s\'imbolo `+' que separa a primeira soma em duas somas. Seja $m$ menor natural tal que $a_m$ est\'a do lado direito do s\'imbolo `+', decerto que $2\leq m\leq n$, logo existe $l\in N$, tal que $l+1=m$; assim
  
  $$
    s(a_1,\ldots,a_n)=s(a_1,\ldots,a_l)+s(a_{l+1},\ldots,a_n).
  $$
  
  Seguidamente provemos que para todo $n\in N$ tal que $n\geq 2$ vale
  $$
    \pi(n)\equiv s(a_1,\ldots,a_n)=S(a_1,\ldots,a_n).
  $$
  Certamente que $\pi(2)$ \'e verdadeira. Suponhamos que $\pi$ seja verdadeira para todo $2\leq i\leq n$, pelo que vimos, para algum $l\leq n$
  $$
    \vbox{
      \halign{
        \hfill#&#\hfill\cr
        $s(a_1,\ldots,a_{n+1})$ & $=s(a_1,\ldots,a_l)+s(a_{l+1},\ldots,a_{n+1})$ \cr
                                & $=S(a_1,\ldots,a_l)+S(a_{l+1},\ldots,a_{n+1})$ \cr
                                & $=S(a_1,\ldots,a_{n+1})$ \cr
      }
    }
  $$
  Em consequ\^encia a propriedade \'e v\'alida para $n+1$, consequentemente pelo princ\'ipio de indu\c c\~ao matem\'atica $\pi$ \'e v\'alida para todo $n\in N$ tal que $n\geq 2$.
  \qed
  
  \definition{}{\sl Um conjunto $C$ \'e finito se existe um $n\in\omega$, tal que $C\sim n$. Denotaremos o natural $n$, tal que $n\sim C$, por $\#C$.}
  
  \theorem{}{\sl Para todo $n\in\omega$, se $C\subseteq n$, ent\~ao $\#C\leq n$.}
  
  \proof Trivialmente v\'alida para $n=0$. Suponhamos que seja v\'alida para $n$, provemos que ser\'a v\'alida para $n+1$. Temos duas possibilidades a considerar, $n\notin C$ ou $n\in C$. No primeiro caso temos $C\subseteq n$, da\'i pela hip\'otese de indu\c c\~ao $\#C\leq n$. Se $n\in C$, ent\~ao $C\setminus\{n\}\subseteq n$, da\'i $\#(C\setminus\{n\})\leq n$, donde $\#C\leq n+1$, como quer\'iamos provar. Do princ\'ipio de indu\c c\~ao matem\'atica vale a condicional para todo $n\in\omega$.\qed
  
  \theorem{}{\sl Nenhum conjunto finito \'e equivalente a um subconjunto pr\'oprio.}
  
  \proof Suponhamos que exista um conjunto finito $C$ que \'e equivalente a um subconjunto pr\'oprio $S$. Sabemos que existe $n\in\omega$, tal que $n+1=\#C$, pois $C$, deve conter ao menos um elemento para possuir subconjuntos pr\'oprios. Existe $a\in C\setminus S$, donde vem que $S\subseteq C\setminus\{a\}$, que por sua vez implica $\#S\leq\#(C\setminus\{a\})=n$, mas por outro lado $\#S=\#C=n+1$, da\'i vem a desigualdade
  $$
    n+1=\#S\leq n
  $$
  o que \'e um absurdo.
  
  \corollary{}{\sl Uma condi\c c\~ao suficiente para que um conjunto $C$ seja infinito, i.e., n\~ao existir $n\in\omega$, tal que $C\sim n$, \'e que ele seja equivalente a um subconjunto pr\'oprio.}
  
  \proof Trival.\qed
  
  Nas condi\c c\~oes do corol\'ario o conjunto $C$ \'e dito Dedekind-infinito. Usando o axioma da escolha \'e poss\'ivel provar que todo conjunto infinito no primeiro sentido \'e Dedekind-infinito, portanto sob o axioma da escolha ser infinito no primeiro sentido \'e indiferente de ser Dedekind-infinito.
  
  \theorem{}{\sl Sejam $C$ um conjunto infinito e $f:\omega\rightarrow C$ uma sobreje\c c\~ao. Ent\~ao $C$ \'e cont\'avel.}
  
  \proof Para todo $c\in C$, podemos escolher $\eta(c)\in f^{-1}({c})$. Sabemos que $A=\{\eta(c)\in\omega:c\in C\}\sim C$, pois $f|_A$ \'e uma bije\c c\~ao. Como $C$ \'e infinito, $A$ tamb\'em o \'e. Dessarte, $A$ \'e cont\'avel pois $A\subseteq\omega$, consequentemente $C$ \'e cont\'avel.
  
  \theorem{}{\sl $\omega\times\omega\sim \omega$.}
  
  \proof Defina $\phi:\omega\times\omega\rightarrow\omega$ por $\phi(m,n)=2^m(2n+1)$. \'E sabido que todo n\'umero pode ser fatorado num produto de n\'umeros primos, e fatores que n\~ao s\~ao $2$ s\~ao \'impares e produto de \'impares \'e \'impar, consequentemente o produto de fatores que n\~ao s\~ao $2$ \'e um n\'umero da forma $2n+1$, para $n\in\omega$. Portanto, para todo $l$ existe $(m,n)\in\omega\times\omega$ tal que $l=2^m(2n+1)$. Assim, $\phi$ \'e sobrejetiva. A injetividade n\~ao \'e dif\'icil, basta  supor que $2^m(2n+1)=2^p(2q+1)$, sem perdas suponhamos que $m<p$, da\'i vem $2n+1=2^{p-m}(2p+1)$, donde conluir\'iamos que $1=2r$ para $r\in\omega$, o que \'e imposs\'ivel em $\omega$. Da\'i $m=p$ e consequentemente $n=q$, logo $\phi$ \'e injetiva, portanto bijetiva, fica, portanto, provado o teorema.\qed
  
  \corollary{}{\sl Uma reuni\~ao de uma fam\'ilia cont\'avel com membros cont\'aveis \'e cont\'avel.}
  
  \proof Podemos supor que nossa fam\'ilia \'e $\{E_n:n\in\omega\}$, para cada $n$ existe uma fun\c c\~ao $f_n:\omega\rightarrow E_n$. Agora basta definirmos $f:\omega\times\omega\rightarrow U$, por $f(m,n)=f_m(n)$, com $U=\bigcup_{n\in\omega}E_n$. Como $f\circ\phi^{-1}:\omega\rightarrow U$ ($\phi$ dada no teorema anterior) \'e sobrejetiva e $U$ infinito, ent\~ao $U$ \'e cont\'avel.
  
  \theorem{}{\sl%
    Seja $\mathscr A$ uma cole\c c\~ao no m\'aximo cont\'avel de membros no m\'aximo cont\'aveis. Ent\~ao $\bigcup\mathscr A$ \'e no m\'aximo cont\'avel.
  }
  
  \proof Das hip\'oteses existe $I\in\omega\cup\{\omega\}$ e uma bije\c c\~ao $\kappa:I\rightarrow\mathscr A$, digamos que $A_i=\kappa(i)$. Para cada $i\in I$, existe uma bije\c c\~ao $f_i:B_i\rightarrow A_i$, com $B_i\in\omega\cup\{\omega\}$. Definamos $f\subseteq(\omega\times\omega)\times A$, com $A=\bigcup\mathscr A$ por
  $$
    ((m,n),a)\in f\leftrightarrow a=f_m(n)
  $$
  Ent\~ao $f$ \'e uma fun\c c\~ao. De fato, suponhamos que $((m,n),a),((m,n),b)\in f$, ent\~ao da defini\c c\~ao $a=f_m(n)=b$. Assim, $f:\hbox{dom}\,f\rightarrow A$, \'e uma sobreje\c c\~ao. Em verdade, dado $a\in A$, existe $i\in I$, tal que $a\in A_i$, como $f_i$ \'e uma bije\c c\~ao existe $m\in B_i$, tal que $f_i(m)=a$, consequentemente $f(i,m)=a$. Usando o axioma da escolha podemos construir um conjunto $C\subseteq\omega\times\omega$ tal que $f|_C:C\rightarrow A$ \'e uma bije\c c\~ao. Como todo conjunto de $\omega\times\omega$ \'e no m\'aximo cont\'avel decorrer\'a por maior raz\~ao que $A=\bigcup\mathscr A$ tamb\'em o ser\'a.

  \chapter{An\'alise matem\'atica}
  
  \theorem{}{\sl%
    Sejam $\mathscr O$ um conjunto com pelo menos dois elementos, linearmente (totalmente) ordenado segundo $<$ (uma ordem estrita), e $\mathscr B$ uma cole\c c\~ao de elementos da forma:
    $$
      \vbox{
        \halign{
          \hfill#&\hfill#&#\hfill\cr
          I.   & $(a,b)$ & $=\{x\in\mathscr{O}:a<x<b\}$;\cr
          II.  & $[l,b)$ & $=\{x\in\mathscr{O}:l\le x<b\}$;\cr
          III. & $(a,g]$ & $=\{x\in\mathscr{O}:a<x\le g\}$;\cr
        }
      }
    $$
    em que $a<b$ e $l=\min\mathscr O$ e $g=\max\mathscr O$, se houverem. Caso contr\'ario, os respectivos conjuntos II--III n\~ao figuram na cole\c c\~ao $\mathscr B$. Ent\~ao $\mathscr B$ \'e uma base para uma topologia em $\mathscr O$.
  }
  
  \proof Seja $x\in\mathscr O$, temos duas possibilidades $x\in\{l,g\}$, neste caso \'e evidente que existe um elemento b\'asico contendo $x$. Caso contr\'ario, devem existir $a,b\in\mathscr O$, tais que $a<x<b$, i.e., $x\in(a,b)$. Sejam $B_i\in\mathscr O$ , $i=1,2$ e $x\in B_1\cap B_2$, tome $a=\max_{i=1,2}a_i$ e $b=\min_{i=1,2}b_i$, em que $a_i$ \'e o extremante inferior de $B_i$ e $b_i$ o extremante superior de $B_i$, temos tr\^es possibilidades mutualmente excludentes $a,b\notin\{l,g\}$ ou $a=l$ ou $b=g$. No primeiro caso $x\in B=(a,b)$, no segundo $x\in B=[l,b)$ e no terceiro $x\in B=(a,g]$ e em todos os casos $B\subseteq B_1\cap B_2$ com $x\in B\in\mathscr B$. Diante do que foi discorrido, $\mathscr B$ \'e uma base de $\mathscr O$.
  
  Tal topologia em um conjunto totalmente ordenado com pelo menos dois elementos $\mathscr O$, \'e chamada de {\bf topologia da ordem}, ou simplesmente {\bf topologia ordem}.
  
  \corollary{}{\sl%
    Seja $R^\#=R\cup\{-\infty,\infty\}$, munido com a ordem $<$ satisfazendo $-\infty<\infty$, $-\infty<a<\infty$ para todo $a\in R$ e tal que $<$ restrita a $R$ \'e a ordem usual em $R$ \'e um espa\c co topol\'ogico.
  }
  
  \proof Primeiro verifiquemos que a ordem \'e linear (ou total). Trata-se de um ordem estrita, portanto devemos atestar que vale a n\~ao reflexividade, a transitividade e a comparabilidade. Pela constru\c c\~ao $a<a$, para todo $a\in R^\#$. Suponhamos que $a<b$ e $b<c$ se $a=-\infty$ ou $c=\infty$ \'e imediato que $a<c$. Caso contr\'ario $a,c\in R$ e da\'i segue da transitividade da ordem em $R$ que $a<c$. Pela pr\'opria constru\c c\~ao $<$ \'e total. Como consequ\^encia $<$ induz a topologia ordem sobre $R^\#$.\qed
  
  N\~ao \'e dif\'icil ver para quaisquer dois elementos $x\neq y$, existem vizinhan\c cas $V_x$ e $V_y$ de $x$ e $y$, respectivamente, tais que $V_x\cap V_y=\emptyset$, i.e., $R^\#$ tem a propriedade de Hausdorff ou satisfaz o axioma ${\rm T}_1$. Como consequ\^encia disto podemos formular o limite de sequ\^encias em $R^\#$. 
  
  
  \definition{}{\sl
   Sejam $X$ um espa\c co topol\'ogico e $x\in X^\omega$, vamos fazer $x(n)=x_n$ e designar $x$ pela fam\'ilia $\{x_n\in X:n\in\omega\}$, denotaremos-la por $\{x_n\}$. Diremos que $\{x_n\}$ converge para $p\in X$, se e somente se, para toda vizinhan\c ca $V$ de $p$, existe $n_V\in\omega$, tal que se $n\ge n_V$, ent\~ao $x_n\in V$. 
  }
  
  
  \theorem{}{\sl
     Seja $X$ um espa\c co topol\'ogico Hausdorff. Ent\~ao, se $\{x_n\}$ converge, ent\~ao convegir\'a para um \'unico elemento $p\in X$. 
  }
  
  \proof Suponha que $\{x_n\}$ converge para $p\in X$, e $q\in X$ com $q\neq p$, existem vizinhan\c cas $V_p$ e $V_q$, de $p$ e $q$ respectivamente, tais que $V_p\cap V_q=\emptyset$. Evidentemente o conjunto $\{n\in\omega:x_n\notin V_q\}$ \'e infinito, consequentemente $\{x_n\}$, n\~ao converge para $q$. 
  
  Uma outra maneira de verificar isto, \'e supor que $\{x_n\}$ convirga para $p$ e $q$, concomitamente. Tomemos $V_p$ e $V_q$ como anteriormente. Da defini\c c\~ao, existe $n\in\omega$, tal que $x_n\in V_p\cap V_q=\emptyset$, o que \'e um absurdo.\qed
  
  \medskip
  
  \definition{}{\sl
    Seja $X$ um espa\c co topol\'ogico Hausdorff. Se $\{x_n\}$ converge para $p\in X$, ent\~ao escreveremos
    $$
      \lim_nx_n=p.
    $$
  }
  
  \theorem{}{\sl
    Sejam $X$ um espa\c co topol\'ogico qualquer, $E=\{x_n\}\subseteq X$, e $p\in X$ um ponto limite de $E$, tal que existe um sistema de vizinhan\c cas $\mathscr{S}=\{V_n:n\in\omega\}$ de $p$, com as seguintes propriedades:
    \begin{enumerate}[label = \Roman*.]
      \item{
        $\forall n(n\in\omega\longrightarrow V_{n+1}\subset V_n)$;
      }
      \item{%
        Para toda vizinhan\c ca $V$ de $p$, existe um $n\in\omega$, tal que $V_n\subset V$.
      }
    \end{enumerate}
    Ent\~ao existe uma subsequ\^encia $\{x_{n_k}\}$ convergindo para $p$. 
  }
  
  \proof Seja $n_1=\min\{n\in\omega:x_n\in V_1\}$, suponha que $n_i$, $i=1,\ldots,k$, estejam constru\'idos seja $n_{k+1}=\min\{n\in\omega: x_n\in V_{k+1}\wedge n>n_k\}$. Desta forma $\{x_{n_k}\}$ converge a $p$. De fato, seja $V$ uma vizinhan\c ca de $p$, consideremos $K\in\omega$, tal que $V_K\subset V$, \'e evidente que $x_{n_k}\in V_k\subseteq V_K\subset V$, para todo $k\ge K$. Portanto, $\{x_{n_k}\}$ converge a $p$.\qed
  
  \corollary{}{\sl
    Sejam $X$ um espa\c co topol\'ogico Hausdorff, tal que todo ponto admite um sistema de vizinhan\c cas satisfazendo as duas condi\c c\~oes do teorema anterior. Se $\{x_n\}\subseteq X$, ent\~ao o conjunto $S$ dos limites subsequenciais \'e fechado.
  }
  
  \proof H\'a duas possibilidades $S$ tem ou n\~ao tem pontos limites. No primeiro caso $S$ \'e trivialmente fechado. Se $S$ tem pontos limites, seja $p$ um deles. Seja $V_p$ uma vizinhan\c ca de $p$, da hip\'otese existe $q\in V_p\cap S$, com $q\neq p$, da hip\'otese do espa\c co ser Hausdorff existe uma vizinhan\c ca $V_q$ de $q$ tal que $p\notin V_q$, podemos supor sem perda de generalidade que $V_q\subset V_p$, pois basta tomar $V_p\cap V_q$. Necessariamente existe $n\in\omega$ tal que $x_n\in V_q$, pois $q$ \'e um limite subsequencial de $\{x_n\}$, consequentemente $x_n\in V_p\setminus\{p\}$. Como $V_p$ \'e uma vizinhan\c ca arbitr\'aria de $p$, decorre que $p$ \'e um ponto limite de $\{x_n\}$. Do teorema anterior, $p\in S$. Assim, provamos que $\overline{S}\subseteq S$, consequentemente $S=\overline{S}$.\qed    
  
  
  \corollary{}{\sl
    Seja $\{x_n\}\subseteq R^\#$. Ent\~ao o conjunto $S$ dos limites subsequenciais de $\{x_n\}$ \'e um conjunto fechado.
  }
  
  \proof N\~ao \'e dif\'icil ver que para todo ponto de $R^\#$ existe um sistema de vizinhan\c cas com aquela propriedade.\qed
  
  \theorem{}{\sl
    Sejam $\{x_n\}\subset R^\#$ e $S$ o conjunto dos limites subsequenciais de $\{x_n\}$. Ent\~ao $S\neq\emptyset$.
  }
  
  \proof Consideremos os conjuntos $A=\{n\in\omega:x_n=-\infty\}$, $B=\{n\in\omega:x_n=\infty\}$ e $C=\{n\in\omega:x_n\in R\}$. Um deles \'e infinito, se $A$ ou $B$ o forem, ent\~ao \'e evidente que $-\infty\in S$ ou $\infty\in S$. No caso restante $C$ \'e limitado ou n\~ao, caso $C$ seja limitado, ele est\'a num conjunto compacto, consequentemente uma subsequ\^encia convergir\'a para $l$. Caso contr\'ario, $-\infty$ ou $\infty$ s\~ao pontos limites de $C$ portanto de $\{x_n\}$. Assim, em qualquer caso $S\neq\emptyset$.\qed
  
  Aqui cabe uma digress\~ao. Este teorema pode ser provado diretamente. Primeiro $R^\#$ \'e compacto, este fato com alguns casos especiais de sequ\^encias com imagem finita, garantem que toda sequ\^encia de $R^\#$ admite uma subsequ\^encia convergente. Portanto sempre $S\neq\emptyset$. Ademais, se considerarmos um espa\c co topol\'ogico compacto Hausdorff, com um sistemas de vizinhan\c cas satisfazendo as condi\c c\~oes I-II de um dos teoremas anteriores, podemos inferir que $S$ o cojunto dos limites subsequenciais \'e sempre n\~ao vazio. Precipuamente este resultado se baseia na proposi\c c\~ao de que um conjunto infinito em um espa\c co topol\'ogico compacto, sempre admite um ponto limite. 
  
  Ap\'os escrever as demonstra\c c\~oes dos teoremas eu consultei o GPT e um livro de topologia, a saber, o {\sl Topology} de James R. Munkres, e percebi que um espa\c co topol\'ogico com um sistema de vizinhan\c cas com aquelas propriedades \'e essencialmente um espa\c co que satisfaz o {\bf primeiro axioma de contabilidade} tamb\'em chamado de {\bf primeiro-cont\'avel}. Este axioma diz que todo ponto admite um sistema de vizinhan\c cas cont\'avel, como consequ\^encia, podemos construir um sistema de vizinhan\c cas cont\'avel descendente, como no teorema. 
  
  \definition{}{\sl%
    Sejam $\{x_n\}\subset R^\#$ e $S$ o conjunto dos limites subsequenciais de $\{x_n\}$. Ent\~ao definimos
    $$
      \liminf x_n=\inf S\quad\hbox{e}\quad\limsup x_n=\sup S.
    $$
  }
  
  \theorem{}{\sl%
    Uma sequ\^encia $\{x_n\}$ em $R^\#$ \'e convergente, se e somente se, $\liminf x_n=\limsup x_n$.
  }
  
  \proof A condi\c c\~ao do espa\c co  ser Hausdorff \'e suficiente para que o conjunto $S$ dos limite subsequenciais degenere a um \'unico ponto caso a sequ\^encia convirja. Se $\liminf x_n=\limsup x_n$ ent\~ao $S$ \'e necessariamente unit\'ario, logo toda subsequ\^encia converje para um \'unico ponto $l\in S$, em particular a pr\'opria sequ\^encia converge para $l$.\qed
  
  \theorem{}{\sl
    Se $\emptyset\neq E\subseteq R^\#$, ent\~ao $\inf E,\sup E\in \overline{E}$.
  }
  
  \proof Seja $\varsigma=\sup E$, temos duas possibilidades $\varsigma\in E$ ou $\varsigma\notin E$. Se $\varsigma\in E$, ent\~ao $\varsigma\in\overline{E}$. Se $\varsigma\notin E$, consideremos um elemento b\'asico $B$ contendo $\varsigma$, existe $a<\varsigma$, pois $E\neq\emptyset$, tal que $\emptyset\neq(a,\varsigma)\cap E\subset B\cap E$. Como $B$ \'e arbitr\'ario decorre que $\varsigma$ \'e um ponto limite. Portanto, $\varsigma\in\overline{E}$. Com um argumento sim\'etrico prova-se que $\inf E\in\overline{E}$.\qed
  
  Em particular, nas condi\c c\~oes dos teoremas anteriores  
  $$
    \liminf x_n,\,\limsup x_n\in S.
  $$
  
  \theorem{}{\sl%
    Se $\{x_n\}\subset R^\#$, ent\~ao
    $$
      \liminf x_n=\sup_{n\in\omega}\inf_{k\ge n} x_k\quad\hbox{e}\quad\limsup x_n=\inf_{n\in\omega}\sup_{k\ge n} x_k.
    $$
  }
  
  \proof Provemos a primeira igualdade, para tanto fa\c camos $b_n=\inf_{k\ge n}x_k$ e $\iota=\sup_nb_n$, da\'i segue imediatamente que qualquer limite subsequencial \'e maior que ou igual a $b_n$, portanto $b_n\le \liminf x_n$, para todo $n\in\omega$, consequentemente $\iota\le\liminf x_n$. Suponhamos por redu\c c\~ao ao absurdo que exista $p\in R$, tal que $\iota<p<\liminf x_n$. Seja $n_0=\min\{n\in\omega:x_n<p\}$ e supondo definidos $n_i$, com $i\in k+1$, defina $n_{k+1}=\min\{n\in\omega:x_n<p\wedge n_k<n\}$, observe que estes conjuntos onde tomamos o m\'inimo s\~ao n\~ao vazios pois $b_n<p$, para todo $n\in\omega$. Podemos supor sem perda de generalidade que $\{x_{n_k}\}$ converge para algum ponto de $R^\#$, pois em $R^\#$ toda sequ\^encia admite uma subsequ\^encia convergente. Todavia como $x_{n_k}<p$, para todo $k\in\omega$, decorrer\'a que $\lim_kx_{n_k}\le p$, como consequ\^encia $\liminf x_n\le p<\liminf x_n$, o que \'e um absurdo. A outra igualdade demonstra-se usando um argumento sim\'etrico.\qed
  
  \corollary{}{\sl%
    Seja $\{x_n\}\subset R^\#$. Se $\kappa<\liminf x_n$ $(\limsup x_n<\kappa)$, ent\~ao existe $n_\kappa\in\omega$, tal que se $n\ge n_\kappa$, ent\~ao $x_n>\kappa$ $(x_n<\kappa)$.
  }
  
  \proof Basta saber que $\liminf x_n=\sup_n\inf_{k\ge n}x_k>\kappa$, da\'i vem que existe um $n_\kappa\in\omega$ tal que $\inf_{k\ge n_\kappa}x_k>\kappa$, como consequ\~encia $x_n>\kappa$, para todo $n\ge n_\kappa$. O outro caso pode ser demonstrado com um argumento sim\'etrico.\qed
  
  \corollary{}{\sl%
    Se $\{x_n\},\{y_n\}\subset R^\#$, s\~ao tais que existe $m\in\omega$, tal que para todo $n\ge m$, temos $x_n\le y_n$, ent\~ao
    $$
      \liminf x_n\le\liminf y_n\quad\hbox{e}\quad\limsup x_n\le\limsup y_n.
    $$
  }
  
  \proof Suponha que $\liminf y_n<\liminf x_n$ e derive uma contradi\c c\~ao com as hip\'oteses. O mesmo pode ser feito para provar a outra desigualdade.\qed
  
  \theorem{}{\sl%
    Sejam $\{a_n\}$ e $\{b_n\}$ sequ\^encias em $R^\#$. Ent\~ao
    $$
      \vbox{
        \halign{
          \hfill#&#\hfill\cr
          $\liminf a_n+\liminf b_n$ & $\leq\liminf(a_n+b_n)$        \cr
          \omit                     & $\leq\limsup(a_n+b_n)$        \cr
          \omit                     & $\leq\limsup a_n+\limsup b_n.$ \cr
        }
      } 
    $$
  }
  
  \proof Sejam $\alpha_n=\sup A_n$ com $A_n=\{a_k\in R^\#:k\geq n\}$ e $\beta_n=\sup B_n$ com $\{b_k\in R^\#:k\geq n\}$. Provemos que $\{\alpha_n\}$ e $\{\beta_n\}$ s\~ao mon\'otonas decrescentes. Basta provar que uma o \'e, pois a outra segue por um racioc\'inio an\'alogo. Em  verdade, dados $A,B\subseteq R^\#$, toda cota inferior de $B$ tamb\'em o \'e de $A$, consequentemente $\inf B\leq\inf A$. Como $\{A_n\}$ \'e mon\'otona decrescente em $\mathscr{P}(R^\#)$ com a ordem induzida por $\subset$, decorre que 
  $$
    \alpha_{n+1}=\sup A_{n+1}\leq\sup A_{n}=\alpha_{n},
  $$
  para todo $n\in\omega$. Ademais, $\lim_n\alpha_n=\inf\{a_n\in R^\#:n\in\omega\}$, isto se deve ao fato de que h\'a somente dois casos a serem considerados para uma sequ\^encia mon\'otona em $R^\#$, ou $\{\alpha_n\}$ admite uma cota inferior em $R$, ou n\~ao. Dependendo do caso $\{a_n\}$ converge para um elemento de $R$ ou para $-\infty$. 
  
  Agora note que para todo $n\in\omega$,
  $$
    \alpha_n+\beta_n\leq a_n+b_n,
  $$
  da\'i vem 
  $$
    \vbox{
      \halign{
        \hfill#&#\hfill\cr
        $\liminf a_n+\liminf b_n$ & $=\lim_n\alpha_n+\lim_n\beta_n$\cr
        \omit                     & $=\lim_n (\alpha_n+\beta_n)$\cr
        \omit                     & $=\liminf (\alpha_n+\beta_n)$\cr
        \omit                     & $\leq\liminf (a_n+b_n).$\cr
      }
    }
  $$
  Adimiti tacitamente o resultado
  $$
    \lim_n\alpha_n+\lim_n\beta_n=\lim_n(\alpha_n+\beta_n),
  $$
  cuja demonstra\c c\~ao \'e corriqueira para sequ\^encias mon\'otonas em $R^\#$, considerando a adi\c c\~ao estendida de $R$ \`a $R^\#$ com as defini\c c\~oes usuais. A demonstra\c c\~ao da outra desigualdade \'e um argumento sim\'etrico dual.\qed
  
  \definition{}{\sl%
    Seja $X$ um espa\c co topol\'ogico e $A,B\subseteq X$, diremos que $A$ e $B$ s\~ao separados $($ou formam uma separa\c c\~ao$)$ se, e somente se, 
  $$
    \overline{A}\cap B=A\cap\overline{B}=\emptyset.
  $$
  Diremos que um conjunto $Y\subseteq X$, \'e {\bf conexo} quando n\~ao for o caso que $X=A\cup B$, com $A,B\neq\emptyset$ separados. Uma cis\~ao de um conjunto $Y$ \'e uma decomposi\c c\~ao $Y=A\cup B$, tais que $A,B$ s\~ao abertos e disjuntos em $Y$. Conforme veremos, $Y$ \'e conexo se, e somente se, a cis\~ao for trivial, isto \'e, um dos operantes da uni\~ao \'e vazio. 
  }
  
  \theorem{}{\sl%
    Seja $X$ um espa\c co topol\'ogico e $Y\subseteq X$. Ent\~ao $Y$ \'e conexo, se e somente se, toda cis\~ao de $Y$ \'e trivial. Em verdade, uma cis\~ao de $Y$ \'e uma separa\c c\~ao de Y. 
  }
  
  \proof Suponhamos que $Y=A\cup B$, uma cis\~ao de $Y$. Observemos que
  $$
    A\cap B\subset\overline{A}\cap B\;\wedge\;A\cap B\subset A\cap\overline{B}.
  $$
  Isto nos diz que se $A\cap B\neq\emptyset$, a fortiori,
  $$
    \overline{A}\cap B\neq\emptyset\;\vee\; A\cap\overline{B}\neq\emptyset.
  $$
  Dessarte $A$ e $B$ formam uma separa\c c\~ao de $Y$.
  
  Agora suponhamos que $Y=A\cup B$, uma separa\c c\~ao de $Y$, observe que 
  $$
    \overline{A}\cap Y=A\;\wedge\;\overline{B}\cap Y=B,
  $$
  consequentemente, $A$ e $B$ s\~ao fechados em $Y$, tamb\'em inferimos que $A\cap B=\emptyset$. Assim $A=Y\setminus B$ e $B=Y\setminus A$, s\~ao abertos em $Y$. Portanto, $A$ e $B$, formam uma cis\~ao. 
  
  Concluindo, a separa\c c\~ao implica na cis\~ao, e vice-versa e, conforme vimos, na realidade s\~ao a mesma coisa. Concluimos por nega\c c\~ao dos membros de uma equival\^encia o requerido.\qed
  
  
  \theorem{}{\sl%
    Seja $X$ um espa\c co topol\'ogico munido com a topologia ordem. Ent\~ao todo conjunto conexo $Y$ com pelos dois pontos, tem a seguinte propriedade
    $$
      \forall x,z\exists y\Bigl(x,y,z\in Y\wedge (x<z\longrightarrow x<y<z)\Bigr).
    $$
  }
  
  \proof \'E suficiente notar que se existem $x,z\in Y$, tais que $(x,z)\cap Y=\emptyset$, ent\~ao $Y=A\cup B$, em que $A=\{y\in Y:y\leq x\}$ e $B=\{y\in Y:y\geq z\}$, observe que $A,B\neq\emptyset$, pois $x\in A$ e $z\in B$. Ademais, $A$ e $B$ s\~ao ambos fechados e disjuntos em $Y$, consequentemente, $Y$ \'e desconexo. A prova decorre por contrapositiva.\qed
  
  \corollary{}{\sl%
    Seja $f:X\rightarrow Y$, cont\'inua\footnote{Pressup\~oe-se obviamente uma topologia sobre $X$.} e $X$ conexo, com $Y$ munido da topologia ordem, ent\~ao $f$ satisfaz a propriedade do valor intermedi\'ario. Em outros termos, se $f(x)<f(z)$, ent\~ao existe $y\in X$ tal que $f(x)<f(y)<f(z)$. 
  }
  
  \proof Decorre de um fato conhecido, que talvez eu o demonstre posteriormente\footnote{Em verdade, este resultado \'e bastante trivial quando se conhece que imagens inversas de intersec\c c\~oes \'e a intersec\c c\~ao de imagens inversas, e da\'i, basta supor por redu\c c\~ao ao absurdo uma cis\~ao n\~ao trivial da imagem e, como conseguinte deduzir uma contradi\c c\~ao.}: {\it fun\c c\~oes cont\'inuas preservam conexos}.\qed
  
  Um outro fato interessate \'e o seguinte, cuja prova pode ser vista no livro do Munkres.
  
  \theorem{}{\sl%
    Seja $f:X\rightarrow Y$, cont\'inua, $X$ compacto e $Y$ munido da topologia ordem. Ent\~ao existem $a,b$, tais que, se $x\in X$, ent\~ao $f(a)\leq f(x)\leq f(b)$.
  }
  
  \proof O caminho para a prova deste teorema \'e {\it reductio ad absurdum}. Suponha por absurdo que n\~ao existam $a$ e $b$, com tais propriedades, construamos $\mathscr{C}=\{]f(x),f(y)[\;\in2^Y:x,y\in X\}$. Afirmamos que, $\mathscr{C}$ \'e uma cobertura aberta de $f(X)$. Com efeito, seja $f(z)\in Y$, como n\~ao existe um menor elemento em $f(X)$, segue que existe $x\in X$, tal que $f(x)<f(z)$. Analogamente como $f(z)$ n\~ao pode ser o maior elemento de $f(X)$, existe $y\in X$ tal que $f(z)<f(y)$. Portanto, $f(X)\subseteq\bigcup\mathscr{C}$, como afirmamos. Da premisa que $X$ \'e compacto e $f$ cont\'inua, decorre que $f(X)$ \'e compacto, consequentemente existe uma subcobertura finita de $\mathscr{C}$, digamos, $\mathscr{D}=\{]f(x_i),f(y_i)[\;\in2^X:i\in n\}$  com $n\in\omega$. Em virtude, da ordem em $Y$ ser linear existem $i,j\in n$, tais que $f(X)=\bigcup\mathscr{D}=\;]f(x_i),f(y_j)[$, mas isto acarretaria $f(x_i),f(y_j)\notin f(X)$, o que \'e uma contradi\c c\~ao. Esta contradi\c c\~ao foi obtida supondo-se que n\~ao existiam $a$ e $b$ com aquelas propriedades. Desta contradi\c c\~ao a condicional \'e necessariamente verdadeira, pois sua nega\c c\~ao acarreta num absurdo.\qed
  
  \begin{exe}[Fun\c c\~ao que satisfaz o teorema do valor intermedi\'ario, mas n\~ao \'e cont\'inua]
    Considere $f:[0,1]\ra \R$, dada por $f=\iota\chi_{]0,1[}+(1-\iota)\chi_{\{0,1\}}$[\footnote{Aqui, $\chi$ \'e a fun\c c\~ao caracter\'istica e $\iota$ a identidade.}.
  \end{exe}
  \prova \'E suficiente notar que $f([0,1])=[0,1]$. No entanto, $f$ \'e descont\'inua em $0$ e $1$.\qed
  
  Este exemplo, apesar de in\'ocuo, nos diz que nem toda fun\c c\~ao que preserva conexos \'e cont\'inua, o mesmo pode ser inferido a respeito de compactos e perfeitos.
  
  
  \definition{}{\sl%
    Seja $f:X\rightarrow Y$, em que $X$ e $Y$ sejam espa\c cos topol\'ogicos satisfazendo o axioma $\rm{T}_1$ e $a\in X'$, i.e., $a$ \'e um ponto limite $($de acumula\c c\~ao$)$ de X. Diremos que $f(x)$ converge $($tende$)$ a $l\in Y$, quando $x$ converge $($tende$)$ a $a$, quando para toda vizinhan\c ca $V_y$ de $y$ existir uma vizinhan\c ca $V_a$ de $a$ tal que
    $$
      x\in V_a\setminus\{a\}\longrightarrow f(x)\in V_y.
    $$
    Quando dispomos de bases ${\cal B}_X$ e ${\cal B}_Y$, das respectivas topologias sobre $X$ e $Y$, podemos verificar que a defini\c c\~ao \'e equivalente a
    $$
      \forall V\exists U\Bigr( (U,V)\in{\cal B}_X\times{\cal B}_Y\wedge(a,l)\in U\times V\wedge(x\in U\setminus\{a\}\longrightarrow f(x)\in V)\Bigr).
    $$
    Sob estas condi\c c\~oes estipulamos a simbologia
    $$
      \lim_af:=l\quad.
    $$
    Esta nota\c c\~ao \'e significativa, pois estamos considerando espa\c cos ${\rm T}_1$. Chamaremos o s\'imbolo $\lim_af$ de limite de $f$ quando o par\^amentro da fun\c c\~ao tende a $a$, \'e comum escrever tal s\'imbolo por $\lim_{x\to a}f(x)$, que conforme notado por M. Spivak em seu livro {\it Calculus}, \'e mais conveniente do ponto de vista pragm\'atico, por exemplo quando se d\'a uma lei para $f$, e.g. $f(x)=e^{-x}/x^n$, escreve-se
    $$
      \lim_\infty f=\lim_{x\to\infty}{e^{-x}\over x^n}.
    $$
  }
  
  \begin{obs}[Teorema 3.54 p. 77 do \pma]\rm%
    Foi afirmado que, as somas parciais da soma constru\'ida n\~ao podem ter limites subsequenciais menores que $\alpha$ e maiores que $\beta$. Antes de prosseguirmos, h\'a um fato que deve ser observado, embora quase evidente, quando n\~ao mencionado conjectura-se, se as sequ\^encias $\{P_n\}$ e $\{Q_n\}$, podem ser constru\'idas. De fato, devem necessariamente existir uma infinidade de termos tanto estritamente positivos como negativos na s\'erie $\sum a_n$. Se s\'o existissem uma quantidade finita de termos negativos, ela convergiria absolutamente e a constru\c c\~ao n\~ao teria significado. Em verdade, conforme se v\^e no teorema posterior, qualquer reordena\c c\~ao converge para mesma soma. O mesmo fato ocorreria se a s\'erie possu\'isse uma quantidade finita de termos positivos, em virtude de ser convergente, seria absolutamente convergente, recaindo no mesmo caso. Logo, \'e perempt\'orio que os conjuntos de termos positivos e o de negativos sejam ambos infinitos. 
    
      Seja $\{r_n\}$ uma subsequ\^encia da sequ\^encia das somas parciais $\{s_n\}$ da s\'erie constru\'ida no teorema. Ent\~ao, para todo $n\in\omega$, $A_n=\{p\in\omega:y_p\le r_n\le x_p\}\neq\emptyset$. Com efeito, o \'ultimo termo de $r_n$ \'e da forma $-Q_k$ ou $P_m$, isto, \'e evidente da pr\'opria constru\c c\~ao da soma, pois seus termos s\~ao de tais formas. Evidentemente, como o n\'umero de termos de $r_n$ cresce, necessariamente $k$ e $m$ crescem. Observe que $\{k_n\}$ \'e uma sequ\^encia estritamente crescente, o mesmo ocorre com $\{m_n\}$, para facilitar meu argumento, suponha que $m_0=k_0=0$. Suponhamos, que o \'ultimo termo de $r_n$ seja $-Q_{k}$, certamente $k\in(k_{i_n-1},k_{i_n}]\cap\omega$ para algum $i_n\in\omega$, pois
    $$
      \omega\setminus\{0\}=\omega\cap\bigcup_{n\in\omega}(k_n,k_{n+1}],
    $$
    Da\'i,vem $y_{i_n}\leq r_n\leq x_{i_n}$. Um argumento an\'alogo pode ser aplicado ao caso que o \'ultimo termo de $r_n$ \'e da forma $P_m$. Assim, conclu\'imos que $A_n\neq\emptyset$ para todo $n\in\omega$. 
    
    
    Seguidamente, seja $p_n=\max A_n$, a sequ\^encia $\{x_{p_n}\}$, potencialmente possui termos iguais. Ademais, $\{p_n\}$ \'e mon\'otona n\~ao decrescente com imagem infinita, pois os \'indices dos \'ultimos temos de $r_n$ crescem com $n$. Dessarte, \'e evidente que $\lim_nx_{p_n}=\beta$ e $\lim_ny_{p_n}=\alpha$. Em virtude do fato $y_{p_n}\leq r_n\leq x_{p_n}$, para todo $n\in\omega$, decorre que
    $$
      \alpha=\liminf_n y_{p_n}\leq\liminf_n r_n\leq\limsup_n r_n\leq\limsup_n x_{p_n}=\beta.
    $$
    \qed
  \end{obs}
  
  \begin{exe}[Item (d) do exerc\'icio 11 do \pma]
    H\'a uma pergunta que me deu bastante trabalho. Supondo que $\{a_n\}$ seja uma sequ\^encia de n\'umeros reais, em $]0,\infty[$, tal que $\sum a_n=\infty$, o que se pode dizer sobre
  
    \begin{equation}
      \sum{a_n\over 1+na_n}?
      \label{query_sum}
    \end{equation}
    Isto \'e, converge ou diverge?
  
  \end{exe}
  
  N\~ao obstante, h\'a uma soma ao lado desta, cuja converg\^encia \'e trivial pelo crit\'erio da compara\c c\~ao, a saber
  $$
    \sum{a_n\over 1+n^2a_n},
  $$
  bastando para isto comparar com a s\'erie 
  $$
    \sum{1\over n^2}.
  $$
  
  Para responder \`a esta pergunta eu percorri o caminho mais dif\'icil. Primeiro provei o seguinte:
    \begin{pro}
      Sejam $\{a_n\}$ uma sequ\^encia em $]0,\infty[$ tal que $\sum a_n=\infty$ e $n_0\in\omega$, tal que 
      \[
        \forall n\Bigl(n\geq n_0\lra a_n\in\;]0,1[\Bigr)
        \label{pri}
      \]
      ou
      \[
        \forall n\Bigl(n\geq n_0\lra a_n\in\;[1,\infty[\Bigr).
        \label{seg}
      \]
      Ent\~ao 
      $$
        \sum{a_n\over 1+n a_n}=\infty.
      $$
    \end{pro}
    \prova Suponhemos que valha (\ref{pri}) ent\~ao temos
    \[
      \Bigl({a_n\over1+a_n}\Bigr)^n={a_n^n\over(1+a_n)^n}<{a_n\over1+na_n},
    \]
    donde segue-se que
    \[
      \sqrt[n]{a_n\over 1+a_n}<\sqrt[n^2]{a_n\over1+na_n}<{a_n\over1+na_n}.
      \label{cadeia}
    \]
    Todavia, provamos no item (a) que
    \[
      \sum{a_n\over 1+a_n}=\infty,
    \]
    consequentemente
    \[
      1\leq\limsup\sqrt[n]{a_n\over1+a_n}
      \label{ls}
    \]
    de (\ref{cadeia}) e de (\ref{ls}) concluimos que
    \[
      1\leq\limsup\sqrt[n]{a_n\over1+a_n}\leq\limsup{a_n\over1+na_n},
    \]
    por conseguinte
    \[
      \sum{a_n\over1+na_n}=\infty,
    \]
    pois \'e uma soma de termos positivos, se n\~ao converge tende a $\infty$.
  
    Agora consideremos o segundo caso (\ref{seg}), basta notar que
    \[
      {1\over1+n}={a_n\over a_n+na_n}\leq{a_n\over1+na_n}
    \]
    para todo $n\geq n_0$. Em conformidade com o crit\'erio de compara\c c\~ao, conclu\'imos
    \[
      \sum{a_n\over1+na_n}=\infty.
    \]
    \qed
    
    Isto me fez atinar que, se queremos que (\ref{query_sum}) convirga ent\~ao $\{a_n\}$ tem infinitos termos, tanto em $]0,1[$, quanto em $[1,\infty[$. Da\'i eu fiz meu chute
    \begin{pro}
      Sejam
      \[
        A=\{n:\exists m(m\in\omega\wedge n=2^m)\}
      \]
      e $\{a_n\}$ definida por
      \[
        a_n=
        \begin{cases}
          \; n,      & n\in A     \cr
          \; 2^{-n}, & n\not\in A. \cr
        \end{cases}
      \]
      Ent\~ao 
      \[
        \sum{a_n}=\infty\quad
      \]
      e
      \[
        \sum{a_n\over1+na_n}<\infty.
      \]
    \end{pro}
    \prova Observe primeiramente que para todo $n\in\omega$, vale
    \[
      \sum_{k=1}^{2^n}a_k\geq2^n,
    \]
    consequentemente
    \[
      \sum a_n=\infty.
    \]
  
    O pr\'oximo passo \'e averiguar por indu\c c\~ao matem\'atica que para todo $n\in\omega$, vale 
    \[
      \sum_{k=1}^{2^n}{a_k\over1+ka_k}\leq\sum_{k=0}^n2^{-k}+\sum_{k=1}^{2^n}2^{-k}.
    \]
    e concluir
    \[
      \sum{a_n\over1+na_n}<\infty.
    \]
    Doravante para todo $n\in\omega$ fa\c camos
    \[
      b_n={a_n\over1+na_n}.
    \]
    Primeiro, 
    \[
      \sum_{k=1}^{2^0}b_k=1\leq1+{1\over2}=\sum_{k=0}^02^{-k}+\sum_{k=1}^{2^0}2^{-k}.
    \]
    Portanto, a propriedade \'e v\'alida para $n=0$. Suponhamos por hip\'otese que ela seja para $n$, observemos em seguida
    \[
      \begin{split}
        \sum_{k=1}^{2^{n+1}}{b_k} & =    \sum_{k=1}^{2^n}b_k+\sum_{k=2^n+1}^{2^{n+1}}b_k\cr
                                  & \leq \Bigl(\sum_{k=0}^n2^{-k}+\sum_{k=1}^{2^n}2^{-k}\Bigr)+2^{-(n+1)}+\sum_{k=2^n+1}^{2^{n+1}}2^{-k}\cr
                                  & = \sum_{k=0}^{n+1}2^{-k}+\sum_{k=1}^{2^{n+1}}2^{-k}.\cr
      \end{split}
    \]
    Conclu\'imos que \'e v\'alida para $n+1$. Por indu\c c\~ao matem\'atica provamos o requerido.\qed
  
  
  \begin{exe}[Exerc\'icio 16 do \pma]
    Considere $\alpha\in\;]0,\infty[$ e a sequ\^encia $x:\omega\lra R$ tal que $x_0>\sqrt{\alpha}$ e
    \begin{equation}
      x_{n+1}=\fr{1}{2}\Bigl(x_n+\fr{\alpha}{x_n}\Bigr),
    \end{equation}
    para todo $n\in\omega$. Primeiro, observemos que
    \begin{equation}
      \forall y\biggl(y>\sqrt{\alpha}\lra\sqrt{\alpha}<\fr{1}{2}\Bigl(y+\fr{\alpha}{y}\Bigr)<y\biggr).
    \end{equation}
    Segue-se de manipula\c c\~oes das duas desigualdades, uma \'e trivialmente verdadeira,
    \begin{equation}
      (y-\sqrt{\alpha})^2>0\llra{1\over 2}\Bigl(y+{\alpha\over y}\Bigr)>\sqrt{\alpha}.
    \end{equation}
    Desta forma, a sequ\^encia \'e trivialmente estritamente decrescente. Ademais, ela \'e limitada inferiormente por $0$, como consequ\^encia \'e convergente. Certamente que pela pr\'opria defini\c c\~ao $x_n\in\;]0,\infty[$, para todo $n$. Se $\lim_nx_n=0$, ter\'iamos
    $$
      \alpha=\lim_nx_n^2+\alpha=\lim_n(2x_{n+1}x_n)=0
    $$
    uma contradi\c c\~ao com a suposi\c c\~ao $\alpha>0$. Destarte, $l=\lim_nx_n>0$, da\'i obtemos
    $$
      l=\lim_nx_{n+1}=\lim_n{x_n^2+\alpha\over 2x_n}={l^2+\alpha\over 2l}
    $$
    donde segue-se que $l=\sqrt{\alpha}$.\qed
  \end{exe}
  
  \begin{Def}
    Sejam $\sum a_n$ uma s\'erie convergente de termos n\~ao negativos, e $A\subset\omega$. Seja $\kappa_A$ a caracter\'istica de $A$. Ent\~ao definimos
    \[
      \Sigma A=\sum_{i=0}^\infty\kappa_A(i)a_i.
    \]
    Evidentemente $\Sigma A$ \'e finito pois $\sum a_n<\infty$.
    \label{SomaEmConjunto}
  \end{Def}
  
  %\begin{lem}
  %  Sejam $\sum a_n$ uma s\'erie convergente de termos n\~ao negativos e $k\in\omega\setminus\{0\}$. Ent\~ao
  %  \[
  %    \sum_{i=0}^\infty\biggl(\sum_{j=ki}^{k(i+1)-1}a_j\biggr)=\sum_{i=0}^\infty a_i.
  %  \]
  %\end{lem}
  %\prova \'E suficiente observar que para todo $n\in\omega$, vale
  %\[
  %  \sum_{i=0}^n\biggl(\sum_{j=ki}^{k(i+1)-1}a_j\biggr)=\sum_{i=0}^{k(n+1)-1}a_i.
  %\]
  %\qed
  
  \begin{lem}
    Sejam $\sum a_n$ como na \rf{Defini\c c\~ao}{SomaEmConjunto} e $A,B\in 2^\omega$. Se $A\subset B$, ent\~ao $\Sigma A\leq\Sigma B$. 
    \label{PropSomaEmConjunto}
  \end{lem}
  \prova \'E suficiente notar que para todo $i\in\omega$ vale
  \[
    \kappa_{A}(i)a_i\leq\kappa_{B}a_i,
  \]
  a desigualdade requerida segue da compara\c c\~ao de s\'eries.\qed
  
  
  \begin{teo}[4.31 Remark: {\it Principles of Mathematical Analysis}]
    Seja $E\subset\;]a,b[$ enumer\'avel, virtualmente denso. Sejam $\sigma\in E^\omega$ uma bije\c c\~ao, $\sum a_n$ uma s\'erie convergente de termos estritamente positivos e $x\in\;]a,b[$, definamos
    \[
      [x]\; =\sigma^{-1}(]a,x[)
    \]
    e 
    \[
      f=\bigl\{z:\exists x\bigl(x\in\;]a,b[\;\a z=(x,\Sigma[x])\bigr)\bigr\}.
    \]
    Ent\~ao $f\in\R^{]a,b[}$ e tem as seguintes propriedades:
    \begin{enumerate}[label = (\alph*)]
      \item{%
        $f$ \'e monotonicamente crescente em $]a,b[$;
      }
      \item{%
        $f$ \'e descont\'inua em todo ponto de $E$; Em verdade,
        \[
          f(\sigma_n+)-f(\sigma_n-)=a_n;
        \]
      }
      \item{%
        $f$ \'e cont\'inua em $]a,b[\,\cp\, E$.
      }
    \end{enumerate}
    \label{teo271220251550}
  \end{teo}
  \prova Primeiramente, suponha que $a<x<y<b$. Evidentemente $[x]\subset [y]$, por conseguinte, do \rf{Lema}{PropSomaEmConjunto} $f(x)\leq f(y)$, segue-se, portanto, o item (a).
  
    Seguidamente, provemos que para todo $x\in\;]a,b[$, tem-se
    \[
      f(x)=\sup\bigl\{f(t):t\in\;]a,x[\bigr\}=f(x-).
      \label{eq2412251821}
    \]
    Para tanto seja $\varepsilon>0$. De $\sum\kappa_{[x]}(i)a_i<\infty$, necessariamente existe $i_\varepsilon\in\omega$, tal que
    \[
      0\leq\sum_{i=0}^{i_\varepsilon-1+n}\kappa_{[x]}(i)a_i-\sum_{i=0}^{i_\varepsilon-1}\kappa_{[x]}(i)a_i<\varepsilon.
      \label{eq2412251816}
    \]
    para todo $n\in\omega$. Podemos {\it a fortiori} escolher $t\in\;]a,x[$, tal que $i_\varepsilon\cap [x]\subset [t]$. Conformemente $f(t)\leq f(x)$ e 
    \[
      \sum_{i=0}^{i_\varepsilon-1}\kappa_{[x]}(i)a_i\leq\sum_{i=0}^\infty\kappa_{[t]}(i)a_i=f(t).
      \label{eq2412251817}
    \]
    Tomando o limite segundo $n$ em (\ref{eq2412251816}) e comparando com (\ref{eq2412251817}) obtemos
    \[
      0\leq f(x)-f(x-)\leq f(x)-f(t)\leq\varepsilon,
    \]
    como $\varepsilon>0$ \'e arbitr\'ario inferimos (\ref{eq2412251821}).
  
  
    Em seguida observemos que 
    \[
      I_x\neq\emptyset\;\;\hbox{se, e somente se,}\;\; x\in{\mathfrak I}\sigma.
    \]
    Com efeito, como $I_x\subset\mathfrak{I}\sigma$, se $I_x\neq\emptyset$, existe $n\in\omega$, tal que $\sigma_n\in I_x$. Da\'i segue que $\sigma_n\in[t]$, para todo $t\in\;]x,b[$, logo, $\sigma_n\in\;]x,t[$, para todo $t\in\;]x,b[$, consequentemente $\sigma_n=x$, i.e., $x\in{\mathfrak I}\sigma$. Reciprocamente, se existir $n\in\omega$ tal que $x=\sigma_n$, \'e evidente que $\sigma_n=x\in[t]$, para todo $t\in\;]x,b[$, consequentemente $\sigma_n\in I_x$, i.e., $I_x\neq\emptyset$.
  
    Sejam $x\notin{\mathfrak I}\sigma=E$, $y\in\;]x,b[$ e $\varepsilon>0$. Como $\sum\kappa_{[y]}(i)a_i<\infty$, necessariamente existe $i_\varepsilon$, tal que 
    \[
      \sum_{i=i_\varepsilon}^{\infty}\kappa_{[y]}(i)a_i<\varepsilon.
    \]
    Podemos escolher $t\in\;]x,y[$, tal que $i_\varepsilon^+\subset\cp [t]$. De fato, basta tomar $t<\min\{\sigma_i:i\in i_\varepsilon^+\}$. Ademais,
    \[
      0\leq f(x+)-f(x)\leq f(t)-f(x)\leq\sum_{i=i_\varepsilon}^\infty\kappa_{[t]}(i)a_i\leq\sum_{i=i_\varepsilon}^\infty\kappa_{[y]}(i)a_i<\varepsilon,
    \]
    como $\varepsilon>0$ \'e arbitr\'ario segue-se que $f(x+)=f(x)=f(-x)$, consequentemente $f$ \'e cont\'inua em $x$. Em virtude de $x\in\;]a,b[\,\cp\, E$ ser arbitr\'ario, provamos (c).
  
    Agora sejam $n\in\omega$ e $y\in\;]\sigma_n,b[$. Analogamente, dado $\varepsilon>0$, escolhamos $i_\varepsilon>n$, tal que
    \[
      \sum_{i=i_\varepsilon}^\infty\kappa_{[y]}(i)a_i<\varepsilon.
    \]
    Tomando $t\in\;]\sigma_n,y[$, tal que $i_\varepsilon^+\subset\cp [t]$; temos em conformidade,
    \[
      0\leq f(\sigma_n+)-f(\sigma_n)-a_n\leq f(t)-f(\sigma_n)-a_n\leq\sum_{i=i_\varepsilon}^\infty\kappa_{[t]}(i)a_i\leq\sum_{i=i_\varepsilon}^\infty\kappa_{[y]}(i)a_i<\varepsilon.
    \]
    Da arbitrariedade de $\varepsilon>0$, conclu\'imos que 
    \[
      f(\sigma_n+)-f(\sigma_n)=f(\sigma_n+)-f(\sigma_n-)=a_n,
    \]
    provando portanto o item (b).\qed

  \theorem{\ (Teorema do ponto fixo)\footnote{\`As vezes chamado de teorema do ponto fixo de Banach. Basta que o espa\c co seja m\'etrico e completo, i.e., toda sequ\^encia de Cauchy \'e convergente.}}{\sl%
    Sejam $(X,\varrho)$ um espa\c co m\'etrico completo, $\varkappa\in R_+$, tal que $\varkappa<1$ e $f:X\rightarrow X$ uma fun\c c\~ao tal que\footnote{Uma fun\c c\~ao com tal propriedade \'e chamada de contra\c c\~ao, pois contrai segundo a m\'etrica sua imagem com rela\c c\~ao ao seu dom\'inio.} 
    $$
      \forall x,y\Bigl(x,y\in X\lra \varrho(f(x),f(y))\leq\varkappa\varrho(x,y)\Bigr).
    $$
    Ent\~ao $f$ admite um \'unico ponto fixo, i.e., existe $p\in X$, tal que $f(p)=p$.
  }
  
  \proof A prova \'e cl\'assica. Primeiro, provemos a unicidade. Com efeito, sejam $p,q$, tais que $f(p)=p$ e $f(q)=q$, temos em conformidade 
  $$
    \varrho(p,q)=\varrho(f(p),f(q))\leq\varkappa\varrho(p,q),
  $$
  da\'i necessariamente $\varrho(p,q)=0$, pois caso contr\'ario, de acordo com a desigualdade anterior incorrer\'iamos numa contradi\c c\~ao, a saber, $\varkappa\geq 1$. Segundo, escolhamos $x\in X$ e definamos a sequ\^encia $x_0=x$ e $x_{n+1}=f(x_n)$, para todo $n\in\omega$. Em seguida, provemos que
  $$
    \forall n\Bigl(n\in\omega\lra\varrho(x_{n+2},x_{n+1})\leq\varkappa^{n+1}\varrho(x_1,x_0)\Bigr).
  $$
  A prova \'e feita por indu\c c\~ao matem\'atica sobre $n\in\omega$. Seja $n=0$, conspicuamente
  $$
    \varrho(x_2,x_1)\leq\varkappa\varrho(x_1,x_0).
  $$
  Seguidamente, suponhamos por hip\'otese de indu\c c\~ao que a tese seja v\'alida para $n\in\omega$, conformemente
  $$
    \varrho(x_{n+3},x_{n+2})\leq\varkappa\varrho(x_{n+2},x_{n+1})\leq\varkappa^{n+2}\varrho(x_1,x_0),
  $$
  o que acarreta que a tese \'e v\'alida para $n+1$. Do princ\'ipio de indu\c c\~ao matem\'atica, a condicional quantificada \'e v\'alida, i.e., a tese (ou consequente) \'e verdadeira para todo $n\in\omega$. Sejam $m,n\in\omega$ tais que $m\leq n$, segue-se em conformidade
  $$
    \varrho(x_n,x_m)\leq\varrho(x_1,x_0)\sum_{i=m}^n\varkappa^i,
  $$
  Como $\sum_n\varkappa^n$ \'e convergente, seque que a sequ\^encia das suas somas parciais \'e Cauchy e, a desigualdade anterior implica que $\{x_n\}$ \'e uma sequ\^encia de Cauchy em $X$. Como $X$ \'e completo $\{x_n\}$ converge para um \'unico ponto em $X$, digamos $\lim_nx_n=p$. Verifica-se da propriedade de $f$ que ela \'e cont\'inua,  notavelmente
  $$
    f(p)=\lim_nf(x_n)=\lim_nx_n=p.
  $$
  \qed
  
  \begin{teo}[Um problema do livro Problems for Mathematicians, Young and Old de Paul R. Halmos]\sl%
    Seja $x\in R$, consideremos a sequ\^encia $\{x_n\}$ com $x_n=\cos^{(n)}(x)$, em que $\cos^{(n)}$ \'e a $n$-\'esima composi\c c\~ao da fun\c c\~ao trigonom\'etrica $\cos$. Ent\~ao, $\{x_n\}$ converge. 
  \end{teo}
  
  \prova Podemos supor sem perda de generalidade que $x\in[0,1]$, pois se $x\in R$, ent\~ao $\cos(x)\in[-1,1]\subseteq[-\pi/2,\pi/2]$, consequentemente $cos^{(2)}(x)\in[0,1]$. Consideremos $\cos:[0,1]\rightarrow [0,1]$, sabemos que $\cos$ \'e deriv\'avel em $R$, e que $\cos'=-\sin$, pelo teorema do valor m\'edio dados quaisquer $x,y\in[0,1]$ com $x<y$, existe $z\in(x,y)$, tal que
  $$
    \cos(x)-\cos(y)=\sin(z)(y-x)
  $$
  donde
  $$
    |\cos(x)-\cos(y)|=|\sin(z)||x-y|.
  $$
  Todavia, $\sin:[0,1]\rightarrow R$ \'e limitada, de tal maneira que atinge o m\'aximo em $\sin(1)<\sin(\pi/2)=1$, pois $\sin$ \'e crescente em $[0,\pi/2]\supset[0,1]$.
  
  Dessarte, existe $\varkappa=\sin(1)<1$, tal que 
  $$
    |\cos(x)-\cos(y)|\leq\varkappa|x-y|.
  $$
  Logo, $\cos|_{[0,1]}$ \'e uma contra\c c\~ao. Como $[0,1]$ \'e completo pois todo compacto de um espa\c co m\'etrico \'e completo, existe $\theta\in[0,1]$, tal que $\cos(\theta)=\theta$. Ademais, vimos que $\lim_nx_n=\theta$[\footnote{\'E not\'avel que $\theta\in(0,1)$.}.\qed 
  
  \theorem{ (Adaptado do exerc\'icio 23 do {\it Calculus} de M. Spivak)}{\sl% 
    Seja $f:X\rightarrow R$ $(X\subseteq R)$ uma fun\c c\~ao e $a\in X'$ com a seguinte propriedade: Para toda fun\c c\~ao $g:X\rightarrow R$ se $\lim_ag$ n\~ao existe, ent\~ao $\lim_afg$ n\~ao existe, se, e somente se, $\lim_af$ existe e $\lim_af\neq 0$ ou $\lim_a|f|=\infty$. Em s\'imbolos \'e equivalente a
    $$
      \forall g\Bigl(g\in R^X\wedge\exists\lim_afg\longrightarrow\exists\lim_ag\Bigr)\longleftrightarrow(\exists\lim_af\wedge\lim_af\neq0)\;\vee\;\lim_a|f|=\infty.
    $$
  }
  
  \proof Suponhamos primeiramente que $\lim_af\neq 0$, evidentemente, se existe $\lim_afg$, ent\~ao as propriedades de limites implicam
  $$
    \lim_ag=\lim_ag{f\over f}={\lim_a fg\over\lim_af},
  $$
  ou seja, $\lim_ag$ existe.  Note que o quociente $f/f$ \'e poss\'ivel, pois o limite \'e valorizado localmente e, em virtude da hip\'otese $\lim_af\neq 0$, em uma vizinhan\c ca de $a$, tem-se $f(x)\neq 0$.
  
  Suponhamos agora que $\lim_a|f|=\infty$, e que $\lim_afg=l$, temos por maior raz\~ao que $\lim_a|l/f|=\lim_a|1/f|=0$. Assim, dado $\varepsilon>0$ existe $\delta>0$ tal que
  
  $$
    x\in X\cap(a-\delta,a+\delta)\longrightarrow\biggl|{l\over f(x)}\biggr|<{\varepsilon\over 2}\;\wedge\;\biggl|g(x)-{l\over f(x)}\biggr|<{\varepsilon\over2}
  $$
  consequentemente
  $$
    x\in X\cap(a-\delta,a+\delta)\longrightarrow |g(x)|=\biggl|g(x)-{l\over f(x)}+{l\over f(x)}\biggr|\leq\biggl|g(x)-{l\over f(x)}\biggr|+\biggl|{l\over f(x)}\biggr|<\varepsilon,
  $$
  i.e., $\lim_ag=0$. Portanto $\lim_ag$ existe.
  
  Agora suponhamos
  $$
    \neg\Bigl((\exists\lim_af\wedge\lim_af\neq0)\;\vee\;\lim_a|f|=\infty\Bigr) 
  $$
  ou equivalentemente
  $$
    \Bigl(\neg(\exists\lim_af)\wedge\lim_a|f|\neq\infty\Bigr)\vee\lim_af=0.
  $$
  Se $\lim_af=0$, tome $g=\chi_{(-\infty,a)}-\chi_{(a,\infty)}$, observe que n\~ao existe $\lim_ag$, mas $\lim_afg=0$. Se n\~ao existe $\lim_af$ e $\lim_a|f|\neq\infty$, temos alguns casos a considerar. Se existe $\delta>0$ tal que se $x\in X\cap(a-\delta,a+\delta)$, ent\~ao $f(x)\neq 0$, tome $g:X\cap(a-\delta,a+\delta)\rightarrow R$ dada por $g(x)=1/f(x)$. Observemos que n\~ao \'e o caso que $\lim_ag=0$, pois isto acarretaria $\lim_a|f|=\infty$. Logo n\~ao existe $\lim_ag$, pois caso contr\'ario existiria $\lim_af$. A outra possibilidade \'e 
  $$
    \forall\delta\exists x\Bigl(\delta>0\;\wedge\;x\in X\cap(a-\delta,a+\delta)\cap f^{-1}(\{0\})\Bigr).
  $$
  
  Neste caso, defina $g:X\rightarrow R$, por
  
  $$
    g(x)=
    \left\{
    \begin{matrix}
      \hfill 1,                               & x\in    f^{-1}(\{0\}); \hfill\cr
      \hfill {\displaystyle {x-a\over f(x)}}, & x\notin f^{-1}(\{0\}). \hfill\cr
    \end{matrix}
    \right.
  $$
  observe que $\lim_ag|_{f^{-1}(\{0\})}=0$ mas $1\in g(X\cap(a-\delta,a+\delta))$, para todo $\delta>0$, consequentemente n\~ao existe $\lim_ag$. N\~ao obstante, $|fg(x)|\leq|x-a|$, para todo $x\in X$, logo $\lim_afg=0$. Estes argumentos resultantes da nega\c c\~ao da hip\'otese, possibilitaram a constru\c c\~ao de uma $g$ adequada para cada caso, negando, portanto, o quantificador universal. Isto por sua vez nos leva a nega\c c\~ao da consequente. Pela contrapositiva fica provada a outra condicional, concluindo, portanto, o teorema. \qed
  
  \chapter{Continuidade uniforme}
  
  A seguir, apresento um resultado sem recorrer a resultados muito sofisticados da an\'alise, em outros termos, uma prova usando conceitos elementares do c\'alculo.
  \begin{pro}[Exerc\'icio 3 p. 146 do {\it Calculus} de [Spivak{]}]
    Se $f$ \'e cont\'inua em $[a,b]$, ent\~ao $f$ \'e uniformemente cont\'inua em $[a,b]$.
  \end{pro}
  \prova Suponhamos por {\it reductio ad absurdum} que n\~ao \'e o caso de $f$ ser uniformemente cont\'inua em $[a,b]$. Existe, portanto $\varepsilon>0$, tal que para todo $\delta>0$ existem $x,y\in [a,b]$, tais que
  \[
    |x-y|<\delta\lra|f(x)-f(y)|\geq\varepsilon.
  \]
  Como consequ\^encia dado $\delta>0$, concebamos o conjunto 
  \[
    X_\delta=\bigl\{(x,y):x\in[a,b]\a \exists y(y\in[a,b]\a|x-y|<\delta\a|f(x)-f(y)|\geq\varepsilon)\bigr\},
  \]
  pela hip\'otese, $X_\delta$ \'e infinito.
  
  Em seguida, bissectamos $[a,b]$. Como $f$ \'e cont\'inua em $(a+b)/2$ existe $\delta_0>0$, tal que
  \[
    x\in[a,b]\a\Big|x-{a+b\over 2}\Big|<{\delta_0\over 2}\lra\Big|f(x)-f\Bigl({a+b\over 2}\Bigr)\Big|<{\varepsilon\over2}.
  \]
  Portanto, 
  \[
    X_{\delta_0}\subset\Bigl[a,{a+b\over 2}\Bigr]\o X_{\delta_0}\subset\Bigl[{a+b\over 2},b\Bigr].
  \]
  Seja $K_0$ uma das bisse\c c\~oes de $[a,b]$ que contem $X_{\delta_0}$. Suponhamos constru\'ida a fam\'ilia $\{K_i:i\in n+1\}$, tal que para todo $i\in n+1$, existe $\delta_i>0$, tal que $X_{\delta_i}\subset K_i$. Seja $K_n=[a_n,b_n]$. Bissectamos $K_n$, e de maneira an\'aloga determinamos $\delta_{n+1}>0$, tal que
  \[
    x\in K_n\a\Big|x-{a_n+b_n\over 2}\Big|<{\delta_{n+1}\over 2}\lra \Big|f(x)-f\Bigl({a_n+b_n\over 2}\Bigr)\Big|<{\varepsilon\over 2}.
  \]
  Da mesma forma, seja $K_{n+1}$ uma das bisse\c c\~oes de $K_n$, tal que $X_{\delta_{n+1}}\subset K_{n+1}$. Destarte, constru\'imos por recurs\~ao matem\'atica uma fam\'ilia enumer\'avel $\{K_n:n\in\omega\}$, ou melhor, uma sequ\^encia $\{K_n\}$, tal que $K_{n+1}\subset K_n$ e $X_{\delta_n}\subset K_n$, para todo $n\in\omega$. Pelo teorema dos intervalos encaixados ({\it Nested Intevals Theorem}) existe $x\in\bigcap_{n\in\omega}K_n$. Da constru\c c\~ao de $\{K_n\}$, temos que 
  \[
    x\in K_n=[a_n,b_n]\a b_n-a_n<2^{-(n+1)}(b-a)
  \]
  e existem sequ\^encias $\{x_n\}$ e $\{y_n\}$ com 
  \[
    x_n,y_n\in K_n\a |f(x_n)-f(y_n)|\geq\varepsilon.
  \]
  Da\'i e da continuidade de $f$ em $[a,b]$ vem
  \[
    0<\varepsilon\leq\lim_n|f(x)-f(x_n)|+\lim_n|f(x)-f(y_n)|=0,
  \]
  o que \'e evidentemente uma contradi\c c\~ao. Isto por sua vez acarreta que a hip\'otese \'e falsa. Consequentemente, $f$ \'e uniformemente cont\'inua em $[a,b]$.
  
  \qed
  
  \begin{exe}
    Sejam $n\in\omega$ e $f,g:[0,\infty[\;\ra R$, dadas por $f(x)=x^{n+2}$ e $g(x)=x^{1/(n+1)}$. Ent\~ao $f$ n\~ao \'e uniformemente cont\'inua e $g$ o \'e.
  \end{exe}
  \prova\qed
  
  \begin{exe}
    Seja $f:\;]0,1]\ra R$ dada por ${\rm sen}(1/x)$ \'e limitada, mas n\~ao \'e uniformemente cont\'inua.
  \end{exe}
  \prova\qed
  
  \begin{exe}
    Seja $f:[0,\infty[\;\ra R$ dada por ${\rm sen}(x^2)$ \'e limitada, mas n\~ao \'e uniformemente cont\'inua.
  \end{exe}
  \prova\qed
  
  \begin{exe}
    Seja $f:[0,\infty[\;\ra R$ cont\'inua e peri\'odica. Ent\~ao $f$ \'e uniformemente cont\'inua.
  \end{exe}
  
  \chapter{Derivadas e integrais}
  
  \begin{lem}
    Seja $f:[a,b]\ra\R$ tal que $f(t)\neq0$, para todo $t\in[a,b]$. Ent\~ao
    \[
      (\ln|f|)'={\sigma(f)f'\over |f|}={f'\over f}.
    \]
  \end{lem}
  \prova Primeiro relembremos da fun\c c\~ao sinal, a saber,
  \[
    \sigma(x)=(\chi_{\R^*_+}-\chi_{\R^*_-})(x)
  \]
  em que $\chi$ \'e a fun\c c\~ao caracter\'\i stica. \'E sabido que $|\,|'=\sigma$ em ${\R^*}$ e n\~ao \'e dif\'\i cil ver que para todo $x\in \R$
  \[
    |x|=\sigma(x)\cdot x.
  \]
  Da regra da cadeia temos,
  \[
    (\ln|f|)'={\sigma(f)f'\over |f|}={f'\over f}.
  \]
  \qed
  
  \begin{Def}
    Denotaremos o conjunto das fun\c c\~oes reais Riemann integr\'aveis em $[a,b]$, por $\mathscr{R}(\R^{[a,b]})$, ou quando n\~ao houver risco de embiguidade denotaremos simplesmente por $\mathscr{R}$. Ademais quando $f\in\mathscr{R}$ definiremos $I_f\in \R^{[a,b]}$ por
    \[
      I_f(x)=\int_a^xf\,d\iota,
    \]
    em que $\iota$ \'e a fun\c c\~ao identidade\footnote{Estou considerando a integral de Riemann-Stieltjes, em que a fun\c c\~ao mon\'otona em quest\~ao \'e a fun\c c\~ao identidade $\iota$.}.
    
  \end{Def}
  \begin{teo}
    Seja $f\in\mathscr{R}(\R^{[a,b]})$[\footnote{Conjunto das fun\c c\~oes Riemann integr\'aveis de $\R^{[a,b]}$.}. Se $f$ \'e cont\'inua em $c\in[a,b]$, ent\~ao $I_f'(c)=f(c)$. {\it A fortiori}, se $f\in\mathscr{C}(\R^{[a,b]})$, ent\~ao $I_f'=f$.
  \end{teo}
  \prova
  Seja $\varepsilon>0$, da continuidade de $f$ em $c$, existe $\delta>0$, tal que
  \[
    \forall x(x\in[a,b]\a |x-c|<\delta\lra |f(x)-f(c)|<\varepsilon),
  \]
  da\'i segue que em qualquer caso, i.e., seja $h<0$ ou seja $h>0$, tem-se 
  \[
    \Big|{I_f(c+h)-I_f(c)\over h}-f(c)\Big|=\Big|{1\over h}\int^{c+h}_cf-f(c)\,d\iota\Big|\leq\varepsilon,
  \]
  ou seja, $I_f'(c)=f(c)$. Particularmente, se $f\in\mathscr{C}(\R^{[a,b]})$, ent\~ao $I_f'=f$.\qed
  
  \begin{Def}
    Denotaremos o conjunto das fun\c c\~oes reais deriv\'aveis em $[a,b]$ por $\mathscr{D}(\R^{[a,b]})$, ou simplesmente por $\mathscr{D}$, quando n\~ao houver ambiguidade. 
  \end{Def}
  
  \begin{teo}
    Em $\mathscr{D}$, podemos considerar a rela\c c\~ao $D\subset\mathscr{D}\times\mathscr{D}$ definida por 
    \[
      (f,g)\in D\llra f'=g'.
    \]
    Desta forma, $D$ \'e uma rela\c c\~ao de equival\^encia. Ademais, dado $f\in\mathscr{D}$ a classe de equival\^encia determinada por $f$ \'e 
    \[
      [f]=\R+f=\{x:\exists\kappa(\kappa\in\R\a x=f+\kappa)\}.
    \]
  \end{teo}
  \prova Provar.\qed
  \begin{Def}
    Seja $f\in\R^{[a,b]}$. Qualquer fun\c c\~ao $F\in\R^{[a,b]}$, tal que $F'=f$, \'e chamada primitiva de $f$.
  \end{Def}
  
  \begin{Def}
    Seja $f\in\mathscr{C}(\R^{[a,b]})$. Definiremos
    \[
      \int f\,d\iota=[I_f].
      \label{int_ind}
    \]
    Ao s\'imbolo no lado esquerdo de (\ref{int_ind}) denominaremos de integral indefinida de $f$. Dessarte, um representante gen\'erico de $\int f\,d\iota$ \'e da forma $I_f+\kappa$, com $\kappa\in\R$.
  \end{Def}
  
  \'E comum na literatura escrever
  \[
    \int f\,d\iota=I_f+\kappa,
  \]
  em que $\kappa$ \'e uma constante real arbitr\'aria, quando na realidade $\int f\,d\iota$ \'e um conjunto.
  
  
  \theorem{}{\sl%
    Sejam $X\subseteq R$, $f:X\rightarrow R$ e $a$ um ponto de m\'inimo\/$($m\'aximo\/$)$ local de $f$ em $X$ tal que $f$ seja deriv\'avel em $a$. Ent\~ao $f'(a)=0$. 
  }
  
  \proof Seja $V_a$ uma vizinhan\c ca relativa de $a$ em $X$, tal que $a$ \'e um ponto de m\'aximo de $f$ em $V_a$. Por conveni\^encia definamos $\varkappa,\varsigma:X\setminus\{a\}\rightarrow R$ pelas respectivas leis
  $$
    \varkappa(x)={f(x)-f(a)\over x-a}\quad\hbox{e}\quad\varsigma(x)={x-a\over|x-a|}
  $$
  de tal maneira que \'e imediato inferir que $\lim_a\varkappa=f'(a)$ e, $\varsigma\varkappa\leq0$ em $V_a\setminus\{a\}$, pois $a$ \'e ponto de m\'aximo em $V_a$. Assim, notando que $\varsigma^2=1$ podemos inferir
  $$
    0\leq \lim_{a-}\varsigma^2\varkappa=\lim_{a-}\varkappa=f'(a)=\lim_{a+}\varkappa=\lim_{a+}\varsigma^2\varkappa\leq 0,
  $$
  \af, $f'(a)=0$.\qed
  
  
  Existe um teorema muito popular que tem como casos particulares o teorema de Rolle e o teorema do valor m\'edio. A fim de colocar as hip\'oteses do teorema sob uma perspectiva mais geral, notei que se usa a compacidade e a perfei\c c\~ao do dom\'inio. Geralmente toma-se o dom\'inio como sendo um compacto $[a,b]\subset R$, este conjunto como se sabe \'e perfeito, a saber, \'e um conjunto fechado onde todos os seus pontos s\~ao pontos limites. Como \'e sabido existe um conjunto compacto perfeito, que n\~ao \'e um intervalo, nomeadamente o conjunto de Cantor, que poderia ser um dom\'inio adequado para os prop\'ositos do teorema. Certamente, digo que esta generaliza\c c\~ao \'e de certa maneira irrelevante em rela\c c\~ao ao que se conhece usualmente. Em verdade, minha ignor\^ancia n\~ao enxerga nenhum fruto al\'em do usual. N\~ao obstante, deixarei-o assim, como uma curiosidade matem\'atica. 
  
  
  \theorem{ (Teorema do valor m\'edio de Cauchy)}{\sl%
    Sejam $K$ compacto perfeito, $(a,b)=(\min K,\max K)$$[$\footnote{Isto \'e uma igualdade de pares ordenados.}, $K^*=K\setminus\{a,b\}$ e $f,g:K\rightarrow R$ fun\c c\~oes cont\'inuas em $K$ e deriv\'aveis em $K^*$. Ent\~ao existe $c\in K^*$, tal que
  $$
    (f(b)-f(a))f'(c)=g'(c)(g(b)-g(a)).
  $$
  }
  
  \proof A prova tamb\'em \'e cl\'assica e consiste somente em construir uma $h:K\rightarrow R$ adequada. Pois bem, considere
  $h$ definida pela lei
  $$
    h(x)=(g(b)-g(a))(f(x)-f(a))+(f(a)-f(b))(g(x)-g(a)).
  $$
  A fun\c c\~ao $h$ \'e cont\'inua em $K$ \'e deriv\'avel em $K^*$. Al\'em disso, observemos que para todo $x\in K^*$
  $$
    h'(x)=(g(b)-g(a))f'(x)+(f(a)-f(b))g'(x).
  $$
  Como $K$ \'e compacto sua imagem \'e compacta, logo, existem $x_{\min},x_{\max}\in K$, tais que $h(x_{\min})=\min h(K)$ e $h(x_{\max})=\max h(K)$. Se $\{x_{\min},x_{\max}\}=\{a,b\}$, como $h(a)=h(b)=0$, decorrer\'a que $h$ ser\'a constante em $K$, como $K$ \'e perfeito, portanto infinito, podemos tomar qualquer $c\in K^*$. N\~ao obstante, se $\{x_{\min},x_{\max}\}\setminus\{a,b\}\neq\emptyset$, ent\~ao certamente existe $c\in\{x_{\min},x_{\max}\}\setminus\{a,b\}\subset K^*$. Em qualquer caso, sendo $h$ constante ou n\~ao, $c$ \'e um ponto cr\'itico de $h$, pois noutro caso ele \'e ponto extremo (m\'aximo ou m\'inimo global de $h$). Consequentemente,
  $$
    0=h'(c)=(g(b)-g(a))f'(c)+(f(a)-f(b))g'(c).
  $$
  ou equivalentemente
  $$
    (f(b)-f(a))f'(c)=g'(c)(g(b)-g(a)).
  $$
  \qed
  
  \'E preciso mencionar o porqu\^e de remover os pontos laterais do conjunto em an\'alise, por exemplo um compacto $[a,b]$. Por que a hip\'otese n\~ao d\'a simplesmente que $f$ seja deriv\'avel em $[a,b]$, mas em $]a,b[\;$? \'E porque existe um contra-exemplo. Com efeito, considere $f:[0,1]\ra R$, dada por $\sqrt{x}$, do c\'alculo sabemos que $f'(x)=(2\sqrt{x})^{-1}$, certamente $\lim_0f'=\infty$, todavia $f$ \'e cont\'inua em $[0,1]$. Dessarte, n\~ao \'e verdadeiro que se $f$ \'e cont\'inua em $[a,b]$ e deriv\'avel em $]a,b[$, ent\~ao existe $f'(a)$ em $[a,b]$, o mesmo vale para $f'(b)$ (considere $f:[-1,0]\ra R$ definida por $f=\sqrt{|x|}$). No que segue, adotarei um caminho can\^onico, considerando intervalos como se faz usualmente.
  
  
  \corollary{ (\tvm: Teorema do valor m\'edio)}{\sl%
    Seja $f$ cont\'inua em $[a,b]$ e diferenci\'avel em $]a,b[$. Ent\~ao existe $c\in]a,b[$, tal que
    $$
      f'(c)={f(b)-f(a)\over b-a}.
    $$
  }
  
  \proof Basta tomar $g:[a.b]\rightarrow R$, definida por $g(x)=x$, segue portanto o resultado.\qed
  
  \corollary{ (Teorema de Rolle)}{\sl%
    Seja $f$ cont\'inua em $[a,b]$ e diferenci\'avel em $]a,b[$. Ent\~ao se, $f(a)=f(b)$, existe $c\in]a,b[$, tal que $f'(c)=0$.
  }
  
  \proof Imediata ao resultado anterior.\qed
  
  \corollary{}{\sl%
    Sejam $f,g$ cont\'inuas em $[a,b]$ e diferenci\'aveis em $]a,b[$. Ent\~ao $f'=g'$ em $]a,b[$, se e somente se, existe $\varkappa\in R$, tal que $f=g+\varkappa$.
  }
  
  \proof Uma das condicionais \'e imediata. Prossigamos com a prova da outra. Com efeito, sejam $x,y\in [a,b]$ distintos, do \tvm\ existe $c\in]x,y[$, tal que
  $$
    f(y)-f(x)=f'(c)=g'(c)=g(y)-g(x) 
  $$
  ou equivalentemente
  $$
    f(x)-g(x)=f(y)-g(y).
  $$
  
  Como $x$ e $y$ s\~ao arbitr\'arios, fixando $y$ e estipulando $\varkappa=f(y)-g(y)$, obtemos que para todo $x\in [a,b]$, vale
  $f(x)=g(x)+\varkappa$, que era o requerido.\qed
  
  \corollary{}{\sl%
    Seja $f$ cont\'inua em $[a,b]$ e diferenci\'avel em $]a,b[$. Se $f'>0$ $(f'<0)$ em $]a,b[$, ent\~ao $f$ \'e crescente\/$($decrescente$\/)$ em $]a,b[$.
  }
  
  \proof Sejam $x,y\in(a,b)$ tais que $x<y$, suponhamos que $f'>0$ em $]a,b[$ do \tvm\ existe $c\in]a,b[$, tal que
  $$
    {f(y)-f(x)\over y-x}=f'(c)>0
  $$
  como $y-x>0$, segue que $f(y)>f(x)$. O outro caso basta tomar $g=-f$.\qed
  
  Neste ponto cabe algumas perguntas. \'E poss\'ivel garantir a difenciabilidade local de uma fun\c c\~ao $f$, bastando para isto que ela seja deriv\'avel em um ponto? A resposta \'e n\~ao. Considere a fun\c c\~ao $f:R\ra R$ dada por 
  $$
  f(x)=x+(-1)^{\chi_Q(x)}x^2
  $$ 
  esta fun\c c\~ao \'e descont\'inua em todo ponto $x\neq0$, consequentemente n\~ao deriv\'avel em $R\setminus\{0\}$, todavia $f'(0)=1$. Esta fun\c c\~ao responde em negativo \`a outra pergunta: Se $f'(c)>0$, ent\~ao $f$ \'e monot\^onica numa vizinhan\c ca de $c$? Podemos at\'e considerar uma fun\c c\~ao cont\'inua $g$ dada por
  $$
    g(x)=
    \left\{
    \begin{matrix}
      \hfill {x\over 2}+\sen({1\over x})x^2, & x\neq 0;\hfill\cr
      \hfill 0,                     & x=0.    \hfill\cr
    \end{matrix}
    \right.
  $$
  Observe que 
  $$
    g'(0)=\lim_{x\to 0}{g(x)-g(0)\over x}=\lim_{x\to 0}\Bigl({1\over2}+\sen\Bigl({1\over x}\Bigr)x\Bigr)={1\over2}>0,
  $$
  e que se $x\neq0$, ent\~ao
  
  $$
    g'(x)={1\over 2}-\cos({1\over x})+2x\sen({1\over x}).
  $$
  
  Considere $\{x_n\}$ e $\{y_n\}$ em $R$ tais que
  $$
    x_n={1\over 2n\pi}\quad\hbox{e}\quad y_n={1\over (2n+1)\pi}.
  $$
  \'E consp\'icuo que $\lim_nx_n=\lim_ny_n=0$, todavia 
  $$
    g'(x_n)=-1/2<0<3/2=g'(y_n).
  $$
  Como $g'$ \'e cont\'inua em $R\setminus\{0\}$ para cada $n\in\omega$ existem vizinhan\c cas $V_{x_n}$ e $V_{y_n}$ de $x_n$ e $y_n$, respectivamente, em que a derivada conserva o sinal, consequentemente existem fam\'ilas de vizinhan\c cas de pontos pr\'oximos de $0$, tais que, ora $f$ \'e estritamente decresente, ora $f$ \'e estritamente crescente.
  
  A seguir apresento um resultado bastante peculiar sobre fun\c c\~oes deriv\'aveis. Em meus estudos posteriores sem lembrar de tal fato, o conjecturei, atestei sua veracidade relendo o {\it Principles of Mathematical Analysis} de W. Rudin, a prova deste resultado encontrada no livro \'e concisa e elegante, a adaptei a seguir.
  
  \theorem{\ (Teorema de Darboux)}{\sl%
    Seja $f$ deriv\'avel em $[a,b]$ tal que $f'(a)<f'(b)$. Se $\lambda\in\;]f'(a),f'(b)[$, ent\~ao existe $c\in\;]a,b[$ tal que $f'(c)=\lambda$.
  }
  
  \proof Defina $g:[a,b]\ra R$, por $g(x)=f(x)-\lambda x$. Decerto que $g$ \'e deriv\'avel e portanto cont\'inua, com dom\'inio compacto. Consequentemente atingir\'a os extremos em seu dom\'inio. Observe que $g'(a)<0<g'(b)$. Localmente $a$ e $b$ s\~ao pontos de m\'aximo local, pois\footnote{Fui corrigido por uma llm, simplesmente ris\'ivel. Eu pensei que era necess\'ario que $g'$ fosse cont\'inua em $a$ e $b$, para determinar os pontos $x$ e $y$ como da demonstra\c c\~ao.}
  $$
    \lim_{x\to a+}{g(x)-g(a)\over x-a}=g'(a)<0<g'(b)=\lim_{x\to b-}{g(b)-g(x)\over b-x}.
  $$
  Portanto, existem $x,y\in\; ]a,b[$ com $x<y$, tais que $g(x)<g(a)$ e $g(y)<g(b)$, consequentemente o m\'inimo de $g$ estar\'a em $[x,y]\subset\;]a,b[$. Assim, existe $c\in\;]a,b[$, tal que $f'(c)-\lambda=g'(c)=0$, ou equivalentemente $f'(c)=\lambda$.\qed
  
  \theorem{ (Teste da derivada primeira)}{\sl%
    Seja $f$ deriv\'avel em $]a,b[$ e $m\in\;]a,b[$ tal $f'(m)=0$. Ent\~ao, se existir uma vizinhan\c ca $V_m$ de $m$ em $]a,b[$, tal que
  $$
    \forall x\Bigl(x\in V_m\setminus\{m\}\lra{f'(x)\over x-m}<0\,\Bigr({f'(x)\over x-m}>0\Bigl)\Bigr).
  $$
  ent\~ao $m$ \'e ponto de m\'aximo\/$($m\'inimo\/$)$ em $V_m$, ou equivalentemente, $m$ \'e ponto de m\'aximo\/$($m\'inimo\/$)$ local.
  }
  
  \proof Suponhamos por {\it reductio ad absurdum} que existe $x\in V_m$, tal que $f(m)<f(x)$. Certamente, $x\neq m$, pois $f(x)\neq f(m)$. Primeiramente, digamos que $x>m$, segue-se do \tvm\ que existe $c\in\;]m,x[$, tal que
  $$
    f'(c)={f(x)-f(m)\over x-m}>0.
  $$
  N\~ao obstante, temos da hip\'otese que
  $$
   {f'(c)\over c-m}<0.
  $$
  Todavia, como $c>m$, tem-se $f'(c)<0$, ou seja, incorremos numa contradi\c c\~ao. Homologamente, suponhamos que $x<m$, novamente invocando o \tvm\ conclu\'imos que existe $c\in\;]x,m[$, tal que
  $$
    f'(c)={f(x)-f(m)\over x-m}<0.
  $$
  No entanto,
  $$
    {f'(c)\over c-m}<0,
  $$
  o que em virtude de $c-m<0$ acarreta $f'(c)>0$, o que \'e outra contradi\c c\~ao. Por fim, incorremos numa contradi\c c\~ao por supor que a tese era falsa, consequentemente a condicional \'e verdadeira, i.e., $m$ \'e um ponto de m\'aximo em $V_m$, consequentemente um ponto de m\'aximo local. A outra prova \'e inteiramente an\'aloga.\qed
  
  
  \theorem{ (Teste da derivada segunda)}{\sl%
    Seja $f$ cont\'inua em $[a,b]$ e diferenci\'avel em $]a,b[$. Se existe $c\in\;]a,b[$ tal que $f'(c)=0$ e $f''(c)>0$ $(f''(c)<0)$, ent\~ao $c$ \'e um ponto de m\'inimo\/$($m\'aximo\/$)$ local de $f$.
  }
  
  \proof Se $f''(c)>0$, ent\~ao existe uma vizinhan\c ca $V_c$ de $c$, tal que 
  $$
    \forall x\Bigl(x\in V_c\setminus\{c\}\lra{f'(x)\over x-c}>0\Bigr).
  $$
  Com efeito, note que
  $$
    \lim_{x\to c-}{f'(c)-f'(x)\over x-c}=f''(c)=\lim_{x\to c+}{f'(c)-f'(x)\over c-x}
  $$
  Da defini\c c\~ao de derivada existe uma vizinhan\c ca $V_c$, tal que
  $$
    \forall x\Bigl(x\in V_c\lra {f'(c)-f'(x)\over x-c}>0\Bigr).
  $$
  
  Seja $x\in V_c$, suponhamos inicialmente $x<c$, segue que $f'(x)<f'(c)=0$, da\'i $f'(x)/(x-c)>0$. Agora, suponhamos que $c<x$, da\'i vem que $0=f'(c)<f'(x)$, logo, $f'(x)/(x-c)>0$. O que prova a quantifica\c c\~ao. Isto por sua vez, em decorr\^encia de resultados anteriores, acarreta que $c$ \'e um ponto de m\'inimo local. A prova \'e inteiramente an\'aloga para $f''(c)>0$, pois o argumento \'e dual.
  \qed
  
  \fi % IGNORAR

  
  \chapter{Geometria anal\'itica}
  
  \iffalse % IGNORAR
  No que segue vamos tentar construir o conceito de vetor. A princ\'inpio era tratar axiomaticamente esta constru\c c\~ao, mas tenho uma certa convic\c c\~ao que esta tarefa foge as minhas capacidades.
  
  Primeiro vamos considerar um conjunto $E$ chamado espa\c co, este espa\c co possui certos objetos indefinidos, como pontos denotados por letras mai\'usculas do alfabeto latino $A,\ldots,Z$; retas denotadas por letras min\'usculas do alfabeto latino $a,\ldots,z$; e planos denotados por letras do alfabeto grego $\alpha,\ldots,\omega$.
  
  Existem v\'arias formas de se alcan\c car a geometria euclidiana, i.e., podemos partir de diferentes axiomas e intuitivamente alcan\c car o espa\c co euclidiano, i.e., constru\'i-lo axiomaticamente.
  
  Doravante explicitaremos os axiomas necess\'arios e algumas defini\c c\~oes auxiliares, ao progredirmos.
  
  \axiom{}{\sl%
    Dois pontos distintos $A$ e $B$, determinam uma, e somente uma, reta. Tal reta ser\'a denotada por $AB$.
  }
  
  \axiom{}{\sl%
      Admitiremos uma no\c c\~ao primitiva, a no\c c\~ao de `estar entre'. Dados tr\^es pontos numa reta, somente um deles est\'a entre os outros dois. 
  }
  
  \definition{}{\sl%
    Ao conjunto de pontos entre dois pontos distintos $A$ e $B$ de uma reta $r$ denotaremos por $]A,B[$. Tamb\'em definiremos
    $$
      \vbox{
        \halign{
          \hfill#&#\hfill\cr
          $[A,B]$&$\;=\;]A,B[\,\cup\,\{A,B\};$\cr
          $[A,B[$&$\;=\;]A,B[\,\cup\,\{A\};$\cr
          $]A,B]$&$\;=\;]A,B[\,\cup\,\{B\}.$\cr
        }
      }
    $$
  
    A qualquer um destes conjuntos chamaremos de seguimento. $A$ e $B$ s\~ao chamados extremos ou extremantes. Segundo esta defini\c c\~ao $[A,B]=[B,A]$. Diremos que um seguimento com extremos $A$ e $B$ \'e nulo se, e somente se, $A=B$.
  }
  
  
  \definition{}{\sl%
    Sejam $A$ e $B$, dois pontos distintos definimos a semirreta com origem em $A$ determinada por $B$, como sendo o conjunto
    $$
      s_{(A,B)}=\Bigl\{P\in AB: P\in[A,B]\vee B\in[A,P]\Bigr\}
    $$
  }
  
  \definition{}{\sl%
    Seja $\alpha$ um plano e $r,s\subset\alpha$ retas quaisquer, diremos que $r$ e $s$ s\~ao paralelas se, $r=s$ ou $r\cap s=\emptyset$. 
  }
  
  \axiom{}{\sl%
    Seja $r$ uma reta e $P\notin r$, existe uma \'unica reta $s$ paralela $r$ passando por $P$, i.e., $P\in s$. O plano fica dividido em dois conjuntos disjuntos. Defina no plano $\alpha$ a seguinte rela\c c\~ao entre pontos $A\sim B$, se e somente se, $[A,B]\cap r=\emptyset$. 
  }
  
  \axiom{}{\sl%
    Existe uma fun\c c\~ao $\mu:E\rightarrow R_+$, satisfazendo as seguintes propriedades:
    \begin{enumerate}[label = \Roman*.]
      \item{%
        $\mu(S)=0$ se, e somente se $S$ \'e um seguimento nulo;
      }
      \item{%
        Seja ${\cal F}=\{S_i\in{\mathscr{P}(E)}:i\in n+1\}$ uma fam\'ilia de seguimentos disjuntos aos pares. Ent\~ao
        $$
          \mu\biggl(\bigcup_{i\in n+1}S_i\biggr)=\sum_{i=0}^n\mu(S_i);
        $$
      }
      \item{%
        Para toda semirreta $s_{(A,B)}$ e todo $c\in R_+$, existe $C\in s_{(A,B)}$, tal que $\mu([A,C])=c$.
      }
    \end{enumerate}
  }
  
  \definition{}{\sl%
    Sejam $\alpha$ um plano e $r\subset\alpha$ uma reta 
  }
  
  \definition{}{\sl% 
    Sejam $A$ e $B$ dois pontos distintos definimos por seguimento orientado como sendo o par ordenado $(A,B)$. Diremos que o seguimento \'e nulo se, e somente se, $A=B$.
  }
  
  N\~ao confundir seguimento orientado que \'e um par ordenado $(A,B)$, com o seguimentos definidos anteriormente. 
  
  \definition{}{\sl%
    Sejam $(A,B)$ e $(C,D)$ dois seguimentos n\~ao nulos.
  }
  
  \medskip
  
  \noindent{\bf Seguimentos equipolentes.} Na geometria anal\'itica existe um conceito chamado {\bf vetor}, cuja defini\c c\~ao baseia-se na equipol\^encia, uma rela\c c\~ao de equival\^encia definida no conjunto dos seguimentos orientados. Seguiremos aqui a nota\c c\~ao de P. Boulos e I. de Camargo dada no livro {\it Geometria Anal\'itica um Tratamento Vetorial}. Seja $E$ o espa\c co euclidiano, um seguimento orientado \'e um par de pontos de $E$, ou em outros termos, os seguimentos orientados s\~ao elementos de $E\times E$. Sobre $E\times E$ definimos a seguinte rela\c c\~ao de equival\^encia
  $(A,B)\sim(C,D)$ se, e somente se 
  $$
   ({\cal E})\quad\bigl((A=C\longleftrightarrow B=D)\vee AC\parallel BD\bigr)\wedge\bigl((A=B\longleftrightarrow C=D)\vee AB\parallel CD\bigr)
  $$
  
  Aqui por conveni\^encia $AB$ denota a reta determinada por $A$ e $B$, quando $A\neq B$ e, $AA=\{A\}$. Definimos a rela\c c\~ao
  $$
    AB\parallel CD\longleftrightarrow AB=CD\;\vee\; AB\cap CD=\emptyset,
  $$
  assim a express\~ao $({\cal E})$ \'e significativa. Nestas condi\c c\~oes seguimentos nulos, isto \'e da forma $(A,A)$, s\~ao equivalentes somente a seguimentos nulos.
  
  Doravante denotaremos tal rela\c c\~ao de equival\^encia por {\bf equipol\^encia}. Assim, se dois seguimentos são equivalentes, s\~ao chamados por este motivo de {\bf equipolentes}.
  
  \theorem{}{\sl Seja $(A,B)$ um seguimento n\~ao nulo, i.e., $A\neq B$, e $P\in E$, ent\~ao existe um seguimento orientado $(P,P_{(A,B)})\sim(A,B)$, i.e., um seguimento equipolente a $(A,B)$ com origem em $P$.}
  
  \proof Primeiro suponhamos que $P\notin AB$, pelo axioma das paralelas na geometria euclidiana existe uma reta $r$ paralela a $AB$ passando por $P$. Note que $B\notin AP$, pois caso contr\'ario $P\in AP=AB$, o que contradiz nossa hip\'otese inicial. Pelo mesmo axioma existe uma reta $s$ paralela a $AP$ passando por $B$. Afirmo que $s\cap r\neq\emptyset$, suponhamos por redu\c c\~ao ao absurdo o contr\'ario. Necessariamente $s\parallel r$, consequentemente $s\parallel AB$, como $B\in s\cap AB$, decorre que $s=AB$, todavia $A\notin s$, pois $s$ \'e uma reta paralela a $AP$, contendo $B\notin AP$. Esta contradi\c c\~ao nos leva a concluir que existe $P_{(A,B)}\in r\cap s$. Seguidamente observemos que $AB\parallel r=PP_{(A,B)}$ e $AP\parallel s=BP_{(A,B)}$, consequentemente $(A,B)\sim (P,P_{(A,B)})$. Suponhamos agora que $P\in AB$, existe $(Q,Q_{(A,B)})\sim (A,B)$, tal que $Q\notin AB$ assim $A\notin QQ_{(A,B)}$, pelo mesmo modo existe $(P,P_{(Q,Q_{(A,B)})})\sim (Q,Q_{(A,B)})$, por transitividade $(P,P_{(Q,Q_{(A,B)})})\sim (A,B)$.\qed
  
  Este teorema nos mostra que se $(A,B)$ \'e n\~ao nulo e $(A,B)\sim(C,D)$ e $C\notin AB$, ent\~ao $ABDC$ \'e um paralelogramo. Da geometria euclidiana tem-se uma no\c c\~ao b\'asica de comprimento (medida), sabe-se que num paralelogramo lados paralelos t\^em o mesmo comprimento. Como consequ\^encia seguimentos equipolentes t\^em o mesmo comprimento.
  
  \definition{}{\sl Sejam $(A,B)$ e $(C,D)$ seguimentos n\~ao nulos, diremos que eles possuem o mesmo sentido se $AB\parallel CD$ e existem $(E,F)\sim(A,B)$ $(G,H)\sim (C,D)$, tais que $EF\cap GH=\emptyset$ e $\overline{EG}\cap \overline{FH}=\emptyset$.}
  
  \theorem{} Seguimentos equipolentes n\~ao nulos t\^em o mesmo sentido.
  
  \proof Podemos sem perdas de generalidade supor que $(A,B)\sim(C,D)$, sejam tais que $AB\cap CD=\emptyset$. Necessariamente $AB\parallel CD$ e $AC\parallel BD$, como $B\notin AC$, decorre que $\overline{AC}\cap\overline{BD}=\emptyset$.
  
  \corollary{}{\sl Seguimentos orientados s\~ao nulos ou t\^em o mesmo sentido e mesmo comprimento.}
  
  \proof Imediata \`a discuss\~ao anterior.\qed
  
  Note que a defini\c c\~ao de equipol\^encia $({\cal E})$ implica o corol\'ario anterior, mas a priori n\~ao \'e sabido se dois seuigmentos n\~ao nulos que t\^em o mesmo sentido e comprimento s\~ao equipolentes. Para o que vamos fazer n\~ao \'e necess\'ario provar a rec\'iproca. A defini\c c\~ao da rela\c c\~ao de equipol\^encia evita fazer coment\'arios pr\'evios sobre comprimentos, e tamb\'em nos poupa de tratar os casos em que seguimentos s\~ao nulos ou n\~ao.
  
  \definition{}{\sl Um vetor \'e o conjunto
    $$
      \overrightarrow{AB}=\Bigl\{(C,D)\in E\times E:(C,D)\sim(A,B)\Bigr\}
    $$
    em outros termos, um vetor \'e uma classe de equival\^encia da rela\c c\~ao de equipol\^encia $({\cal E})$.
  
    Quando n\~ao se quiser fazer refer\^encia a um representante da classe, escolheremos uma letra do alfabeto latino comumente $u,v,w,x,y,z$ e encimamos-la por uma flecha, e.g., $\vec{x}$.
  
    Os vetores herdam as propriedades dos seguimentos orientados e n\~ao orientados dados por de seus representantes, neste caso, paralelismo, sentido e comprimento. Denotaremos o comprimento de um vetor $\vec x$, por $\|\vec x\|$ e a este n\'umero daremos o nome de norma. Observe que a no\c c\~ao de comprimento faz sentido para seguimentos nulos, neste caso trivialmente se  $\vec x$ possui representante nulo, ent\~ao $\|\vec x\|=0$. Ademais, nestas condi\c c\~oes escreveremos $\vec 0$ para denotar a classe de equival\^encia dos seguimentos orientados nulos. O conjunto das classes de equival\^encia ser\'a denotada por $\cal V$. 
  }
  
  A seguir muniremos o conjunto $\cal V$ de uma estrutura alg\'ebrica, a saber a adi\c c\~ao $+$.
  
  \definition{}{\sl Dados dois vetores $\overrightarrow{AB}$ e $\overrightarrow{CD}$, definiremos o s\'imbolo $\overrightarrow{AB}+\overrightarrow{CD}$ como sendo o vetor $\overrightarrow{AB_{(C,D)}}$}, em outros termos a adi\c c\~ao trata-se de construir um tri\^angulo (possivelmente degenerado) com v\'ertices $A$, $B$ e $B_{(C,D)}$, em que $(B,B_{(C,D)})\sim (C,D)$, e considerar o vetor (a classe) cujo representante seja o seguimento orientado $(A,B_{(C,D)})$.
  
  $\dagger$ Mostrar que a defini\c c\~ao indepente do representante!
  
  \theorem{}{\sl Sejam $\overrightarrow{AB}$ e $\overrightarrow{CD}$ dois vetores quaisquer. Ent\~ao $(A,B_{(C,D)})\sim(C,D_{(A,B)})$}.
  
  \newpage

  \fi % IGNORAR

  \section{C\^onicas}

  \begin{Def}
    Consideremos $g$ um polin\^omio de duas vari\'aveis de grau $2$ com coeficientes em $\R$ dado por
    \[
      g(x,y)=ax^2+bxy+cy^2+dx+ey+f.
    \]
    Uma c\^onica $C$ \'e o locus (lugar geom\'etico) ou conjunto de pontos
    \[
      C=\{(x,y):(x,y)\in\R^2\,\a\, g(x,y)=0\}.
    \]
    \label{def3012252158}
  \end{Def}

  O objetivo das pr\'oximas se\c c\~oes \'e aplicar transforma\c c\~oes ao sistema de coordenadas, de tal maneira que a c\^onica seja facilmente reconhecida. Doravante, a menos que haja men\c c\~ao expl\'icita em contr\'ario, sempre quando referirmos a $g$, estaremos considerando um polin\^omio como estipulado na \rf{Defini\c c\~ao}{def3012252158}.

  % TODO: Completar e organizar esta seção.
  \subsection{Elimina\c c\~ao dos termos lineares por transla\c c\~oes}

  Dados pontos $\varOmega=(h,k)$ e $(x,y)$ de $\R^2$, as coordenadas $(u,v)$ de $(x,y)$ relativo ao sistema cuja origem \'e $\varOmega$, s\~ao simplesmente dadas pela identidade
  \[
    (u,v)=(x,y)-(h,k).
  \]
  Em verdade, o sistema de coordenadas \'e a transforma\c c\~ao afim $T:\R^2\ra\R^2$, dada por
  \[
    T(v)=v-\varOmega.
  \]
  Em outros termos, estamos calculando as coordenadas dum ponto $v$ relativo a um sistema de coordenadas cuja a origem \'e precisamente $\varOmega$.

  A rela\c c\~ao das coordenadas $(u,v)$ do sistema transladado segundo o sistema antigo \'e expressa pelo sistema
  \[
    \Biggl\{\,
    \begin{aligned}
      x&=u+h\cr
      y&=v+k\cr
    \end{aligned}
  \]
  Em seguida, calculemos $g(x,y)=g(u+h,v+k)$. Conformemente,
  \[
    \begin{aligned}
      g(x,y)&=a(u+h)^2+b(u+h)(v+k)+c(v+k)^2+d(u+h)+e(v+h)+f\cr
            &=au^2+buv+cv^2+u(2ah+bk+d)+v(bh+2ck+e)\cr
            &\qquad+\underbrace{ah^2+bhk+ck^2+dh+ek+f}_{g(h,k)}\cr
            &=au^2+buv+cv^2+u(2ah+bk+d)+v(bh+2ck+e)+g(h,k)\cr
    \end{aligned}
  \]

  Nosso objetivo \'e eliminar os termos lineares segundo $u$ e $v$. Para tanto, precisamos resolver o sistema

  % TODO: Consertar.
  \[
    \Biggl\{\,
    \begin{aligned}
      2ah +bk + d &=0\cr
      bh +2ck + e &=0\cr
    \end{aligned}
    \label{S0301261148}
  \]
  que por sua vez, \'e equivalente ao sistema
  % TODO: Consertar.
  \[
    \left\{\,
    \begin{aligned}
               ah +{b\over2}k &=-{d\over2} \cr
      {b\over 2}h +ck         &=-{e\over2} \cr
    \end{aligned}
    \right.
    \label{S2912252034}
  \]
  Cuja solubilidade \'e atestada pelo determinante
  \[
    \left|\;\,
    \begin{aligned}
      a         &\;\;&{b\over 2} \cr
      {b\over2} &\;\;&c          \cr
    \end{aligned}
    \;\,\right|
    \;=\;ac-{b^2\over4}.
  \]
  \'E sabido que o sistema (\ref{S2912252034}) admite uma \'unica solu\c c\~ao, i.e., \'e determinado, se, e somente se, $ac-b^2/4\neq0$. Caso contr\'ario, o sistema admite infinitas solu\c c\~oes, neste caso diz-se que ele \'e indetemindado, ou n\~ao admite solu\c c\~oes, i.e., \'e imposs\'ivel.

  Destarte, temos, um m\'etodo pragm\'atico para determinar se \'e poss\'ivel eliminar os termos lineares do polin\^omio $g$, a saber, se, e somente se, o sistema (\ref{S2912252034}) admite solu\c c\~oes.

  A seguir est\'a provado que no caso do sistema (\ref{S2912252034}) admitir infinitas solu\c c\~oes, o termo independente de $g$ no novo sistema \'e inexor\'avel \`a escolha da solu\c c\~ao do sistema (\ref{S2912252034}).


  \begin{teo}[Exerc\'icio 23-10 {[Boulos e Camargo]}]
    Se o sistema (\ref{S2912252034}) admite infinitas solu\c c\~oes, ent\~ao $g$ \'e constante no seu conjunto de solu\c c\~oes.
  \end{teo}
  \prova Sejam $(h,k)$ e $(u,v)$ solu\c c\~oes do sistema (\ref{S2912252034}). Em verdade n\~ao \'e dif\'icil notar que $(h+k,u+v)$ tamb\'em o \'e, bastando para isto substituir as respectivas coordenadas do par no sistema (\ref{S2912252034}) e atestar as igualdades.
 
  Notemos em seguida que qualquer solu\c c\~ao $(x,y)$ de (\ref{S2912252034}), satisfaz
  \[
    ax^2+bxy+cy^2+{d\over2}x+{e\over2}y=x\Bigl(ax+{b\over2}y+{d\over2}\Bigr)+y\Bigl({b\over2}x+cy+{e\over2}\Bigr)=0
    \label{eq2912252053}
  \]
  e
  \[
    \begin{aligned}
      g(x,y)&=ax^2+bxy+cy^2+dx+ey+f\cr
            &=x\Bigl(ax+{b\over2}y+{d\over2}\Bigr)+y\Bigl({b\over2}x+cy+{e\over2}\Bigr)+{d\over2}x+{e\over2}y+f\cr
            &={d\over2}x+{e\over2}y+f.
    \end{aligned}
    \label{eq2912252123}
  \]

  Em conformidade, segue-se
  \[
    2\Bigl[u\Bigl(a(u+h)+{b\over 2}(k+v)+{d\over2}\Bigr)+v\Bigl({b\over 2}(u+h)+c(v+k)+{e\over2}\Bigr)\Bigr]=0
    \label{eq2912252055}
  \]

  de (\ref{eq2912252053}) com $(x,y)=(u,v)$ e (\ref{eq2912252055}) podemos inferir
  \[
    au^2+buv+cv^2+{d\over 2}u+{e\over 2}v+2ahu+2ckv+bhv+buk=0.
    \label{eq2912252132}
  \]
  Como consequ\^encia, de (\ref{eq2912252053}), (\ref{eq2912252123}) e (\ref{eq2912252132}) temos
  \[
    \begin{split}
      g(h,k)&=ah^2+bhk+ck^2+dh+ck+f\cr
            &=ah^2+bhk+ck^2+dh+ck+f+au^2+buv+cv^2\cr
            &\qquad+{d\over 2}u+{e\over 2}v+2ahu+2ckv+bhv+buk\cr
            &=a(h+u)^2+b(u+h)(v+k)+c(k+v)^2\cr
            &\qquad+{d\over2}(h+u)+{e\over2}(k+v)+{d\over2}u+{e\over2}v+f\cr
            &={d\over2}u+{e\over2}v+f\cr
            &=g(u,v)
    \end{split}
  \]
  Em outros termos $g$ \'e constante no conjunto de solu\c c\~oes do sistema (\ref{S2912252034}).\qed

  \subsection{Rota\c c\~oes de sistema de coordenadas}

  Podemos enxergar algrebica e isometricamente $\R^2$ como $\C^2$, da\'i a base can\^onica \'e simplesmente ${\mathcal B}=\{1,i\}$. Como \'e sabido uma rota\c c\~ao da base por um \^angulo $\theta$ em rela\c c\~ao ao eixo gerado por $1$ no sentido anti-hor\'ario \'e simplesmente uma multiplica\c c\~ao complexa por $e^{i\theta}$, a nova base \'e simplesmente ${\mathcal B}_\theta=\{e^{i\theta},e^{i(\theta+\pi/2)}\}$. Destarte, dado $w$, existem $u,v\in\R$, tais que $w=(u,b)_{{\cal B}_\theta}$, evidentemente 
  \[
    w=(u\cos\theta-v\sen\theta)+i(u\sen\theta+v\cos\theta).
  \]
  Assim, as coordenadas $x$ e $y$ de $w$ relativas ao sistema can\^onico \'e dado pela identidade
  \[
    \begin{bmatrix}
      x\cr
      y\cr
    \end{bmatrix}
    =
    \begin{bmatrix}
      \cos\theta & -\sen\theta\cr
      \sen\theta & \cos\theta\cr
    \end{bmatrix}
    \begin{bmatrix}
      u\cr
      v\cr
    \end{bmatrix}
  \]

  \subsection{Elimina\c c\~ao do termo quadr\'atico misto por rota\c c\~oes}


  \chapter{\'Algebra linear}
  
  \section{Espa\c cos vetoriais}
  
  A seguir apresentarei alguns exerc\'icios pertinentes de [Hoffman].
  
  \begin{pro}
    Sejam $V$ um espa\c co vetorial e $W_i\subset V$, com $i\in\{1,2\}$, subespa\c cos vetoriais tais que $W=W_1\cup W_2$ \'e um subespa\c co de $V$. Ent\~ao, existe um $i\in\{1,2\}$, tal que $W_i\subset W_j$, com $i\neq j$.
  \end{pro}
  \prova Suponhamos por {\it reductio ad absurdum} que
  \[
    W_1\setminus W_2\neq\emptyset\a W_2\setminus W_1\neq\emptyset
  \]
  Sejam $w_1\in W_1\setminus W_2$ e $w_2\in W_2\setminus W_1$, das hip\'oteses $w=w_1+w_2\in W$. Todavia, $w\not\in W_1$ e $w\not\in W_2$. Pois, digamos que $w\in W_1$, ent\~ao $w_2=w-w_1\in W_1$, o que \'e uma contradi\c c\~ao pois $w_2\in W_2\setminus W_1$. Agora se $w\in W_2$, ent\~ao analogamente $w_1=w-w_2\in W_2$, outra contradi\c c\~ao, pois $W_1\setminus W_2$. Conclu\'imos, portanto, que existe $i\in\{1,2\}$, tal que $W_i\subset W_j$, com $i\neq j$. \qed
  
  \begin{pro}
    Seja $K$ um corpo, $W$ um espa\c co vetorial sobre $K$. Considere $V=W^K$, munido com a soma e produto por escalar usuais. Ent\~ao 
    \[
      V_i=\{f:f\in V\a f(-x)=-f(x)\}
    \]
    e
    \[
      V_p=\{f:f\in V\a f(-x)=f(x)\}
    \]
    s\~ao subespa\c cos de $V$, tais que 
    \[
      V_i\cap V_p=\{0\}\a V=V_i+V_p.
    \]
  \end{pro}
  \prova Primeiro \'e evidente que $0\in V_i$. Sejam agora $f,g\in V_i$ e $\varkappa\in K$, temos que para todo $x\in K$, vale
  \[
    \begin{split}
      (\varkappa f+g)(-x)&=\varkappa f(-x)+g(-x)\cr 
                        &=-\varkappa f(x)-g(x)\cr
                        &=-(\varkappa f(x)+g(x))\cr
                        &=-(\varkappa f+g)(x),
    \end{split}
  \]
  consequentemente $V_i$ \'e um subespa\c co de $V$. O outro caso \'e ainda mais simples de ser provado, o argumento \'e inteiramente semelhante.
  
  Seja agora $f\in V_i\cap V_p$, temos para todo $x\in K$
  \[ 
    f(x)=f(-x)=-f(x),
  \]
  logo $f=0$ e, consequentemente $V_i\cap V_p=\{0\}$.
  
  Por fim, seja $f\in V$. Note que $f_i,f_p\in V$ dadas por
  \[
    f_i(x)={f(x)-f(-x)\over 2}\quad\a\quad f_p(x)={f(x)+f(-x)\over 2}
  \]
  s\~ao tais que, para todo $x\in K$ vale
  \[
    f_i(-x)={f(-x)-f(x)\over 2}=-{f(x)-f(-x)\over 2}=-f_i(x),
  \]
  e. 
  \[
    f_p(-x)={f(-x)+f(x)\over 2}={f(x)+f(-x)\over 2}=f_p(x),
  \]
  i.e., $(f_i,f_p)\in V_i\times V_p$. Ademais, $f=f_i+f_p$, consequentemente $V=V_i+V_p$.\qed
  \begin{pro}
    Sejam $V_i\subset V$, com $i\in\{1,2\}$, subespa\c cos do espa\c co vetorial $V$, tais que 
    \[ 
      V=V_1+V_2\quad\a\quad V_1\cap V_2=\{0\}.
    \]
    Ent\~ao para todo $v\in V$ existe \'unico $(v_1,v_2)\in V_1\times V_2$, tais que $v=v_1+v_2$.
  \end{pro}
  \prova A exist\^encia decorre da defini\c c\~ao, provemos, portanto, a unicidade. Para tanto, sejam $v\in V$ e $v_i,w_i\in V_1$, com $i\in\{1,2\}$, tais que
  \[
    v=v_1+v_2=w_1+w_2,
  \]
  temos que 
  \[
    v_1-w_1=w_2-v_2\in V_1\cap V_2=\{0\}.
  \]
  Decorre que
  \[
    v_1-w_1=0=v_2-w_2,
  \]
  o que por sua vez acarreta $v_i=w_i$, com $i\in\{1,2\}$.\qed
  
  \parte{Ci\^encias naturais}
  
  \chapter{Equa\c c\~ao de carga de um capacitor}
  \theorem{}{\sl%
  Seja um capacitor de capacit\^ancia $C$ conectado em s\'erie com um resistor de resist\^encia $R$ e \`a uma bateira de tens\~ao $V_f$. Admitindo-se que o capacitor esteja inicialmente descarregado a equa\c c\~ao de carga do capacitor \'e dada por:
  $$
    V(t)=V_f(1-e^{-{t\over RC}}).
  $$
  }
  
  \proof A corrente no capacitor $I(t)$ \'e a mesma que a corrente no resistor, consequentemente
  $$
    (\ddag)\qquad{V_f-V(t)\over R}=I(t)={dQ\over dt}(t)=C{dV\over dt}(t)
  $$
  em que $V(t)$ \'e a tens\~ao no capacitor no instante $t$. Temos portanto de $(\ddag)$ que
  $$
    {1\over RC}={dV\over dt}(t){1\over V_f-V(t)}=-{d\over dt}\bigl(V_f-V(t)\bigr){1\over V_f-V(t)}
  $$
  em consequ\^encia do lema que vimos e multiplicando ambos membros por $-1$, decorre que
  $$
    -{1\over RC}={d\over dt}\bigl(V_f-V(t)\bigr){1\over V_f-V(t)}={d\over dt}\Bigl(\ln\bigl|V_f-V(t)\bigr|\Bigr)
  $$
  do teorema fundamental do c\'alculo vem
  $$
    -{T\over RC}=\int_{0}^T{d\over dt}\Bigl(\ln\bigl|V_f-V(t)\bigr|\Bigr)\,dt=\ln\bigl|V_f-V(t)\bigr|\biggr|_0^T
  $$
  para $T\geq0$.
  
  Agora admitindo-se que $V(0)=0$, e que $V_f-V(t)>0$ para todo $t\in R_+$ obtemos
  $$
    -{T\over RC}=\ln\bigl(V_f-V(t)\bigr)\biggr|_0^T=\ln {V_f-V(T)\over V_f}
  $$
  donde se conclui
  $$
    {V_f-V(T)\over V_f}=e^{-T\over RC}
  $$
  que por sua vez implica
  $$
    V(T)=V_f(1-e^{-{T\over RC}})
  $$
  como $T$ \'e arbitr\'ario podemos escrever sem perdas
  $$
    V(t)=V_f(1-e^{-{t\over RC}})
  $$
  para todo $t\geq0$.\qed
  
  \noindent{\bf Como tratar adequadamente convers\~oes entre grandezas, e.g., grandezas angulares?\ }{\sl%
    Como explicar rigorosamente a igualdade que se encontra rotineiramente em livros: 
    $$
      2\pi\,{\rm rad}=360^\circ\,{\rm?}
    $$
    Rigorosamente, os dois objetos s\~ao distintos, de maneira que a identidade n\~ao \'e siginificativa.
  }
  
  \parte{A fazer}
  \begin{enumerate}[label = {\bf \Roman*.}]
    \item{
      Geometria euclidiana;
    }
    \item{
      Geometria anal\'itica;
    }
    \item{
      \'Algebra Linear;
    }
    \item{
      Anel dos inteiros;
    }
    \item{
      MMC e MDC;
    }
    \item{
      Anel dos polin\^omios;
    }
    \item{
      Anel das matrizes;
    }
    \item{
      Espa\c cos vetoriais;
    }
    \item{
      M\'odulos;
    }
    \item{
      Regra de tr\^es simples e composta;
    }
    \item{
      Inconsist\^encia de modelos LLM;
    }
    \item{
      Diferencibilidade de fun\c c\~oes reais inversas de vari\'aveis reais;
    }
    \item{
      Mudan\c ca de vari\'aveis em integrais;
    }
    \item{
      Defini\c c\~ao de primitivas ou integrais indefinidas.
    }
  \end{enumerate}


  \printbibliography
\enddoc


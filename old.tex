  % TODO: Dá para melhorar?
  \begin{teo}[Bernstein-Cantor-Schr\"oder {[Revisar]}]
  Sejam $X$ e $Y$ conjuntos tais que existe uma inje\c c\~ao pr\'opria de $X$ em $Y$ e vice-versa. Ent\~ao $X$ \'e equivalente a $Y$.
  \end{teo}

  \prova Supomos que $f(X)\subset Y$ e $g(Y)\subset X$ (aqui $\subset$ \'e a inclus\~ao pr\'opria). Destas inclus\~oes deriva-se facilmente a proposi\c c\~ao
  $$
    (\Delta)\qquad\forall i\Bigl(i\in N\longrightarrow(g\!f)^i(X)\supset(g\!f)^i\bigl(g(Y)\bigr)\supset(g\!f)^{i+1}(X)\Bigr).
  $$
  Para cada $i\in N$, fa\c camos:
  $$
    A_{2i}=(g\!f)^i(X)\quad\hbox{e}\quad A_{2i+1}=(g\!f)^i\bigl(g(Y)\bigr)
  $$
  em vista de $(\Delta)$ decorre que $\{A_i:i\in N\}$ \'e uma fam\'\i lia de conjuntos estritamente decrescente.

  Sejam agora
  $$
    C_i=(g\!f)^i\bigr(X\setminus g(Y)\bigr)\quad\hbox{e}\quad D_i=(g\!f)^ig\bigl(Y\setminus f(X)\bigr)
  $$
  com $i\in N$.

  Definamos
  $$
    A=\Cap_{i\in N}A_i,\quad C=\Cup_{i\in N}C_i\quad e\quad D=\Cup_{i\in N}D_i.
  $$
  Estes s\~ao disjuntos aos pares. De fato, inicialmente provemos que $C\cap D=\emptyset$. Note que
  $$
    \forall i,j\bigl(i,j\in N\longrightarrow C_i\cap D_j=\emptyset\bigr),
  $$
  pois dados $i,j\in N$, temos
  $$
    C_i\cap D_j=A_{2i}\cap A_{2j+1}\setminus\bigl(A_{2i+1}\cup A_{2(j+1)}\bigr)\subset A_a\setminus A_b=\emptyset,
  $$
  sendo
  $$
    a=\max\{2i, 2j\}\quad\hbox{e}\quad b=\min\{2(i+1), 2(j+1)\}.
  $$
  uma vez que $A_b\subset A_a$, como consequ\^encia $C\cap D=\emptyset$.

  Seguidamente, dados $i,j\in N$, obtem-se
  $$
    A\cap (C_i\cup D_j)=A\setminus\bigl(A_{2i+1}\cap A_{2(j+1)}\bigr)=\emptyset,
  $$
  consequentemente $A\cap(C\cup D)=\emptyset$.

  Afirmo que
  $$
    X=A\cup C\cup D
  $$
  Decerto, observe que se $x\in X\setminus(C\cup D)$, ent\~ao necessariamente $x\in A_i$ para todo $i\in N$, pois, se existir $i\in N$ tal que $x\notin A_i$, ent\~ao consideremos $m=\min\{i\in N:x\notin A_i\}$, \'e certo que $m>0$. Assim, existe $l\in N$, tal que $l+1=m$, e consequentemente $x\in A_l\setminus A_{l+1}\subset C\cup D$, uma contradi\c c\~ao.

  De maneira inteiramente an\'aloga para cada $i\in N$, fa\c camos:
  $$
    B_{2i}=(f\!g)^i(Y)\quad\hbox{e}\quad B_{2i+1}=(f\!g)^i\bigl(f(X)\bigr);
  $$
  $$
    E_i=(f\!g)^i\bigr(Y\setminus f(X)\bigr)\quad\hbox{e}\quad F_i=(f\!g)^if\bigl(X\setminus g(Y)\bigr).
  $$
  Prova-se sem dificuldades que
  $$
    Y=B\cup E\cup F,
  $$
  em que
  $$
    B=\Cap_{i\in N}B_i,\quad E=\Cup_{i\in N}E_i\quad e\quad F=\Cup_{i\in N}F_i.
  $$
  Em verdade, basta trocar os pap\'eis de $f$ com $g$ e $X$ com $Y$, concomitamente.

  Agora observemos que por indu\c c\~ao matem\'atica decorre que
  $$
    \forall i\bigl(i\in N\longrightarrow \phi(\gamma\phi)^i=(\phi\gamma)^i\phi\bigr).
  $$
  para quaisquer que sejam as fun\c c\~oes $\phi$ e $\gamma$, em que a composi\c c\~ao fa\c ca sentido. Em verdade, para $i=0$ temos necessariamente
  $$
    \phi(\gamma\phi)^0=\phi=(\phi\gamma)^i\phi.
  $$
  Suponhamos por hip\'otese de indu\c c\~ao que a propriedade seja v\'alida para $i\in N$, notemos que
  $$
    \phi(\gamma\phi)^{i+1}=\phi(\gamma\phi)^i(\gamma\phi)=(\phi\gamma)^i(\phi\gamma)\phi=(\phi\gamma)^{i+1}\phi
  $$
  o que prova que \'e v\'alida para $i+1$, conclu\'\i mos por indu\c c\~ao matem\'atica que a propriedade \'e v\'alida para todo $i\in N$.

  Posteriormente, em vista do resultado anterior observemos que para cada $i\in N$
  $$
    A_{2i+1}=(g\!f)^ig(Y)=g(f\!g)^i(Y)=g(B_{2i})
  $$
  e
  $$
    B_{2i+1}=(f\!g)^if(X)=f(g\!f)^i(X)=f(A_{2i}).
  $$
  Em consequ\^encia para todo $i\in N$
  $$
    B_{2i}=g^{-1}(A_{2i+1})\quad\hbox{e}\quad B_{2i+1}=f(A_{2i}).
  $$
  Em seguida, definamos $\varphi:X\rightarrow Y$ por

  $$
    \varphi(x)=
    \left\{
    \begin{matrix}
      \hfill f(x),      & x\in A\cup C;\hfill   \cr
      \hfill g^{-1}(x), & x\in D.\hfill \cr
    \end{matrix}
    \right.
  $$
  Pela pr\'opria defini\c c\~ao $\varphi$ \'e injetiva.

  Imediatamente observe que para todo $i\in N$
  $$
    \varphi(C_i)=f\Bigl((g\!f)^i\bigl(X\setminus g(Y)\bigr)\Bigr)=(f\!g)^if\bigl(X\setminus g(Y)\bigr)=F_i
  $$
  e
  $$
    \varphi(D_i)=g^{-1}\Bigl((g\!f)^ig\bigl(Y\setminus f(X)\bigr)\Bigr)=(f\!g)^i\bigl(Y\setminus f(X)\bigr)=E_i.
  $$
  Al\'em do mais, para todo $i\in N$, temos
  $$
    \varphi(A_{2i})=f(A_{2i})=f(g\!f)^i(X)=(f\!g)^if(X)=B_{2i+1}
  $$
  e
  $$
    \varphi(A_{2i+1})=f(A_{2i+1})=f(g\!f)^ig(Y)=(f\!g)^if\!g(Y)=B_{2(i+1)}.
  $$
  Em suma,
  $$
    \forall i \bigl(i\in N\longrightarrow\varphi(A_i)=B_{i+1}\wedge\varphi(C_i)=F_i\wedge\varphi(D_i)=E_i\bigr).
  $$
  Conformemente
  $$
    \varphi(A)=B,\quad\varphi(C)=F\quad\hbox{e}\quad\varphi(D)=E.
  $$
  {\it a fortiori}
  $$
    \varphi(X)=\varphi(A\cup C\cup D)=\varphi(A)\cup\varphi(C)\cup\varphi(D)=B\cup E\cup F=Y,
  $$
  i.e., $X\sim Y$.\qed

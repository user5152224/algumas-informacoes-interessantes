%!tex

\input ../macros.tex

\section{FUN\c C\~OES}

\proc{Teorema}{(67 Theorem)}{$\dom(\U)=\U$ e $\img(\U)=\U$}. 

\prova Note que para todo $x\in\U$, temos que $(x,x)\in\U$. \qed

\proc{Defini\c c\~ao}{(68 Definition)}{$f(x)=\bigcap\{y:(x,y)\in f\}$.}

A classe $f(x)$ \'e o {\it valor} de $f$ em $x$ ou a {\it imagem} de $x$ sob $f$. \'E importante ser observado que $x$ pode ser entendido como elemento do $\dom(f)$, e como classe. Vale deixar expl\'{\i}cito que $f(x)$ \'e sempre interpretado na primeira acep\c c\~ao e n\~ao na segunda, ou seja

$$
  f(x)\neq\{y:\exists z(z\in x\,\a\, (z,y)\in f)\}.
$$

\proc{Teorema}{(69 Theorem)}{Se $x\notin\dom(f)$, ent\~ao $f(x)=\U$; se $x\in\dom(f)$, ent\~ao $f(x)\in\U$.}

\prova Se $x\notin\dom(f)$, ent\~ao ${y:(x,y)\in f}=\emptyset$, consequentemente 
$$
  f(x)=\bigcap\{y:(x,y)\in f\}=\bigcap\emptyset=\U.
$$
Agora se $x\in\dom(f)$, ent\~ao $\{y:(x,y)\in f\}\neq\emptyset$, pelo {\bf 35 Theorem} conclu\'\i mos que 
$$
f(x)=\bigcap\{y:(x,y)\in f\}\in\U.
$$
\qed

\proc{Teorema}{(70 Theorem)}{%
  Se $f$ \'e uma fun\c c\~ao, ent\~ao $f=\{(x,y):y=f(x)\}$
}

\prova Se $f$ \'e uma fun\c c\~ao ent\~ao para todo $x\in\dom(f)$, existe um $y$ tal que $(x,y)\in f$. N\~ao obstante, $(x,z)\in f$ se, e somente se, $z=y$. Consequentemente, 
$$
\{z:(x,z)\in f\}=\{z:z=y\}=\{y\}
$$
da\'\i
$$
  f(x)=\bigcap\{z:(x,z)\in f\}=\bigcap\{y\}=y.
$$
Assim, para todo $x\in\dom(f)$, vale $(x,f(x))\in f$, particularmente 
$$
  f=\{(x,y):y=f(x)\}.
$$
\qed

Imediatamente temos o

\proc{Teorema}{(71 Theorem\kern 1.5pt\raise 1pt\hbox{\vcmr *})}{%
  Se $f$ e $g$ s\~ao fun\c c\~oes, ent\~ao $f(x)=g(x)$, para todo $x$.
}

\prova Imediata ao {\spft Theorem 70}.\qed

\proc{Axioma}{(V Axiom of substitution)}{%
  Se $f$ \'e uma fun\c c\~ao e $\dom(f)\in\U$, ent\~ao $\img(f)\in\U$.
}

\proc{Axioma}{(VI Axioma of amalgamation)}{%
  Se $x\in U$, ent\~ao $\bigcup x\in\U$.
}

\proc{Defini\c c\~ao}{(72 Definition)}{%
  $x\times y=\{(u,v):u\in x\,\a\,v\in y\}$.
}

A classe $x\times y$ \'e chamanda de produto cartesiano de $x$ e $y$.

\proc{Teorema}{(73 Theorem)}{%
  Se $u,y\in\U$, ent\~ao $\{u\}\times y\in\U$.
}

\prova Construamos a seguinte fun\c c\~ao
$$
  f=\{(x,(u,x)):x\in y\}.
$$
Como $\dom(f)=x\in\U$, pelo axioma {\spft V} conclu\'\i mos 
$$
  \{u\}\times y=\{(u,x):x\in y\}=\img(f)\in\U.
$$
\qed

\proc{Axioma}{(V-VI Axiom)}{%
  Se $\dom(f)\in\U$, ent\~ao $\bigcup\img(f)\in\U$.
}

Este \'ultimo axioma \'e a s\'{\i}ntese dos axiomas {\spft V} e {\spft VI}, isto \'e elucidado pelo seguinte

\proc{Teorema}{}{%
  Os axiomas {\spft V} e {\spft VI} s\~ao equivalentes ao axioma {\spft V-VI}.
}

\prova Adimitamos que $\dom(f)\in\U$, e os axiomas {\spft V} e {\spft VI}, segue naturalmente $\img(f)\in U$ por {\spft V} e $\bigcup\img(f)\in\U$ por {\spft VI}. 

Reciprocamente, suponhamos que $\dom(f)\in\U$, pelo axioma {\spft V-VI}, segue-se que $x=\bigcup\img(f)\in\U$, logo $2^x\in\U$, como $\img(f)\subset 2^x$, pelo {\spft III Axiom of subsets}, incorremos que $\img(f)\in\U$ o que prova {\spft V}. Ademais, suponhamos que $x\in\U$, e considemos $i=\{(u,u):u\in x\}$, por {\spft V} segue-se que $\bigcup x=\bigcup\img(i)\in\U$, o que prova {\spft VI}.\qed

\proc{Teorema}{(74 Theorem)}{%
  Se $x,y\in\U$, ent\~ao $x\times y\in\U$.
}

\prova Provar.\qed


\bye

%!tex

\input ../macros.tex

\proc{Teorema}{(67 Theorem)} $\D(\U)=\U$ e $I(\U)=\U$. 

\prova Note que para todo $x\in\U$, temos que $(x,x)\in\U$. \qed

\proc{Defini\c c\~ao}{(68 Definition)} $f(x)=\bigcap\{y:(x,y)\in f\}$.

A classe $f(x)$ \'e o {\it valor} de $f$ em $x$ ou a {\it imagem} de $x$ sob $f$. \'E importante ser observado que $x$ pode ser entendido como elemento do $\D(f)$, e como classe. Vale deixar expl\'{\i}cito que $f(x)$ \'e sempre interpretado na primeira acep\c c\~ao e n\~ao na segunda, ou seja

$$
  f(x)\neq\bigl\{y:\exists z(z\in x\,\a\, (z,y)\in f)\bigr\}
$$

\bye
